\documentclass[]{book}
\usepackage{lmodern}
\usepackage{amssymb,amsmath}
\usepackage{ifxetex,ifluatex}
\usepackage{fixltx2e} % provides \textsubscript
\ifnum 0\ifxetex 1\fi\ifluatex 1\fi=0 % if pdftex
  \usepackage[T1]{fontenc}
  \usepackage[utf8]{inputenc}
\else % if luatex or xelatex
  \ifxetex
    \usepackage{mathspec}
  \else
    \usepackage{fontspec}
  \fi
  \defaultfontfeatures{Ligatures=TeX,Scale=MatchLowercase}
\fi
% use upquote if available, for straight quotes in verbatim environments
\IfFileExists{upquote.sty}{\usepackage{upquote}}{}
% use microtype if available
\IfFileExists{microtype.sty}{%
\usepackage{microtype}
\UseMicrotypeSet[protrusion]{basicmath} % disable protrusion for tt fonts
}{}
\usepackage[margin=1in]{geometry}
\usepackage{hyperref}
\hypersetup{unicode=true,
            pdftitle={Aprender R: iniciación y perfeccionamiento},
            pdfauthor={François Rebaudo},
            pdfborder={0 0 0},
            breaklinks=true}
\urlstyle{same}  % don't use monospace font for urls
\usepackage{natbib}
\bibliographystyle{apalike}
\usepackage{color}
\usepackage{fancyvrb}
\newcommand{\VerbBar}{|}
\newcommand{\VERB}{\Verb[commandchars=\\\{\}]}
\DefineVerbatimEnvironment{Highlighting}{Verbatim}{commandchars=\\\{\}}
% Add ',fontsize=\small' for more characters per line
\usepackage{framed}
\definecolor{shadecolor}{RGB}{248,248,248}
\newenvironment{Shaded}{\begin{snugshade}}{\end{snugshade}}
\newcommand{\KeywordTok}[1]{\textcolor[rgb]{0.13,0.29,0.53}{\textbf{#1}}}
\newcommand{\DataTypeTok}[1]{\textcolor[rgb]{0.13,0.29,0.53}{#1}}
\newcommand{\DecValTok}[1]{\textcolor[rgb]{0.00,0.00,0.81}{#1}}
\newcommand{\BaseNTok}[1]{\textcolor[rgb]{0.00,0.00,0.81}{#1}}
\newcommand{\FloatTok}[1]{\textcolor[rgb]{0.00,0.00,0.81}{#1}}
\newcommand{\ConstantTok}[1]{\textcolor[rgb]{0.00,0.00,0.00}{#1}}
\newcommand{\CharTok}[1]{\textcolor[rgb]{0.31,0.60,0.02}{#1}}
\newcommand{\SpecialCharTok}[1]{\textcolor[rgb]{0.00,0.00,0.00}{#1}}
\newcommand{\StringTok}[1]{\textcolor[rgb]{0.31,0.60,0.02}{#1}}
\newcommand{\VerbatimStringTok}[1]{\textcolor[rgb]{0.31,0.60,0.02}{#1}}
\newcommand{\SpecialStringTok}[1]{\textcolor[rgb]{0.31,0.60,0.02}{#1}}
\newcommand{\ImportTok}[1]{#1}
\newcommand{\CommentTok}[1]{\textcolor[rgb]{0.56,0.35,0.01}{\textit{#1}}}
\newcommand{\DocumentationTok}[1]{\textcolor[rgb]{0.56,0.35,0.01}{\textbf{\textit{#1}}}}
\newcommand{\AnnotationTok}[1]{\textcolor[rgb]{0.56,0.35,0.01}{\textbf{\textit{#1}}}}
\newcommand{\CommentVarTok}[1]{\textcolor[rgb]{0.56,0.35,0.01}{\textbf{\textit{#1}}}}
\newcommand{\OtherTok}[1]{\textcolor[rgb]{0.56,0.35,0.01}{#1}}
\newcommand{\FunctionTok}[1]{\textcolor[rgb]{0.00,0.00,0.00}{#1}}
\newcommand{\VariableTok}[1]{\textcolor[rgb]{0.00,0.00,0.00}{#1}}
\newcommand{\ControlFlowTok}[1]{\textcolor[rgb]{0.13,0.29,0.53}{\textbf{#1}}}
\newcommand{\OperatorTok}[1]{\textcolor[rgb]{0.81,0.36,0.00}{\textbf{#1}}}
\newcommand{\BuiltInTok}[1]{#1}
\newcommand{\ExtensionTok}[1]{#1}
\newcommand{\PreprocessorTok}[1]{\textcolor[rgb]{0.56,0.35,0.01}{\textit{#1}}}
\newcommand{\AttributeTok}[1]{\textcolor[rgb]{0.77,0.63,0.00}{#1}}
\newcommand{\RegionMarkerTok}[1]{#1}
\newcommand{\InformationTok}[1]{\textcolor[rgb]{0.56,0.35,0.01}{\textbf{\textit{#1}}}}
\newcommand{\WarningTok}[1]{\textcolor[rgb]{0.56,0.35,0.01}{\textbf{\textit{#1}}}}
\newcommand{\AlertTok}[1]{\textcolor[rgb]{0.94,0.16,0.16}{#1}}
\newcommand{\ErrorTok}[1]{\textcolor[rgb]{0.64,0.00,0.00}{\textbf{#1}}}
\newcommand{\NormalTok}[1]{#1}
\usepackage{longtable,booktabs}
\usepackage{graphicx,grffile}
\makeatletter
\def\maxwidth{\ifdim\Gin@nat@width>\linewidth\linewidth\else\Gin@nat@width\fi}
\def\maxheight{\ifdim\Gin@nat@height>\textheight\textheight\else\Gin@nat@height\fi}
\makeatother
% Scale images if necessary, so that they will not overflow the page
% margins by default, and it is still possible to overwrite the defaults
% using explicit options in \includegraphics[width, height, ...]{}
\setkeys{Gin}{width=\maxwidth,height=\maxheight,keepaspectratio}
\IfFileExists{parskip.sty}{%
\usepackage{parskip}
}{% else
\setlength{\parindent}{0pt}
\setlength{\parskip}{6pt plus 2pt minus 1pt}
}
\setlength{\emergencystretch}{3em}  % prevent overfull lines
\providecommand{\tightlist}{%
  \setlength{\itemsep}{0pt}\setlength{\parskip}{0pt}}
\setcounter{secnumdepth}{5}
% Redefines (sub)paragraphs to behave more like sections
\ifx\paragraph\undefined\else
\let\oldparagraph\paragraph
\renewcommand{\paragraph}[1]{\oldparagraph{#1}\mbox{}}
\fi
\ifx\subparagraph\undefined\else
\let\oldsubparagraph\subparagraph
\renewcommand{\subparagraph}[1]{\oldsubparagraph{#1}\mbox{}}
\fi

%%% Use protect on footnotes to avoid problems with footnotes in titles
\let\rmarkdownfootnote\footnote%
\def\footnote{\protect\rmarkdownfootnote}

%%% Change title format to be more compact
\usepackage{titling}

% Create subtitle command for use in maketitle
\newcommand{\subtitle}[1]{
  \posttitle{
    \begin{center}\large#1\end{center}
    }
}

\setlength{\droptitle}{-2em}

  \title{Aprender R: iniciación y perfeccionamiento}
    \pretitle{\vspace{\droptitle}\centering\huge}
  \posttitle{\par}
    \author{François Rebaudo}
    \preauthor{\centering\large\emph}
  \postauthor{\par}
      \predate{\centering\large\emph}
  \postdate{\par}
    \date{2018-07-27}

\usepackage{booktabs}
\usepackage{longtable}
\usepackage[bf,singlelinecheck=off]{caption}

\setmainfont[UprightFeatures={SmallCapsFont=Arial}]{Arial} % AlegreyaSC-Regular % Alegreya

\usepackage{framed,color}
\definecolor{shadecolor}{RGB}{248,248,248}

\renewcommand{\textfraction}{0.05}
\renewcommand{\topfraction}{0.8}
\renewcommand{\bottomfraction}{0.8}
\renewcommand{\floatpagefraction}{0.75}

\renewenvironment{quote}{\begin{VF}}{\end{VF}}
\let\oldhref\href
\renewcommand{\href}[2]{#2\footnote{\url{#1}}}

\ifxetex
  \usepackage{letltxmacro}
  \setlength{\XeTeXLinkMargin}{1pt}
  \LetLtxMacro\SavedIncludeGraphics\includegraphics
  \def\includegraphics#1#{% #1 catches optional stuff (star/opt. arg.)
    \IncludeGraphicsAux{#1}%
  }%
  \newcommand*{\IncludeGraphicsAux}[2]{%
    \XeTeXLinkBox{%
      \SavedIncludeGraphics#1{#2}%
    }%
  }%
\fi

\makeatletter
\newenvironment{kframe}{%
\medskip{}
\setlength{\fboxsep}{.8em}
 \def\at@end@of@kframe{}%
 \ifinner\ifhmode%
  \def\at@end@of@kframe{\end{minipage}}%
  \begin{minipage}{\columnwidth}%
 \fi\fi%
 \def\FrameCommand##1{\hskip\@totalleftmargin \hskip-\fboxsep
 \colorbox{shadecolor}{##1}\hskip-\fboxsep
     % There is no \\@totalrightmargin, so:
     \hskip-\linewidth \hskip-\@totalleftmargin \hskip\columnwidth}%
 \MakeFramed {\advance\hsize-\width
   \@totalleftmargin\z@ \linewidth\hsize
   \@setminipage}}%
 {\par\unskip\endMakeFramed%
 \at@end@of@kframe}
\makeatother

\makeatletter
\@ifundefined{Shaded}{
}{\renewenvironment{Shaded}{\begin{kframe}}{\end{kframe}}}
\makeatother

\newenvironment{rmdblock}[1]
  {
  \begin{itemize}
  \renewcommand{\labelitemi}{
    \raisebox{-.7\height}[0pt][0pt]{
      {\setkeys{Gin}{width=3em,keepaspectratio}\includegraphics{myIcons/#1}} %FR
    }
  }
  \setlength{\fboxsep}{1em}
  \begin{kframe}
  \item
  }
  {
  \end{kframe}
  \end{itemize}
  }
\newenvironment{rmdnote}      %FR
  {\begin{rmdblock}{note}}    %FR
  {\end{rmdblock}}            %FR
\newenvironment{rmdstyle}     %FR
  {\begin{rmdblock}{style}}   %FR
  {\end{rmdblock}}            %FR
\newenvironment{rmdcaution}
  {\begin{rmdblock}{caution}}
  {\end{rmdblock}}
\newenvironment{rmdimportant}
  {\begin{rmdblock}{important}}
  {\end{rmdblock}}
\newenvironment{rmdtip}
  {\begin{rmdblock}{tip}}
  {\end{rmdblock}}
\newenvironment{rmdwarning}
  {\begin{rmdblock}{warning}}
  {\end{rmdblock}}

\usepackage{makeidx}
\makeindex

\urlstyle{tt}

\usepackage{amsthm}
\makeatletter
\def\thm@space@setup{%
  \thm@preskip=8pt plus 2pt minus 4pt
  \thm@postskip=\thm@preskip
}
\makeatother

\mainmatter

\begin{document}
\maketitle

{
\setcounter{tocdepth}{1}
\tableofcontents
}
\chapter{Preámbulo}\label{preambulo}

Este libro está incompleto por el momento y está leyendo su versión
preliminar. Si tiene algún comentario, sugerencia o si identifica
errores, no dude en enviarme un correo electrónico
(\href{mailto:francois.rebaudo@ird.fr}{\nolinkurl{francois.rebaudo@ird.fr}}),
o si está familiarizado con GitHub en el sitio web del proyecto
(\url{https://github.com/frareb/myRBook_SP}).

Últimas modificaciones:

\textbf{27/07/2018}

\begin{itemize}
\tightlist
\item
  Capítulo sobre las funciones (2/3)
\end{itemize}

\textbf{25/07/2018}

\begin{itemize}
\tightlist
\item
  Capítulo sobre las funciones (1/3)
\end{itemize}

\textbf{17/07/2018}

\begin{itemize}
\tightlist
\item
  edición y corrección del español (Susi L.)
\item
  tercera parte del capítulo \emph{Contenedores de datos}: El contenedor
  \emph{data.frame}
\item
  cuarta parte del capítulo \emph{Contenedores de datos}: El contenedor
  \emph{matrix}
\item
  quinta parte del capítulo \emph{Contenedores de datos}: El contenedor
  \emph{array}
\end{itemize}

\textbf{16/07/2018}

\begin{itemize}
\tightlist
\item
  edición y corrección del español (Estefanía Q.)
\end{itemize}

\textbf{12/07/2018}

\begin{itemize}
\tightlist
\item
  segunda parte del capítulo \emph{Contenedores de datos}: El contenedor
  \emph{list}
\end{itemize}

\textbf{06/07/2018}

\begin{itemize}
\tightlist
\item
  edición y corrección del español (Camila B.)
\item
  primera parte del capítulo \emph{Contenedores de datos}: El contenedor
  \emph{vector}
\end{itemize}

\textbf{04/07/2018}

\begin{itemize}
\tightlist
\item
  tabla de contenidos con los próximos capítulos
\item
  error de tipografía en \emph{Elegir un entorno de desarrollo}
\end{itemize}

\textbf{02/07/2018}

\begin{itemize}
\tightlist
\item
  tres capítulos en línea (\emph{primeros pasos}, \emph{elegir un
  entorno de desarrollo}, \emph{tipos de datos})
\end{itemize}

\chapter{Agradecimientos}\label{remerciements}

Agradezco a todos los colaboradores que ayudaron a mejorar este libro
con sus consejos, sugerencias de cambios y correcciones (en orden
alfabético):

\begin{verbatim}
## Colaboradores :
Camila Benavides Frias (Bolivia)
Susi Loza Herrera (Bolivia)
Estefania Quenta Herrera (Bolivia)
\end{verbatim}

Las versiones de gitbook, html y epub de este libro usan los iconos de
fuente abierta de Font Awesome (\url{https://fontawesome.com}). La
versión en PDF utiliza los iconos del proyecto Tango disponibles en
openclipart (\url{https://openclipart.org/}). Este libro fue escrito con
el paquete R bookdown (\url{https://bookdown.org/}). El código fuente
está disponible en GitHub (\url{https://github.com/frareb/myRBook_SP}).
La compilación usa Travis CI (\url{https://travis-ci.org}). La versión
en línea se aloja y actualiza a través de Netlify
(\url{http://myrbooksp.netlify.com/}).

\chapter{Licencia}\label{licence}

Licencia Reconocimiento-NoComercial-SinObraDerivada 3.0 España (CC
BY-NC-ND 3.0 ES ;
\url{https://creativecommons.org/licenses/by-nc-nd/3.0/es/})

Esto es un resumen inteligible para humanos (y no un sustituto) de la
licencia.

\textbf{Usted es libre de:}

\begin{itemize}
\tightlist
\item
  Compartir --- copiar y redistribuir el material en cualquier medio o
  formato.
\item
  El licenciador no puede revocar estas libertades mientras cumpla con
  los términos de la licencia.
\end{itemize}

\textbf{Bajo las condiciones siguientes:}

\begin{itemize}
\item
  Reconocimiento --- Debe reconocer adecuadamente la autoría,
  proporcionar un enlace a la licencia e indicar si se han realizado
  cambios\textless{}. Puede hacerlo de cualquier manera razonable, pero
  no de una manera que sugiera que tiene el apoyo del licenciador o lo
  recibe por el uso que hace.
\item
  NoComercial --- No puede utilizar el material para una finalidad
  comercial.
\item
  SinObraDerivada --- Si remezcla, transforma o crea a partir del
  material, no puede difundir el material modificado.
\item
  No hay restricciones adicionales --- No puede aplicar términos legales
  o medidas tecnológicas que legalmente restrinjan realizar aquello que
  la licencia permite.
\end{itemize}

\textbf{Avisos: }

No tiene que cumplir con la licencia para aquellos elementos del
material en el dominio público o cuando su utilización esté permitida
por la aplicación de una excepción o un límite. No se dan garantías. La
licencia puede no ofrecer todos los permisos necesarios para la
utilización prevista. Por ejemplo, otros derechos como los de
publicidad, privacidad, o los derechos morales pueden limitar el uso del
material.

\chapter{Introducción}\label{intro}

\section{¿Por qué aprender R?}\label{por-que-aprender-r}

\section{Este libro}\label{este-libro}

\section{Lectura adicional en
español}\label{lectura-adicional-en-espanol}

\begin{itemize}
\tightlist
\item
  R para Principiantes, Emmanuel Paradis
  (\url{https://cran.r-project.org/doc/contrib/rdebuts_es.pdf})
\item
  xxx
\end{itemize}

\part{Conceptos básicos}\label{part-conceptos-basicos}

\chapter{Primeros pasos}\label{premiersPas}

\section{Instalar R}\label{instalar-r}

El programa para instalar el software R se puede descargar desde el
sitio web de R: \url{https://www.r-project.org/}. En el sitio web de R,
primero es necesario elegir un espejo CRAN (servidor desde el cual se
debe descargar R, y desde el más cercano a su ubicación geográfica),
luego descargue el archivo \emph{base}. Los usuarios de Linux pueden
preferir un \texttt{sudo\ apt-get\ install\ r-base}.

\begin{rmdnote}
El software R se puede descargar de muchos servidores CRAN
(Comprehensive R Archive Network) de todo el mundo. Estos servidores se
llaman espejos. La elección del espejo es manual. La información
adicional como esta nota siempre estará representada con este pictograma
\emph{información}.
\end{rmdnote}

\section{R como calculadora}\label{r-como-calculadora}

Una vez que se inicia el programa, aparece una ventana cuya apariencia
puede variar dependiendo de su sistema operativo (Figura
\ref{fig:screenCapConsole}). Esta ventana se llama \emph{consola}.

\begin{figure}
\includegraphics[width=7.53in]{myFigures/screencap_rConsoleFR} \caption{Captura de pantalla de la consola R en Windows.\label{fig:screenCapConsole}}\label{fig:screenCapConsole}
\end{figure}

La consola corresponde a la interfaz donde se interpretará el código, es
decir, donde el código será transformado en lenguaje de máquina,
ejecutado por la computadora y retransmitido en forma legible por
humanos. Esto es análogo a lo que sucede en una calculadora (Figura
\ref{fig:screenCapConsoleCal}). Así es como se usará R más adelante en
esta sección.

\begin{rmdnote}
A lo largo de este libro, los ejemplos del código R aparecerán sobre un
fondo gris. Se pueden copiar y pegar directamente en la consola, aunque
es mejor reproducir los ejemplos escribiéndolos en la consola (o más
adelante en los scripts) para una mejor comprensión del manejo del
programa R. El resultado de lo que se envía en la consola también
aparecerá en un fondo gris con \texttt{\#\#} delante del código para
hacer la distinción entre el código y el resultado del código.
\end{rmdnote}

\begin{figure}
\includegraphics[width=6.71in]{myFigures/screencap_rConsoleCalculatrice} \caption{Captura de pantalla de la consola R al lado de la calculadora de Windows.\label{fig:screenCapConsoleCal}}\label{fig:screenCapConsoleCal}
\end{figure}

\hypertarget{l011opari}{\subsection{Los operadores
aritméticos}\label{l011opari}}

\begin{Shaded}
\begin{Highlighting}[]
\DecValTok{5} \OperatorTok{+}\StringTok{ }\DecValTok{5}
\end{Highlighting}
\end{Shaded}

\begin{verbatim}
## [1] 10
\end{verbatim}

Si escribimos \texttt{5\ +\ 5} en la consola y luego \texttt{Enter}, el
resultado aparece precedido por el número {[}1{]} entre corchetes. Este
número corresponde al número del resultado (en nuestro caso, solo hay un
resultado, volveremos a este aspecto más adelante). También podemos
observar en este ejemplo el uso de espacios antes y después del signo
\texttt{+}. Estos espacios no son necesarios, pero permiten que el
código sea más legible para los humanos (es decir, más agradable de leer
tanto para nosotros como para las personas con las que queremos
compartir nuestro código). Los operadores aritméticos disponibles en R
se resumen en la tabla \ref{tab:tabOpAri}.

\begin{table}

\caption{\label{tab:tabOpAri}Operadores aritméticos.\label{tab:tabOpAri}}
\centering
\begin{tabular}[t]{l|l}
\hline
Label & Operador\\
\hline
adición & +\\
\hline
resta & -\\
\hline
multiplicación & *\\
\hline
división & /\\
\hline
potencia & \textasciicircum{}\\
\hline
módulo & \%\%\\
\hline
cociente decimal & \%/\%\\
\hline
\end{tabular}
\end{table}

Clásicamente, las multiplicaciones y divisiones tienen prioridad sobre
las adiciones y sustracciones. Si es necesario, podemos usar paréntesis.

\begin{Shaded}
\begin{Highlighting}[]
\DecValTok{5} \OperatorTok{+}\StringTok{ }\DecValTok{5} \OperatorTok{*}\StringTok{ }\DecValTok{2}
\end{Highlighting}
\end{Shaded}

\begin{verbatim}
## [1] 15
\end{verbatim}

\begin{Shaded}
\begin{Highlighting}[]
\NormalTok{(}\DecValTok{5} \OperatorTok{+}\StringTok{ }\DecValTok{5}\NormalTok{) }\OperatorTok{*}\StringTok{ }\DecValTok{2}
\end{Highlighting}
\end{Shaded}

\begin{verbatim}
## [1] 20
\end{verbatim}

\begin{Shaded}
\begin{Highlighting}[]
\NormalTok{(}\DecValTok{5} \OperatorTok{+}\StringTok{ }\DecValTok{5}\NormalTok{) }\OperatorTok{*}\StringTok{ }\NormalTok{(}\DecValTok{2} \OperatorTok{+}\StringTok{ }\DecValTok{2}\NormalTok{)}
\end{Highlighting}
\end{Shaded}

\begin{verbatim}
## [1] 40
\end{verbatim}

\begin{Shaded}
\begin{Highlighting}[]
\NormalTok{(}\DecValTok{5} \OperatorTok{+}\StringTok{ }\DecValTok{5}\NormalTok{) }\OperatorTok{*}\StringTok{ }\NormalTok{((}\DecValTok{2} \OperatorTok{+}\StringTok{ }\DecValTok{2}\NormalTok{) }\OperatorTok{/}\StringTok{ }\DecValTok{3}\NormalTok{)}\OperatorTok{^}\DecValTok{2}
\end{Highlighting}
\end{Shaded}

\begin{verbatim}
## [1] 17.77778
\end{verbatim}

El operador \emph{módulo} corresponde al resto de la división
euclidiana. Se usa en ciencias de la computación, por ejemplo, para
saber si un número es par o impar (un número módulo 2 devolverá 1 si es
impar y 0 si es par).

\begin{Shaded}
\begin{Highlighting}[]
\DecValTok{451} \OperatorTok\StringTok{ }\DecValTok{2}
\end{Highlighting}
\end{Shaded}

\begin{verbatim}
## [1] 1
\end{verbatim}

\begin{Shaded}
\begin{Highlighting}[]
\DecValTok{288} \OperatorTok\StringTok{ }\DecValTok{2}
\end{Highlighting}
\end{Shaded}

\begin{verbatim}
## [1] 0
\end{verbatim}

\begin{Shaded}
\begin{Highlighting}[]
\NormalTok{(}\DecValTok{5} \OperatorTok{+}\StringTok{ }\DecValTok{5} \OperatorTok{*}\StringTok{ }\DecValTok{2}\NormalTok{) }\OperatorTok\StringTok{ }\DecValTok{2}
\end{Highlighting}
\end{Shaded}

\begin{verbatim}
## [1] 1
\end{verbatim}

\begin{Shaded}
\begin{Highlighting}[]
\NormalTok{((}\DecValTok{5} \OperatorTok{+}\StringTok{ }\DecValTok{5}\NormalTok{) }\OperatorTok{*}\StringTok{ }\DecValTok{2}\NormalTok{) }\OperatorTok\StringTok{ }\DecValTok{2}
\end{Highlighting}
\end{Shaded}

\begin{verbatim}
## [1] 0
\end{verbatim}

R también incorpora algunas constantes que incluyen \texttt{pi}. Además,
el signo infinito está representado por \texttt{Inf}.

\begin{Shaded}
\begin{Highlighting}[]
\NormalTok{pi}
\end{Highlighting}
\end{Shaded}

\begin{verbatim}
## [1] 3.141593
\end{verbatim}

\begin{Shaded}
\begin{Highlighting}[]
\NormalTok{pi }\OperatorTok{*}\StringTok{ }\DecValTok{5}\OperatorTok{^}\DecValTok{2}
\end{Highlighting}
\end{Shaded}

\begin{verbatim}
## [1] 78.53982
\end{verbatim}

\begin{Shaded}
\begin{Highlighting}[]
\DecValTok{1}\OperatorTok{/}\DecValTok{0}
\end{Highlighting}
\end{Shaded}

\begin{verbatim}
## [1] Inf
\end{verbatim}

\begin{rmdstyle}
el \emph{estilo} del código es importante porque el código está
destinado a ser leído por nosotros y por otras personas. Para tener un
estilo legible, se recomienda colocar espacios antes y después de los
operadores aritméticos, excepto ``*'', ``/'' y ``\^{}'', aunque a veces
es útil agregarlos como es el caso en nuestro ejemplos. La información
sobre el \emph{estilo} siempre estará representada con este pictograma
para que sea fácilmente identificable.
\end{rmdstyle}

\hypertarget{l011opcomp}{\subsection{Los operadores
comparativos}\label{l011opcomp}}

Sin embargo, R es mucho más que una simple calculadora porque permite
otro tipo de operadores: operadores de comparación, para \emph{comparar}
los valores (Table \ref{tab:tabOpCom}).

\begin{table}

\caption{\label{tab:tabOpCom}Operadores de comparación.\label{tab:tabOpCom}}
\centering
\begin{tabular}[t]{l|l}
\hline
Label & Operador\\
\hline
más pequeño que & <\\
\hline
mayor que & >\\
\hline
más pequeño o igual a & <=\\
\hline
más grande o igual a & >=\\
\hline
igual a & ==\\
\hline
diferente de & !=\\
\hline
\end{tabular}
\end{table}

Por ejemplo, si queremos saber si un numero es más grande que otro,
podemos escribir:

\begin{Shaded}
\begin{Highlighting}[]
\DecValTok{5} \OperatorTok{>}\StringTok{ }\DecValTok{3} 
\end{Highlighting}
\end{Shaded}

\begin{verbatim}
## [1] TRUE
\end{verbatim}

R devuelve \texttt{TRUE} si la comparación es verdadera y \texttt{FALSE}
si la comparación es falsa.

\begin{Shaded}
\begin{Highlighting}[]
\DecValTok{5} \OperatorTok{>}\StringTok{ }\DecValTok{3}
\end{Highlighting}
\end{Shaded}

\begin{verbatim}
## [1] TRUE
\end{verbatim}

\begin{Shaded}
\begin{Highlighting}[]
\DecValTok{2} \OperatorTok{<}\StringTok{ }\FloatTok{1.5}
\end{Highlighting}
\end{Shaded}

\begin{verbatim}
## [1] FALSE
\end{verbatim}

\begin{Shaded}
\begin{Highlighting}[]
\DecValTok{2} \OperatorTok{<=}\StringTok{ }\DecValTok{2}
\end{Highlighting}
\end{Shaded}

\begin{verbatim}
## [1] TRUE
\end{verbatim}

\begin{Shaded}
\begin{Highlighting}[]
\FloatTok{3.2} \OperatorTok{>=}\StringTok{ }\FloatTok{1.5}
\end{Highlighting}
\end{Shaded}

\begin{verbatim}
## [1] TRUE
\end{verbatim}

Podemos combinar operadores aritméticos con operadores de comparación.

\begin{Shaded}
\begin{Highlighting}[]
\NormalTok{(}\DecValTok{5} \OperatorTok{+}\StringTok{ }\DecValTok{8}\NormalTok{) }\OperatorTok{>}\StringTok{ }\NormalTok{(}\DecValTok{3} \OperatorTok{*}\StringTok{ }\DecValTok{45}\OperatorTok{/}\DecValTok{2}\NormalTok{) }
\end{Highlighting}
\end{Shaded}

\begin{verbatim}
## [1] FALSE
\end{verbatim}

\begin{rmdstyle}
En la comparación \texttt{(5\ +\ 8)\ \textgreater{}\ (3\ *\ 45/2)} no se
necesitan paréntesis, pero permiten que el código sea más fácil de leer.
\end{rmdstyle}

Un operador de comparación particular es \emph{igual a}. Veremos en la
siguiente sección que el signo \texttt{=} está reservado para otro uso:
permite asignar un valor a un objeto. El operador de comparación
\emph{igual a} debe ser diferente, por eso R usa \texttt{==}.

\begin{Shaded}
\begin{Highlighting}[]
\DecValTok{42} \OperatorTok{==}\StringTok{ }\DecValTok{53}
\end{Highlighting}
\end{Shaded}

\begin{verbatim}
## [1] FALSE
\end{verbatim}

\begin{Shaded}
\begin{Highlighting}[]
\DecValTok{58} \OperatorTok{==}\StringTok{ }\DecValTok{58}
\end{Highlighting}
\end{Shaded}

\begin{verbatim}
## [1] TRUE
\end{verbatim}

Otro operador particular es \emph{diferente de}. Se usa con \emph{un
signo de admiración} seguido de \emph{igual}, \texttt{!=}. Este operador
permite obtener la respuesta opuesta a \texttt{==}.

\begin{Shaded}
\begin{Highlighting}[]
\DecValTok{42} \OperatorTok{==}\StringTok{ }\DecValTok{53}
\end{Highlighting}
\end{Shaded}

\begin{verbatim}
## [1] FALSE
\end{verbatim}

\begin{Shaded}
\begin{Highlighting}[]
\DecValTok{42} \OperatorTok{!=}\StringTok{ }\DecValTok{53}
\end{Highlighting}
\end{Shaded}

\begin{verbatim}
## [1] TRUE
\end{verbatim}

\begin{Shaded}
\begin{Highlighting}[]
\NormalTok{(}\DecValTok{3} \OperatorTok{+}\StringTok{ }\DecValTok{2}\NormalTok{) }\OperatorTok{!=}\StringTok{ }\DecValTok{5}
\end{Highlighting}
\end{Shaded}

\begin{verbatim}
## [1] FALSE
\end{verbatim}

\begin{Shaded}
\begin{Highlighting}[]
\DecValTok{10}\OperatorTok{/}\DecValTok{2} \OperatorTok{==}\StringTok{ }\DecValTok{5}
\end{Highlighting}
\end{Shaded}

\begin{verbatim}
## [1] TRUE
\end{verbatim}

R usa \texttt{TRUE} y \texttt{FALSE}, que también son valores que se
pueden probar con operadores de comparación. Pero R también asigna un
valor a \texttt{TRUE} y \texttt{FALSE}:

\begin{Shaded}
\begin{Highlighting}[]
\OtherTok{TRUE} \OperatorTok{==}\StringTok{ }\OtherTok{TRUE}
\end{Highlighting}
\end{Shaded}

\begin{verbatim}
## [1] TRUE
\end{verbatim}

\begin{Shaded}
\begin{Highlighting}[]
\OtherTok{TRUE} \OperatorTok{>}\StringTok{ }\OtherTok{FALSE}
\end{Highlighting}
\end{Shaded}

\begin{verbatim}
## [1] TRUE
\end{verbatim}

\begin{Shaded}
\begin{Highlighting}[]
\DecValTok{1} \OperatorTok{==}\StringTok{ }\OtherTok{TRUE}
\end{Highlighting}
\end{Shaded}

\begin{verbatim}
## [1] TRUE
\end{verbatim}

\begin{Shaded}
\begin{Highlighting}[]
\DecValTok{0} \OperatorTok{==}\StringTok{ }\OtherTok{FALSE}
\end{Highlighting}
\end{Shaded}

\begin{verbatim}
## [1] TRUE
\end{verbatim}

\begin{Shaded}
\begin{Highlighting}[]
\OtherTok{TRUE} \OperatorTok{+}\StringTok{ }\DecValTok{1}
\end{Highlighting}
\end{Shaded}

\begin{verbatim}
## [1] 2
\end{verbatim}

\begin{Shaded}
\begin{Highlighting}[]
\OtherTok{FALSE} \OperatorTok{+}\StringTok{ }\DecValTok{1}
\end{Highlighting}
\end{Shaded}

\begin{verbatim}
## [1] 1
\end{verbatim}

\begin{Shaded}
\begin{Highlighting}[]
\NormalTok{(}\OtherTok{FALSE} \OperatorTok{+}\StringTok{ }\DecValTok{1}\NormalTok{) }\OperatorTok{==}\StringTok{ }\OtherTok{TRUE}
\end{Highlighting}
\end{Shaded}

\begin{verbatim}
## [1] TRUE
\end{verbatim}

El valor de \texttt{TRUE} es 1 y el valor de \texttt{FALSE} es 0.
Veremos más adelante cómo usar esta información en los próximos
capítulos.

R es también un lenguaje relativamente permisivo, significa que admite
cierta flexibilidad en la forma de escribir el código. Debatir sobre la
idoneidad de esta flexibilidad está fuera del alcance de este libro,
pero podemos encontrar en el código R en Internet o en otras obras el
atajo \texttt{T} para \texttt{TRUE} y \texttt{F} for \texttt{FALSE}.

\begin{Shaded}
\begin{Highlighting}[]
\NormalTok{T }\OperatorTok{==}\StringTok{ }\OtherTok{TRUE}
\end{Highlighting}
\end{Shaded}

\begin{verbatim}
## [1] TRUE
\end{verbatim}

\begin{Shaded}
\begin{Highlighting}[]
\NormalTok{F }\OperatorTok{==}\StringTok{ }\OtherTok{FALSE}
\end{Highlighting}
\end{Shaded}

\begin{verbatim}
## [1] TRUE
\end{verbatim}

\begin{Shaded}
\begin{Highlighting}[]
\NormalTok{T }\OperatorTok{==}\StringTok{ }\DecValTok{1}
\end{Highlighting}
\end{Shaded}

\begin{verbatim}
## [1] TRUE
\end{verbatim}

\begin{Shaded}
\begin{Highlighting}[]
\NormalTok{F }\OperatorTok{==}\StringTok{ }\DecValTok{0}
\end{Highlighting}
\end{Shaded}

\begin{verbatim}
## [1] TRUE
\end{verbatim}

\begin{Shaded}
\begin{Highlighting}[]
\NormalTok{(F }\OperatorTok{+}\StringTok{ }\DecValTok{1}\NormalTok{) }\OperatorTok{==}\StringTok{ }\OtherTok{TRUE}
\end{Highlighting}
\end{Shaded}

\begin{verbatim}
## [1] TRUE
\end{verbatim}

Aunque esta forma de referirse a \texttt{TRUE} y \texttt{FALSE} por
\texttt{T} y \texttt{F} está bastante extendida, en este libro siempre
usaremos \texttt{TRUE} y \texttt{FALSE} para que el código sea más fácil
de leer. Como mencionado anterioramente, el objetivo de un código no
solo es ser funcional sino también fácil de leer y volver a leer.

\subsection{Los operadores lógicos}\label{l011oplog}

Hay un último tipo de operador, los operadores lógicos. Estos son útiles
para combinar operadores de comparación (Table \ref{tab:tabOpLog}).

\begin{table}

\caption{\label{tab:tabOpLog}Operadores lógicos.\label{tab:tabOpLog}}
\centering
\begin{tabular}[t]{l|l}
\hline
Label & Operador\\
\hline
no es & !\\
\hline
y & \&\\
\hline
o & |\\
\hline
o exclusivo & xor()\\
\hline
\end{tabular}
\end{table}

\begin{Shaded}
\begin{Highlighting}[]
\OperatorTok{!}\OtherTok{TRUE}
\end{Highlighting}
\end{Shaded}

\begin{verbatim}
## [1] FALSE
\end{verbatim}

\begin{Shaded}
\begin{Highlighting}[]
\OperatorTok{!}\OtherTok{FALSE}
\end{Highlighting}
\end{Shaded}

\begin{verbatim}
## [1] TRUE
\end{verbatim}

\begin{Shaded}
\begin{Highlighting}[]
\NormalTok{((}\DecValTok{3} \OperatorTok{+}\StringTok{ }\DecValTok{2}\NormalTok{) }\OperatorTok{==}\StringTok{ }\DecValTok{5}\NormalTok{) }\OperatorTok{&}\StringTok{ }\NormalTok{((}\DecValTok{3} \OperatorTok{+}\StringTok{ }\DecValTok{3}\NormalTok{) }\OperatorTok{==}\StringTok{ }\DecValTok{5}\NormalTok{)}
\end{Highlighting}
\end{Shaded}

\begin{verbatim}
## [1] FALSE
\end{verbatim}

\begin{Shaded}
\begin{Highlighting}[]
\NormalTok{((}\DecValTok{3} \OperatorTok{+}\StringTok{ }\DecValTok{2}\NormalTok{) }\OperatorTok{==}\StringTok{ }\DecValTok{5}\NormalTok{) }\OperatorTok{&}\StringTok{ }\NormalTok{((}\DecValTok{3} \OperatorTok{+}\StringTok{ }\DecValTok{3}\NormalTok{) }\OperatorTok{==}\StringTok{ }\DecValTok{6}\NormalTok{)}
\end{Highlighting}
\end{Shaded}

\begin{verbatim}
## [1] TRUE
\end{verbatim}

\begin{Shaded}
\begin{Highlighting}[]
\NormalTok{(}\DecValTok{3} \OperatorTok{<}\StringTok{ }\DecValTok{5}\NormalTok{) }\OperatorTok{&}\StringTok{ }\NormalTok{(}\DecValTok{5} \OperatorTok{<}\StringTok{ }\DecValTok{5}\NormalTok{)}
\end{Highlighting}
\end{Shaded}

\begin{verbatim}
## [1] FALSE
\end{verbatim}

\begin{Shaded}
\begin{Highlighting}[]
\NormalTok{(}\DecValTok{3} \OperatorTok{<}\StringTok{ }\DecValTok{5}\NormalTok{) }\OperatorTok{&}\StringTok{ }\NormalTok{(}\DecValTok{5} \OperatorTok{<=}\StringTok{ }\DecValTok{5}\NormalTok{)}
\end{Highlighting}
\end{Shaded}

\begin{verbatim}
## [1] TRUE
\end{verbatim}

El operador lógico \texttt{xor()} es \emph{o exclusivo}. Es decir, uno
de los dos \textbf{argumentos} de la \textbf{función} \texttt{xor()}
debe ser verdadero, pero no ambos. Más adelante volveremos a las
\textbf{funciones} y sus \textbf{argumentos}, pero recuerde que
identificamos una función por sus paréntesis que contienen argumentos
separados por comas.

\begin{Shaded}
\begin{Highlighting}[]
\KeywordTok{xor}\NormalTok{((}\DecValTok{3} \OperatorTok{+}\StringTok{ }\DecValTok{2}\NormalTok{) }\OperatorTok{==}\StringTok{ }\DecValTok{5}\NormalTok{, (}\DecValTok{3} \OperatorTok{+}\StringTok{ }\DecValTok{3}\NormalTok{) }\OperatorTok{==}\StringTok{ }\DecValTok{6}\NormalTok{)}
\end{Highlighting}
\end{Shaded}

\begin{verbatim}
## [1] FALSE
\end{verbatim}

\begin{Shaded}
\begin{Highlighting}[]
\KeywordTok{xor}\NormalTok{((}\DecValTok{3} \OperatorTok{+}\StringTok{ }\DecValTok{2}\NormalTok{) }\OperatorTok{==}\StringTok{ }\DecValTok{5}\NormalTok{, (}\DecValTok{3} \OperatorTok{+}\StringTok{ }\DecValTok{2}\NormalTok{) }\OperatorTok{==}\StringTok{ }\DecValTok{6}\NormalTok{)}
\end{Highlighting}
\end{Shaded}

\begin{verbatim}
## [1] TRUE
\end{verbatim}

\begin{Shaded}
\begin{Highlighting}[]
\KeywordTok{xor}\NormalTok{((}\DecValTok{3} \OperatorTok{+}\StringTok{ }\DecValTok{3}\NormalTok{) }\OperatorTok{==}\StringTok{ }\DecValTok{5}\NormalTok{, (}\DecValTok{3} \OperatorTok{+}\StringTok{ }\DecValTok{2}\NormalTok{) }\OperatorTok{==}\StringTok{ }\DecValTok{6}\NormalTok{)}
\end{Highlighting}
\end{Shaded}

\begin{verbatim}
## [1] FALSE
\end{verbatim}

\begin{Shaded}
\begin{Highlighting}[]
\KeywordTok{xor}\NormalTok{((}\DecValTok{3} \OperatorTok{+}\StringTok{ }\DecValTok{3}\NormalTok{) }\OperatorTok{==}\StringTok{ }\DecValTok{5}\NormalTok{, (}\DecValTok{3} \OperatorTok{+}\StringTok{ }\DecValTok{3}\NormalTok{) }\OperatorTok{==}\StringTok{ }\DecValTok{6}\NormalTok{)}
\end{Highlighting}
\end{Shaded}

\begin{verbatim}
## [1] TRUE
\end{verbatim}

\begin{rmdstyle}
Se recomienda que las comas \texttt{,} sean seguidas de un espacio para
que el código sea más agradable de leer.
\end{rmdstyle}

\subsection{Ayuda a los operadores}\label{ayuda-a-los-operadores}

El archivo de ayuda en inglés sobre operadores aritméticos se puede
obtener con el comando \texttt{?\textquotesingle{}+\textquotesingle{}}.
El de los operadores de comparación con el comando
\texttt{?\textquotesingle{}==\textquotesingle{}} y el de los operadores
lógicos con el comando \texttt{?\textquotesingle{}\&\textquotesingle{}}.

\section{El concepto de objeto}\label{el-concepto-de-objeto}

Un aspecto importante de la programación con R, pero también la
programación en general es la noción de objeto. Como se indica en la
página web de wikipedia
(\url{https://ia.wikipedia.org/wiki/Objecto_(informatica)}), en ciencias
de la computación, un objeto es un \emph{contenedor}, es decir, algo que
contendrá información. La información contenida en un objeto puede ser
muy diversa, pero por el momento contendremos en un objeto el número 5.
Para hacer esto (y para reutilizarlo más adelante), debemos darle un
nombre a nuestro objeto. En R, los nombres de los objetos no deben
contener caracteres especiales como \emph{\^{} \$ ? \textbar{} + ()
{[}{]} \{\}}, entre otros. No deben comenzar con un número ni contener
espacios. El nombre del objeto debe ser representativo de lo que
contiene, sin ser demasiado corto ni demasiado largo. Imagine que
nuestro número 5 corresponde al número de repeticiones de un
experimento. Nos gustaría darle un nombre que se refiera a \emph{numero}
y \emph{repeticiones}, que podríamos reducir a \emph{nbr} y \emph{rep},
respectivamente (\emph{nbr} para number en ingles). Hay varias
posibilidades que son bastante comunes bajo R:

\begin{itemize}
\tightlist
\item
  la separación mediante \emph{guión bajo} (underscore):
  \texttt{nbr\_rep}
\item
  la separación mediante el carácter \emph{punto}: \texttt{nbr.rep}
\item
  el uso de letras minúsculas: \texttt{nbrrep}
\item
  el estilo \emph{lowerCamelCase} que consiste en una primera palabra en
  minúscula y la primera letra de las siguientes palabras con una letra
  mayúscula: \texttt{nbrRep}
\item
  el estilo \emph{UpperCamelCase} donde cada palabra comienza con una
  letra mayúscula: \texttt{NbrRep}
\end{itemize}

Todas estas formas de nombrar un objeto son equivalentes. En este libro
usaremos el estilo \emph{lowerCamelCase}. En general, debemos evitar los
nombres que son demasiado largos, como
\texttt{miNumeroDeRepeticionesDeMiExperimento} o demasiado cortos como
\texttt{nR}, y los nombres que no permiten identificar los contenidos
como \texttt{miVariable} o \texttt{miNumero}, asi que nombres como
\texttt{a} o \texttt{b}. El objetivo es de tener una idea de lo que hay
en cada objeto en base a su nombre.

\begin{rmdstyle}
Hay diferentes maneras de definir un nombre para los objetos que
crearemos con R. En este libro, utilizamos el estilo
\emph{lowerCamelCase}. Lo importante no es la elección del estilo, sino
la consistencia en su elección. El objetivo es tener un código
funcional, pero también un código que sea fácil y agradable de leer para
nosotros y para los demás.
\end{rmdstyle}

Ahora que hemos elegido un nombre para nuestro objeto, debemos crearlo y
hacer que R entienda que nuestro objeto debe contener el número 5. Hay
tres maneras de crear un objeto bajo R:

\begin{itemize}
\tightlist
\item
  con \texttt{\textless{}-}
\item
  con \texttt{=}
\item
  o con \texttt{-\textgreater{}}
\end{itemize}

\begin{Shaded}
\begin{Highlighting}[]
\NormalTok{nbrRep <-}\StringTok{ }\DecValTok{5}
\NormalTok{nbrRep =}\StringTok{ }\DecValTok{5}
\DecValTok{5}\NormalTok{ ->}\StringTok{ }\NormalTok{nbrRep}
\end{Highlighting}
\end{Shaded}

En este libro siempre usaremos la forma \texttt{\textless{}-} para
coherencia y también porque es la forma más común.

\begin{Shaded}
\begin{Highlighting}[]
\NormalTok{nbrRep <-}\StringTok{ }\DecValTok{5}
\end{Highlighting}
\end{Shaded}

Acabamos de crear un objeto \texttt{nbrRep} y establecerlo con el valor
\emph{5}. Este objeto ahora está disponible en nuestro entorno de
computación y puede ser utilizado. Algunos ejemplos :

\begin{Shaded}
\begin{Highlighting}[]
\NormalTok{nbrRep }\OperatorTok{+}\StringTok{ }\DecValTok{2}
\end{Highlighting}
\end{Shaded}

\begin{verbatim}
## [1] 7
\end{verbatim}

\begin{Shaded}
\begin{Highlighting}[]
\NormalTok{nbrRep }\OperatorTok{*}\StringTok{ }\DecValTok{5} \OperatorTok{-}\StringTok{ }\DecValTok{45}\OperatorTok{/}\DecValTok{56}
\end{Highlighting}
\end{Shaded}

\begin{verbatim}
## [1] 24.19643
\end{verbatim}

\begin{Shaded}
\begin{Highlighting}[]
\NormalTok{pi }\OperatorTok{*}\StringTok{ }\NormalTok{nbrRep}\OperatorTok{^}\DecValTok{2}
\end{Highlighting}
\end{Shaded}

\begin{verbatim}
## [1] 78.53982
\end{verbatim}

El valor asociado con nuestro objeto \texttt{nbrRep} se puede modificar
de la misma manera que cuando se creó:

\begin{Shaded}
\begin{Highlighting}[]
\NormalTok{nbrRep <-}\StringTok{ }\DecValTok{5}
\NormalTok{nbrRep }\OperatorTok{+}\StringTok{ }\DecValTok{2}
\end{Highlighting}
\end{Shaded}

\begin{verbatim}
## [1] 7
\end{verbatim}

\begin{Shaded}
\begin{Highlighting}[]
\NormalTok{nbrRep <-}\StringTok{ }\DecValTok{10}
\NormalTok{nbrRep }\OperatorTok{+}\StringTok{ }\DecValTok{2}
\end{Highlighting}
\end{Shaded}

\begin{verbatim}
## [1] 12
\end{verbatim}

\begin{Shaded}
\begin{Highlighting}[]
\NormalTok{nbrRep <-}\StringTok{ }\DecValTok{5} \OperatorTok{*}\StringTok{ }\DecValTok{2} \OperatorTok{+}\StringTok{ }\DecValTok{7}\OperatorTok{/}\DecValTok{3}
\NormalTok{nbrRep }\OperatorTok{+}\StringTok{ }\DecValTok{2}
\end{Highlighting}
\end{Shaded}

\begin{verbatim}
## [1] 14.33333
\end{verbatim}

El uso de objetos tiene sentido cuando tenemos operaciones complejas
para realizar y hace que el código sea más agradable de leer y entender.

\begin{Shaded}
\begin{Highlighting}[]
\NormalTok{(}\DecValTok{5} \OperatorTok{+}\StringTok{ }\DecValTok{9}\OperatorTok{^}\DecValTok{2} \OperatorTok{-}\StringTok{ }\DecValTok{1}\OperatorTok{/}\DecValTok{18}\NormalTok{) }\OperatorTok{/}\StringTok{ }\NormalTok{(}\DecValTok{32} \OperatorTok{*}\StringTok{ }\DecValTok{45}\OperatorTok{/}\DecValTok{8} \OperatorTok{+}\StringTok{ }\DecValTok{3}\NormalTok{)}
\end{Highlighting}
\end{Shaded}

\begin{verbatim}
## [1] 0.4696418
\end{verbatim}

\begin{Shaded}
\begin{Highlighting}[]
\NormalTok{termino01 <-}\StringTok{ }\DecValTok{5} \OperatorTok{+}\StringTok{ }\DecValTok{9}\OperatorTok{^}\DecValTok{2} \OperatorTok{-}\StringTok{ }\DecValTok{1}\OperatorTok{/}\DecValTok{18}
\NormalTok{termino02 <-}\StringTok{ }\DecValTok{32} \OperatorTok{*}\StringTok{ }\DecValTok{45}\OperatorTok{/}\DecValTok{8} \OperatorTok{+}\StringTok{ }\DecValTok{3}
\NormalTok{termino01 }\OperatorTok{/}\StringTok{ }\NormalTok{termino02}
\end{Highlighting}
\end{Shaded}

\begin{verbatim}
## [1] 0.4696418
\end{verbatim}

\section{Los scripts}\label{los-scripts}

R es un lenguaje de programación denominado \emph{lenguaje de
scripting}. Esto se refiere al hecho de que la mayoría de los usuarios
escribirán pequeñas piezas de código en lugar de programas completos. R
se puede usar como una simple calculadora, y en este caso no será
necesario mantener un historial de las operaciones que se han realizado.
Pero si las operaciones a implementar son largas y complejas, puede ser
necesario e interesante guardar lo que se ha hecho para poder continuar
más adelante. El archivo en el que se almacenarán las operaciones es lo
que comúnmente se llama el \emph{script}. Un \emph{script}, por lo
tanto, es un archivo que contiene una sucesión de información
comprensible por R y que es posible ejecutar.

\subsection{Crear un script y
documentarlo}\label{crear-un-script-y-documentarlo}

Para crear un nuevo script basta con abrir un documento vacío de texto,
que será editado por un editor de texto como el bloc de notas en Windows
o Mac OSX, o Gedit o incluso nano en Linux. Por convención, este archivo
toma la extensión ``.r'' o ``.R'' (lo mas comun). Esta última convención
se usará en este libro (\emph{``miArchivo.R''}). Desde la interfaz
gráfica de R, es posible crear un nuevo script en Mac OS y Windows a
través de \emph{file}, luego \emph{new script} y \emph{save as}. Al
igual que el nombre de los objetos, el nombre del script es importante
para que podamos identificar fácilmente su contenido. Por ejemplo,
podríamos crear un archivo \texttt{formRConceptsBase.R} que contenga los
objetos que acabamos de crear y los cálculos que hicimos. Pero incluso
con nombres de objetos y archivos bien definidos, será difícil recordar
el significado de este archivo sin la documentación que acompaña a este
script. Para documentar un script utilizaremos \emph{comentarios}. Los
\emph{comentarios} son elementos que R identificará como tales y no se
ejecutarán. Para especificar a R que vamos a hacer un \emph{comentario},
debemos usar el carácter octothorpe (corsé o numeral) \texttt{\#}. Los
comentarios se pueden insertar en una nueva línea o al final de la
línea.

\begin{Shaded}
\begin{Highlighting}[]
\CommentTok{# creación objeto número de repeticiones}
\NormalTok{nbrRep <-}\StringTok{ }\DecValTok{5} \CommentTok{# Comentario de fin de línea}
\end{Highlighting}
\end{Shaded}

\begin{rmdstyle}
Todo lo que hay despues del simbolo numeral \texttt{\#} no sera
ejecutado. Significa que podriamos usar comentarios como \texttt{\#\#\#}
o \texttt{\#comentario}, aun que se recomienda hacer comentarios con un
solo simbolo numeral seguido por un espacio y despues su comentario:
\texttt{\#\ mi\ comentario}.
\end{rmdstyle}

Los comentarios también se pueden usar para hacer que una línea ya no se
ejecute. En este caso no queremos ejecutar la secunda linea:

\begin{Shaded}
\begin{Highlighting}[]
\NormalTok{nbrRep <-}\StringTok{ }\DecValTok{5}
\CommentTok{# nbrRep + 5}
\end{Highlighting}
\end{Shaded}

Para volver a la documentación del script, se recomienda comenzar cada
uno de nuestros scripts con una breve descripción de su contenido, luego
cuando el script sea extenso, estructurarlo en diferentes partes para
facilitar su lectura.

\begin{Shaded}
\begin{Highlighting}[]
\CommentTok{# ------------------------------------------------------------}
\CommentTok{# Aquí hay un script para adquirir los conceptos básicos}
\CommentTok{# con R}
\CommentTok{# fecha de creación : 27/06/2018}
\CommentTok{# autor : François Rebaudo}
\CommentTok{# ------------------------------------------------------------}

\CommentTok{# [1] creación del objeto número de repeticiones}
\CommentTok{# ------------------------------------------------------------}

\NormalTok{nbrRep <-}\StringTok{ }\DecValTok{5}

\CommentTok{# [2] cálculos simples}
\CommentTok{# ------------------------------------------------------------}

\NormalTok{pi }\OperatorTok{*}\StringTok{ }\NormalTok{nbrRep}\OperatorTok{^}\DecValTok{2}
\end{Highlighting}
\end{Shaded}

\begin{verbatim}
## [1] 78.53982
\end{verbatim}

\begin{rmdstyle}
Para ir más allá en el estilo del código, una guía completa de
recomendaciones está disponible en línea en el sitio web
\emph{tidyverse} (en ingles ; \url{http://style.tidyverse.org/}).
\end{rmdstyle}

\subsection{Ejecutar un script}\label{ejecutar-un-script}

Como tenemos un script, no trabajamos directamente en la consola. Pero
solo la consola puede \emph{entender} el código R y devolvernos los
resultados que queremos obtener. Por ahora, la técnica más simple es
copiar y pegar las líneas que queremos ejecutar desde nuestro script
hasta la consola. A partir de ahora, ya no utilizaremos editores de
texto como bloc de notas, sino editores especializados para la creación
de scripts R. Sera es el objetivo del siguiente capítulo.

\section{Conclusión}\label{conclusion}

Felicitaciones, hemos llegado al final de este primer capítulo sobre la
base de R. Sabemos:

\begin{itemize}
\tightlist
\item
  Instalar R
\item
  Usar R como una calculadora
\item
  Crear \textbf{objetos} y utilisarlos para los calculos aritméticos,
  comparativos y logicos
\item
  Elejir nombres pertinentes para los objetos
\item
  Crear un nuevo \textbf{script}
\item
  Elejir un nombre pertinente para el archivo del script
\item
  Ejecutar el codigo de un script
\item
  Documentar los scripts con \textbf{comentarios}
\item
  Usar un estilo de código para que sea agradable de leer y facil de
  entender
\end{itemize}

\chapter{Elegir un entorno de desarrollo}\label{IDE}

\section{Editores de texto y entorno de
desarrollo}\label{editores-de-texto-y-entorno-de-desarrollo}

Hay muchos editores de texto, el capítulo anterior permitió introducir
algunos de los más simples como el bloc de notas de Windows. Sin
embargo, los límites de estos editores han hecho tediosa la tarea de
escribir un script. De hecho, incluso estructurando su script con
comentarios, sigue siendo difícil entenderlo. Aquí es donde entran los
editores de texto especializados para facilitar la escritura y la
lectura de scripts. El editor de texto para R más común es Rstudio, pero
hay muchos más. Hacer una lista exhaustiva de todas las soluciones
disponibles está más allá del alcance de este libro, por lo que nos
centraremos en las tres soluciones que utilizo a diario:
\textbf{Notepad++}, \textbf{Rstudio} y \textbf{Geany}. No necesita
instalar más de un editor de texto. Aquí recomendamos RStudio para
principiantes a R.

\section{RStudio}\label{rstudio}

\begin{figure}
\includegraphics[width=7.78in]{myLogos/RStudio} \caption{Logo RStudio.\label{fig:logoRStudio}}\label{fig:logoRStudio}
\end{figure}

\subsection{Instalar RStudio}\label{instalar-rstudio}

El programa para instalar el software RStudio se encuentra en la parte
\emph{Products} del sitio web de RStudio
(\url{https://www.rstudio.com/}). Instalaremos RStudio para uso local
(en nuestra computadora), por lo que la versión que nos interesa es
\emph{Desktop}. Usaremos la versión \emph{Open Source} que es gratuita.
Luego, seleccionamos la versión que corresponde a nuestro sistema
operativo (Windows, Mac OS, Linux), descargamos el archivo
correspondiente y lo ejecutamos para comenzar la instalación. Podemos
mantener las opciones predeterminadas durante la instalación.

\subsection{Un script con RStudio}\label{un-script-con-rstudio}

Podemos abrir RStudio. En la primera apertura, la interfaz se divide en
dos con la consola R a la izquierda que vimos en el capítulo anterior
(Figura \ref{fig:screenCapRStudio01}). Para abrir un nuevo script, vamos
al menú \emph{Archivo} (o \emph{File}), \emph{Nuevo archivo} (o
\emph{New File}), \emph{R script}. Por defecto, este archivo tiene el
nombre \emph{Untitled1}. Hemos visto en el capítulo anterior la
importancia de dar un nombre pertinente a nuestros scripts, por lo que
lo cambiaremos de nombre a \emph{selecEnvDev.R}, en el menú
\emph{Archivo} (o \emph{File}), con la opción \emph{Guardar como
\ldots{}} (o \emph{Save As\ldots{}}). Podemos notar que el lado
izquierdo de RStudio ahora está dividido en dos, con la consola en la
parte inferior de la pantalla y el script en la parte superior.

\begin{figure}
\includegraphics[width=10in]{myFigures/screencap_RStudio_01} \caption{Captura de pantalla de RStudio en Windows: pantalla por defecto.\label{fig:screenCapRStudio01}}\label{fig:screenCapRStudio01}
\end{figure}

Luego podemos comenzar a escribir nuestro script con los comentarios que
describan lo que vamos a encontrar allí, y agregar un cálculo simple.
Una vez que hayamos copiado o escrito un código, podemos guardar nuestro
script con el comando \texttt{CTRL\ +\ S} o yendo a \emph{Archivo} (o
\emph{File}, luego \emph{Guardar} (o \emph{Save}).

\begin{Shaded}
\begin{Highlighting}[]
\CommentTok{# ------------------------------------------------------------}
\CommentTok{# Un script para seleccionar su entorno de desarrollo}
\CommentTok{# fecha de creación : 27/06/2018}
\CommentTok{# autor : François Rebaudo}
\CommentTok{# ------------------------------------------------------------}

\CommentTok{# [1] cálculos simples}
\CommentTok{# ------------------------------------------------------------}
\NormalTok{nbrRep <-}\StringTok{ }\DecValTok{5}
\NormalTok{pi }\OperatorTok{*}\StringTok{ }\NormalTok{nbrRep}\OperatorTok{^}\DecValTok{2}
\end{Highlighting}
\end{Shaded}

\begin{verbatim}
## [1] 78.53982
\end{verbatim}

Para ejecutar nuestro script, simplemente seleccionamos las líneas que
deseamos ejecutar y usamos la combinación de teclas
\texttt{CTRL\ +\ ENTER}. El resultado aparece en la consola (Figura
\ref{fig:screenCapRStudio02}).

\begin{figure}
\includegraphics[width=10in]{myFigures/screencap_RStudio_02} \caption{Captura de pantalla de RStudio en Windows: ejecutar nuestro script con CTRL + ENTER.\label{fig:screenCapRStudio02}}\label{fig:screenCapRStudio02}
\end{figure}

Podemos ver que, de forma predeterminada, en la parte del script, los
comentarios aparecen en verde, los números en azul y el resto del código
en negro. En la parte de la consola, lo que se ejecutó aparece en azul y
los resultados de la ejecución en negro. También podemos observar que en
la parte del código cada línea tiene un número correspondiente al número
de línea a la izquierda sobre un fondo gris. Este es el resaltado de
preferencias de sintaxis predeterminado con RStudio. Estas preferencias
de sintaxis pueden modificarse yendo al menú \emph{Herramientas} (o
\emph{Tools}), \emph{Opciones globales \ldots{}} (o \emph{Global
Options\ldots{}}), \emph{Aspecto} (o \emph{Appearance}), y luego
seleccionando otro tema del \emph{Editor de tema:} (o \emph{Editor
theme:}). Elegiremos el tema \emph{Cobalt}, luego \emph{OK} (Figura
\ref{fig:screenCapRStudio03}).

\begin{figure}
\includegraphics[width=10in]{myFigures/screencap_RStudio_03} \caption{Captura de pantalla de RStudio en Windows: cambiar preferencias de sintaxis.\label{fig:screenCapRStudio03}}\label{fig:screenCapRStudio03}
\end{figure}

Sabemos cómo crear un nuevo script, guardarlo, ejecutar su contenido y
cambiar la apariencia de RStudio. Veremos los muchos otros beneficios de
RStudio a lo largo de este libro, ya que es el entorno de desarrollo que
se utilizará. Sin embargo, seremos especialmente cuidadosos de que todos
los scripts desarrollados a lo largo de este libro se ejecuten de la
misma manera, independientemente del entorno de desarrollo utilizado.

\section{Notepad++ avec Npp2R}\label{notepad-avec-npp2r}

\begin{figure}
\includegraphics[width=7.78in]{myLogos/Notepadpp} \caption{Logo Notepad++\label{fig:logoNotepad}}\label{fig:logoNotepad}
\end{figure}

\subsection{Instalar Notepad++ (solamente para
Windows)}\label{instalar-notepad-solamente-para-windows}

El programa para instalar Notepad ++ se puede encontrar en la pestaña
\emph{Downloads} (\url{https://notepad-plus-plus.org/download/}).
Podemos elegir entre la versión de 32-bit y la de 64-bit (64-bit si no
sabe qué versión elegir). Notepad++ es suficiente para escribir un
script, pero es aún más poderoso con \emph{Notepad++ to R}
(\emph{Npp2R}) que permite ejecutar automáticamente nuestros scripts en
una consola localmente en nuestra computadora o remotamente en un
servidor.

\subsection{Instalar Npp2R}\label{instalar-npp2r}

El programa para instalar Npp2R está alojado en el sitio de Sourceforge
(\url{https://sourceforge.net/projects/npptor/}). Npp2R debe instalarse
después de Notepad++.

\subsection{Un script con Notepad++}\label{un-script-con-notepad}

Al abrir por primera vez, Notepad++ muestra un archivo vacío \emph{new
1} (Figura \ref{fig:screenCapNpp01}).

\begin{figure}
\includegraphics[width=10in]{myFigures/screencap_Npp_01} \caption{Captura de pantalla de Notepad++ en Windows: pantalla por defecto.\label{fig:screenCapNpp01}}\label{fig:screenCapNpp01}
\end{figure}

Como ya hemos creado un script para probarlo con RStudio, lo abriremos
de nuevo con Notepad++. En \emph{Archivo}, seleccionamos \emph{Abrir
\ldots{}} luego elijemos el script \emph{selecEnvDev.R} creado
previamente. Una vez que el script está abierto, vamos a \emph{Idioma},
luego \emph{R}, y de nuevo \emph{R}. Aparece el resaltado de sintaxis
(Figura \ref{fig:screenCapNpp02}).

\begin{figure}
\includegraphics[width=10in]{myFigures/screencap_Npp_02} \caption{Captura de pantalla de Notepad++ en Windows: ejecutar nuestro script con F8.\label{fig:screenCapNpp02}}\label{fig:screenCapNpp02}
\end{figure}

La ejecución del script solo se puede realizar si se está ejecutando
Npp2R. Para hacerlo, es necesario ejecutar el programa Npp2R desde el
prompt de Windows. Un icono debe aparecer en la parte inferior de su
pantalla demostrando que Npp2R está prendido. La ejecución automática
del código de Notepad++ se realiza seleccionando el código para ejecutar
y luego usando el comando \texttt{F8}. Si el comando no funciona, puede
ser necesario reiniciar la computadora. Si el comando funciona, se
abrirá una nueva ventana con una consola que ejecuta las líneas deseadas
(Figura \ref{fig:screenCapNpp03}).

\begin{figure}
\includegraphics[width=14.83in]{myFigures/screencap_Npp_03} \caption{Captura de pantalla de Notepad++ en Windows: la consola con F8.\label{fig:screenCapNpp03}}\label{fig:screenCapNpp03}
\end{figure}

Al igual que con RStudio, el resaltado de sintaxis se puede cambiar
desde el menú \emph{Configuración}, y se puede seleccionar un nuevo tema
(por ejemplo, \emph{Solarized} en la Figura \ref{fig:screenCapNpp04})

\begin{figure}
\includegraphics[width=10in]{myFigures/screencap_Npp_04} \caption{Captura de pantalla de Notepad++ en Windows con el tema Solarized.\label{fig:screenCapNpp04}}\label{fig:screenCapNpp04}
\end{figure}

Comparado con otros editores de texto, Notepad++ tiene la ventaja de ser
muy liviano, rapido y ofrece una amplia gama de opciones para
personalizar la escritura de códigos.

\section{Geany}\label{geany}

\begin{figure}
\includegraphics[width=7.78in]{myLogos/Geany} \caption{Logo Geany\label{fig:logoGeany}}\label{fig:logoGeany}
\end{figure}

\subsection{Instalar Geany (para Linux, Mac OSX y
Windows)}\label{instalar-geany-para-linux-mac-osx-y-windows}

El programa para instalar Geany se puede encontrar en la pestaña
\emph{Downloads} en el menú de la izquierda \emph{Releases} de la página
web (\url{https://www.geany.org/}). Luego solo descargamos el ejecutable
para Windows o el dmg para Mac OSX. Los usuarios de Linux preferirán un
\texttt{sudo\ apt-get\ install\ geany}.

\subsection{Un script con Geany}\label{un-script-con-geany}

Al abrir por primera vez, como para RStudio y Notepad++, se crea un
archivo vacío (Figura \ref{fig:screenCapGeany01}).

\begin{figure}
\includegraphics[width=10in]{myFigures/screencap_Geany_01} \caption{Captura de pantalla de Geany en Windows: pantalla por defecto.\label{fig:screenCapGeany01}}\label{fig:screenCapGeany01}
\end{figure}

Podemos abrir nuestro script con \emph{Archivo}, \emph{Abrir} (Figura
\ref{fig:screenCapGeany02}).

\begin{figure}
\includegraphics[width=10in]{myFigures/screencap_Geany_02} \caption{Captura de pantalla de Geany en Windows: abrir un script.\label{fig:screenCapGeany02}}\label{fig:screenCapGeany02}
\end{figure}

Para ejecutar nuestro script, la versión de Geany para Windows no tiene
un terminal incorporado, lo que hace que su uso sea limitado bajo este
sistema operativo. La ejecución de un script se puede hacer abriendo R
en una ventana separada y copiando y pegando las líneas que se
ejecutarán. En Linux y Mac OSX, podemos abrir R en el terminal en la
parte inferior de la ventana de Geany con el comando \texttt{R}. Podemos
configurar Geany para una combinación de teclas para ejecutar el código
seleccionado (por ejemplo \texttt{CTRL\ +\ R}). Para esto hay que
permitir el envío de selección al terminal
(\texttt{send\_selection\_unsafe\ =\ true}) in
\texttt{archivo\ geany.conf} y elegir el comando para enviar al terminal
(en \emph{Editar}, \emph{Preferencias}, \emph{Combinaciones}). Para
cambiar el tema de Geany, hay una colección de temas disponibles en
GitHub (\url{https://github.com/geany/geany-themes/}). El tema se puede
cambiar a través del menú \emph{Ver}, \emph{cambiar Esquema del color
\ldots{}} (un ejemplo con el tema \emph{Solarized} Figura @ref(Fig:
screenCapGeany03)).

\begin{figure}
\includegraphics[width=10in]{myFigures/screencap_Geany_03} \caption{Captura de pantalla de Geany en Windows: cambiar esquema de color.\label{fig:screenCapGeany03}}\label{fig:screenCapGeany03}
\end{figure}

\section{Otras soluciones}\label{otras-soluciones}

Hay muchas otras soluciones, algunas especializadas para R como
\textbf{Tinn-R} (\url{https://sourceforge.net/projects/tinn-r/}), y
otras más generales para programación como \textbf{Atom}
(\url{https://atom.io/}), \textbf{Sublime Text}
(\url{https://www.sublimetext.com/}), \textbf{Vim}
(\url{https://www.vim.org/}), \textbf{Gedit}
(\url{https://wiki.gnome.org/Apps/Gedit}), \textbf{GNU Emacs}
(\url{https://www.gnu.org/software/emacs/}), o \textbf{Brackets}
(\url{http://brackets.io/}) y \textbf{Eclipse}
(\url{http://www.eclipse.org/}).

\section{Conclusión}\label{conclusion-1}

Felicitaciones, llegamos al final de este capítulo sobre el entorno de
desarrollo para el uso de R. Hasta ahora sabemos:

\begin{itemize}
\tightlist
\item
  Instalar RStudio, Geany o Notepad++
\item
  Reconocer y elegir nuestro entorno de preferencia
\end{itemize}

A partir de acá podremos concentrarnos en el lenguaje de programación R
en un ambiente, facilitando el trabajo de lectura y de escritura del
código. Esto ya es un gran paso para manejar R.

\chapter{Tipos de datos}\label{dataType1}

Vimos anteriormente cómo crear un objeto. Un objeto es como una caja en
la que almacenaremos \emph{información}. Hasta ahora solo hemos
almacenado números, pero en este capítulo veremos que es posible
almacenar otra información y nos detendremos en los tipos más comunes.
En este capítulo utilizaremos \textbf{funciones} sobre las cuales
volveremos más adelante.

\section{\texorpdfstring{El tipo
\texttt{numeric}}{El tipo numeric}}\label{el-tipo-numeric}

El tipo \texttt{numeric} es lo que hemos hecho hasta ahora, almacenando
números. Hay dos tipos principales de números en R: enteros
(\emph{integer}) y números decimales (\emph{double}). Por defecto, R
considera todos los números como números decimales y asigna el tipo
\texttt{double}. Para verificar el tipo de datos utilizaremos la
\emph{función} \texttt{typeof()} que toma como \emph{argumento} un
objeto (o directamente la información que queremos probar). También
podemos usar la función \texttt{is.double()} que devolverá \texttt{TRUE}
si el número está en formato \texttt{double} y \texttt{FALSE} en caso
contrario. La función genérica \texttt{is.numeric()} devolverá
\texttt{TRUE} si el objeto está en formato \texttt{numeric} y
\texttt{FALSE} en caso contrario.

\begin{Shaded}
\begin{Highlighting}[]
\NormalTok{nbrRep <-}\StringTok{ }\DecValTok{5}
\KeywordTok{typeof}\NormalTok{(nbrRep)}
\end{Highlighting}
\end{Shaded}

\begin{verbatim}
## [1] "double"
\end{verbatim}

\begin{Shaded}
\begin{Highlighting}[]
\KeywordTok{typeof}\NormalTok{(}\FloatTok{5.32}\NormalTok{)}
\end{Highlighting}
\end{Shaded}

\begin{verbatim}
## [1] "double"
\end{verbatim}

\begin{Shaded}
\begin{Highlighting}[]
\KeywordTok{is.numeric}\NormalTok{(}\DecValTok{5}\NormalTok{)}
\end{Highlighting}
\end{Shaded}

\begin{verbatim}
## [1] TRUE
\end{verbatim}

\begin{Shaded}
\begin{Highlighting}[]
\KeywordTok{is.double}\NormalTok{(}\DecValTok{5}\NormalTok{)}
\end{Highlighting}
\end{Shaded}

\begin{verbatim}
## [1] TRUE
\end{verbatim}

Si queremos decirle a R que vamos a trabajar con un entero, entonces
necesitamos convertir nuestro número decimal en un entero con la función
\texttt{as.integer()}. También podemos usar la función
\texttt{is.integer()} que devolverá \texttt{TRUE} si el número está en
formato \texttt{integer} y \texttt{FALSE} en caso contrario.

\begin{Shaded}
\begin{Highlighting}[]
\NormalTok{nbrRep <-}\StringTok{ }\KeywordTok{as.integer}\NormalTok{(}\DecValTok{5}\NormalTok{)}
\KeywordTok{typeof}\NormalTok{(nbrRep)}
\end{Highlighting}
\end{Shaded}

\begin{verbatim}
## [1] "integer"
\end{verbatim}

\begin{Shaded}
\begin{Highlighting}[]
\KeywordTok{typeof}\NormalTok{(}\FloatTok{5.32}\NormalTok{)}
\end{Highlighting}
\end{Shaded}

\begin{verbatim}
## [1] "double"
\end{verbatim}

\begin{Shaded}
\begin{Highlighting}[]
\KeywordTok{typeof}\NormalTok{(}\KeywordTok{as.integer}\NormalTok{(}\FloatTok{5.32}\NormalTok{))}
\end{Highlighting}
\end{Shaded}

\begin{verbatim}
## [1] "integer"
\end{verbatim}

\begin{Shaded}
\begin{Highlighting}[]
\KeywordTok{as.integer}\NormalTok{(}\FloatTok{5.32}\NormalTok{)}
\end{Highlighting}
\end{Shaded}

\begin{verbatim}
## [1] 5
\end{verbatim}

\begin{Shaded}
\begin{Highlighting}[]
\KeywordTok{as.integer}\NormalTok{(}\FloatTok{5.99}\NormalTok{)}
\end{Highlighting}
\end{Shaded}

\begin{verbatim}
## [1] 5
\end{verbatim}

\begin{Shaded}
\begin{Highlighting}[]
\KeywordTok{is.numeric}\NormalTok{(nbrRep)}
\end{Highlighting}
\end{Shaded}

\begin{verbatim}
## [1] TRUE
\end{verbatim}

Vemos aquí que convertir un número como \texttt{5.99} a \texttt{integer}
solo devolverá la parte entera, \texttt{5}.

\begin{Shaded}
\begin{Highlighting}[]
\KeywordTok{is.integer}\NormalTok{(}\DecValTok{5}\NormalTok{)}
\end{Highlighting}
\end{Shaded}

\begin{verbatim}
## [1] FALSE
\end{verbatim}

\begin{Shaded}
\begin{Highlighting}[]
\KeywordTok{is.numeric}\NormalTok{(}\DecValTok{5}\NormalTok{)}
\end{Highlighting}
\end{Shaded}

\begin{verbatim}
## [1] TRUE
\end{verbatim}

\begin{Shaded}
\begin{Highlighting}[]
\KeywordTok{is.integer}\NormalTok{(}\KeywordTok{as.integer}\NormalTok{(}\DecValTok{5}\NormalTok{))}
\end{Highlighting}
\end{Shaded}

\begin{verbatim}
## [1] TRUE
\end{verbatim}

\begin{Shaded}
\begin{Highlighting}[]
\KeywordTok{is.numeric}\NormalTok{(}\KeywordTok{as.integer}\NormalTok{(}\DecValTok{5}\NormalTok{))}
\end{Highlighting}
\end{Shaded}

\begin{verbatim}
## [1] TRUE
\end{verbatim}

La suma de un número entero \texttt{integer} y un número decimal
\texttt{double} devuelve un número decimal.

\begin{Shaded}
\begin{Highlighting}[]
\NormalTok{sumIntDou <-}\StringTok{ }\KeywordTok{as.integer}\NormalTok{(}\DecValTok{5}\NormalTok{) }\OperatorTok{+}\StringTok{ }\FloatTok{5.2}
\KeywordTok{typeof}\NormalTok{(sumIntDou)}
\end{Highlighting}
\end{Shaded}

\begin{verbatim}
## [1] "double"
\end{verbatim}

\begin{Shaded}
\begin{Highlighting}[]
\NormalTok{sumIntInt <-}\StringTok{ }\KeywordTok{as.integer}\NormalTok{(}\DecValTok{5}\NormalTok{) }\OperatorTok{+}\StringTok{ }\KeywordTok{as.integer}\NormalTok{(}\DecValTok{5}\NormalTok{)}
\KeywordTok{typeof}\NormalTok{(sumIntInt)}
\end{Highlighting}
\end{Shaded}

\begin{verbatim}
## [1] "integer"
\end{verbatim}

Para resumir, el tipo \texttt{numeric} contiene dos subtipos, los tipos
\texttt{integer} para enteros y el tipo \texttt{double} para los números
decimales. Por defecto, R asigna el tipo \texttt{double} a los números.

\section{\texorpdfstring{El tipo
\texttt{character}}{El tipo character}}\label{el-tipo-character}

El tipo \texttt{character} es texto. De hecho, R permite trabajar con
texto. Para especificar a R que la información contenida en un objeto
está en formato de texto (o generalmente para todos los textos), usamos
las comillas dobles (\texttt{"}) o las comillas simples
(\texttt{\textquotesingle{}}).

\begin{Shaded}
\begin{Highlighting}[]
\NormalTok{myText <-}\StringTok{ "azerty"}
\NormalTok{myText2 <-}\StringTok{ 'azerty'}
\NormalTok{myText3 <-}\StringTok{ 'azerty uiop qsdfg hjklm'}
\KeywordTok{typeof}\NormalTok{(myText3)}
\end{Highlighting}
\end{Shaded}

\begin{verbatim}
## [1] "character"
\end{verbatim}

Tanto las comillas dobles y simples son útiles en nuestro texto. También
podemos \emph{escapar} un carácter especial como comillas gracias al
signo de barra invertida \texttt{\textbackslash{}}.

\begin{Shaded}
\begin{Highlighting}[]
\NormalTok{myText <-}\StringTok{ "a 'ze' 'rt' y"}
\KeywordTok{print}\NormalTok{(myText)}
\end{Highlighting}
\end{Shaded}

\begin{verbatim}
## [1] "a 'ze' 'rt' y"
\end{verbatim}

\begin{Shaded}
\begin{Highlighting}[]
\NormalTok{myText2 <-}\StringTok{ 'a "zert" y'}
\KeywordTok{print}\NormalTok{(myText2)}
\end{Highlighting}
\end{Shaded}

\begin{verbatim}
## [1] "a \"zert\" y"
\end{verbatim}

\begin{Shaded}
\begin{Highlighting}[]
\NormalTok{myText3 <-}\StringTok{ 'azerty uiop qsdfg hjklm'}
\KeywordTok{print}\NormalTok{(myText3)}
\end{Highlighting}
\end{Shaded}

\begin{verbatim}
## [1] "azerty uiop qsdfg hjklm"
\end{verbatim}

\begin{Shaded}
\begin{Highlighting}[]
\NormalTok{myText4 <-}\StringTok{ "qwerty }\CharTok{\textbackslash{}"}\StringTok{ azerty "}
\KeywordTok{print}\NormalTok{(myText4)}
\end{Highlighting}
\end{Shaded}

\begin{verbatim}
## [1] "qwerty \" azerty "
\end{verbatim}

\begin{Shaded}
\begin{Highlighting}[]
\NormalTok{myText5 <-}\StringTok{ "qwerty }\CharTok{\textbackslash{}\textbackslash{}}\StringTok{ azerty "}
\KeywordTok{print}\NormalTok{(myText5)}
\end{Highlighting}
\end{Shaded}

\begin{verbatim}
## [1] "qwerty \\ azerty "
\end{verbatim}

De forma predeterminada, cuando creamos un objeto, su contenido no es
devuelto por la consola. En Internet o en muchos libros podemos
encontrar el nombre del objeto en una línea para devolver sus
contenidos:

\begin{Shaded}
\begin{Highlighting}[]
\NormalTok{myText <-}\StringTok{ "a 'ze' 'rt' y"}
\NormalTok{myText}
\end{Highlighting}
\end{Shaded}

\begin{verbatim}
## [1] "a 'ze' 'rt' y"
\end{verbatim}

En este libro, no lo usaremos de esta manera y preferiremos el uso de la
función \texttt{print()}, que permite mostrar en la consola el contenido
de un objeto. El resultado es el mismo, pero el código es más fácil de
leer y más explícito sobre lo que hace.

\begin{Shaded}
\begin{Highlighting}[]
\NormalTok{myText <-}\StringTok{ "a 'ze' 'rt' y"}
\KeywordTok{print}\NormalTok{(myText)}
\end{Highlighting}
\end{Shaded}

\begin{verbatim}
## [1] "a 'ze' 'rt' y"
\end{verbatim}

\begin{Shaded}
\begin{Highlighting}[]
\NormalTok{nbrRep <-}\StringTok{ }\DecValTok{5}
\KeywordTok{print}\NormalTok{(nbrRep)}
\end{Highlighting}
\end{Shaded}

\begin{verbatim}
## [1] 5
\end{verbatim}

También podemos poner números en formato de texto, pero no debemos
olvidar poner comillas para especificar el tipo \texttt{character} o
usar la función\texttt{as.character()}. Una operación entre un texto y
un número devuelve un error. Por ejemplo, si agregamos \texttt{10} a
\texttt{5}, R nos dice que un \textbf{argumento} de la \textbf{función}
\texttt{+} no es un tipo \texttt{numeric} y que, por lo tanto, la
operación no es posible. Tampoco podemos agregar texto a texto, pero
veremos más adelante cómo \emph{concatenar} dos cadenas de texto.

\begin{Shaded}
\begin{Highlighting}[]
\NormalTok{myText <-}\StringTok{ "qwerty"}
\KeywordTok{typeof}\NormalTok{(myText)}
\end{Highlighting}
\end{Shaded}

\begin{verbatim}
## [1] "character"
\end{verbatim}

\begin{Shaded}
\begin{Highlighting}[]
\NormalTok{myText2 <-}\StringTok{ }\DecValTok{5}
\KeywordTok{typeof}\NormalTok{(myText2)}
\end{Highlighting}
\end{Shaded}

\begin{verbatim}
## [1] "double"
\end{verbatim}

\begin{Shaded}
\begin{Highlighting}[]
\NormalTok{myText3 <-}\StringTok{ "5"}
\KeywordTok{typeof}\NormalTok{(myText3)}
\end{Highlighting}
\end{Shaded}

\begin{verbatim}
## [1] "character"
\end{verbatim}

\begin{Shaded}
\begin{Highlighting}[]
\NormalTok{myText2 }\OperatorTok{+}\StringTok{ }\DecValTok{10}
\end{Highlighting}
\end{Shaded}

\begin{verbatim}
## [1] 15
\end{verbatim}

\begin{Shaded}
\begin{Highlighting}[]
\KeywordTok{as.character}\NormalTok{(}\DecValTok{5}\NormalTok{)}
\end{Highlighting}
\end{Shaded}

\begin{verbatim}
## [1] "5"
\end{verbatim}

\begin{Shaded}
\begin{Highlighting}[]
\CommentTok{# myText3 + 10 # Error in myText3 + 10 : non-numeric argument to binary operator}
\CommentTok{# "a" + "b" # Error in "a" + "b" : non-numeric argument to binary operator}
\end{Highlighting}
\end{Shaded}

Para resumir, el tipo \texttt{character} permite el ingreso de texto,
podemos reconocerlo con comillas simples o dobles.

\section{\texorpdfstring{El tipo
\texttt{factor}}{El tipo factor}}\label{el-tipo-factor}

El tipo \texttt{factor} corresponde a los factores. Los factores son una
elección dentro de una lista finita de posibilidades. Por ejemplo, los
países son factores porque existe una lista finita de países en el mundo
en un momento dado. Un factor puede definirse con la función
\texttt{factor()} o transformarse utilizando la función
\texttt{as.factor()}. Al igual que con otros tipos de datos, podemos
usar la función \texttt{is.factor()} para verificar el tipo de datos.
Para obtener una lista de todas las posibilidades, existe la función
\texttt{levels()} (esta función tendrá más sentido cuando nos acerquemos
a los tipos de contenedores de información).

\begin{Shaded}
\begin{Highlighting}[]
\NormalTok{factor01 <-}\StringTok{ }\KeywordTok{factor}\NormalTok{(}\StringTok{"aaa"}\NormalTok{)}
\KeywordTok{print}\NormalTok{(factor01)}
\end{Highlighting}
\end{Shaded}

\begin{verbatim}
## [1] aaa
## Levels: aaa
\end{verbatim}

\begin{Shaded}
\begin{Highlighting}[]
\KeywordTok{typeof}\NormalTok{(factor01)}
\end{Highlighting}
\end{Shaded}

\begin{verbatim}
## [1] "integer"
\end{verbatim}

\begin{Shaded}
\begin{Highlighting}[]
\KeywordTok{is.factor}\NormalTok{(factor01)}
\end{Highlighting}
\end{Shaded}

\begin{verbatim}
## [1] TRUE
\end{verbatim}

\begin{Shaded}
\begin{Highlighting}[]
\KeywordTok{levels}\NormalTok{(factor01)}
\end{Highlighting}
\end{Shaded}

\begin{verbatim}
## [1] "aaa"
\end{verbatim}

Un factor se puede transformar en texto con la función
\texttt{as.character()} pero también en número con
\texttt{as.numeric()}. Al cambiar al tipo \texttt{numeric}, cada factor
toma el valor de su posición en la lista de posibilidades. En nuestro
caso, solo hay una posibilidad, por lo que la función
\texttt{as.numeric()} devolverá \texttt{1}:

\begin{Shaded}
\begin{Highlighting}[]
\NormalTok{factor01 <-}\StringTok{ }\KeywordTok{factor}\NormalTok{(}\StringTok{"aaa"}\NormalTok{)}
\KeywordTok{as.character}\NormalTok{(factor01)}
\end{Highlighting}
\end{Shaded}

\begin{verbatim}
## [1] "aaa"
\end{verbatim}

\begin{Shaded}
\begin{Highlighting}[]
\KeywordTok{as.numeric}\NormalTok{(factor01)}
\end{Highlighting}
\end{Shaded}

\begin{verbatim}
## [1] 1
\end{verbatim}

\section{\texorpdfstring{El tipo
\texttt{logical}}{El tipo logical}}\label{el-tipo-logical}

El tipo \texttt{logical} corresponde a los valores \texttt{TRUE} y
\texttt{FALSE} (y \texttt{NA}) que ya hemos visto con los operadores de
comparación.

\begin{Shaded}
\begin{Highlighting}[]
\NormalTok{aLogic <-}\StringTok{ }\OtherTok{TRUE}
\KeywordTok{print}\NormalTok{(aLogic)}
\end{Highlighting}
\end{Shaded}

\begin{verbatim}
## [1] TRUE
\end{verbatim}

\begin{Shaded}
\begin{Highlighting}[]
\KeywordTok{typeof}\NormalTok{(aLogic)}
\end{Highlighting}
\end{Shaded}

\begin{verbatim}
## [1] "logical"
\end{verbatim}

\begin{Shaded}
\begin{Highlighting}[]
\KeywordTok{is.logical}\NormalTok{(aLogic)}
\end{Highlighting}
\end{Shaded}

\begin{verbatim}
## [1] TRUE
\end{verbatim}

\begin{Shaded}
\begin{Highlighting}[]
\NormalTok{aLogic }\OperatorTok{+}\StringTok{ }\DecValTok{1}
\end{Highlighting}
\end{Shaded}

\begin{verbatim}
## [1] 2
\end{verbatim}

\begin{Shaded}
\begin{Highlighting}[]
\KeywordTok{as.numeric}\NormalTok{(aLogic)}
\end{Highlighting}
\end{Shaded}

\begin{verbatim}
## [1] 1
\end{verbatim}

\begin{Shaded}
\begin{Highlighting}[]
\KeywordTok{as.character}\NormalTok{(aLogic)}
\end{Highlighting}
\end{Shaded}

\begin{verbatim}
## [1] "TRUE"
\end{verbatim}

\section{\texorpdfstring{Acerca de
\texttt{NA}}{Acerca de NA}}\label{acerca-de-na}

El valor \texttt{NA} se puede usar para especificar que no hay datos o
datos faltantes. Por defecto, \texttt{NA} es \texttt{logical}, pero se
puede usar para texto o números.

\begin{Shaded}
\begin{Highlighting}[]
\KeywordTok{print}\NormalTok{(}\OtherTok{NA}\NormalTok{)}
\end{Highlighting}
\end{Shaded}

\begin{verbatim}
## [1] NA
\end{verbatim}

\begin{Shaded}
\begin{Highlighting}[]
\KeywordTok{typeof}\NormalTok{(}\OtherTok{NA}\NormalTok{)}
\end{Highlighting}
\end{Shaded}

\begin{verbatim}
## [1] "logical"
\end{verbatim}

\begin{Shaded}
\begin{Highlighting}[]
\KeywordTok{typeof}\NormalTok{(}\KeywordTok{as.integer}\NormalTok{(}\OtherTok{NA}\NormalTok{))}
\end{Highlighting}
\end{Shaded}

\begin{verbatim}
## [1] "integer"
\end{verbatim}

\begin{Shaded}
\begin{Highlighting}[]
\KeywordTok{typeof}\NormalTok{(}\KeywordTok{as.character}\NormalTok{(}\OtherTok{NA}\NormalTok{))}
\end{Highlighting}
\end{Shaded}

\begin{verbatim}
## [1] "character"
\end{verbatim}

\begin{Shaded}
\begin{Highlighting}[]
\OtherTok{NA} \OperatorTok{==}\StringTok{ }\OtherTok{TRUE}
\end{Highlighting}
\end{Shaded}

\begin{verbatim}
## [1] NA
\end{verbatim}

\begin{Shaded}
\begin{Highlighting}[]
\OtherTok{NA} \OperatorTok{==}\StringTok{ }\OtherTok{FALSE}
\end{Highlighting}
\end{Shaded}

\begin{verbatim}
## [1] NA
\end{verbatim}

\begin{Shaded}
\begin{Highlighting}[]
\OtherTok{NA} \OperatorTok{>}\StringTok{ }\DecValTok{1}
\end{Highlighting}
\end{Shaded}

\begin{verbatim}
## [1] NA
\end{verbatim}

\begin{Shaded}
\begin{Highlighting}[]
\OtherTok{NA} \OperatorTok{+}\StringTok{ }\DecValTok{1}
\end{Highlighting}
\end{Shaded}

\begin{verbatim}
## [1] NA
\end{verbatim}

\section{Conclusión}\label{conclusion-2}

Felicitaciones, hemos llegado al final de este capítulo sobre los tipos
de datos. Ahora sabemos:

\begin{itemize}
\tightlist
\item
  Reconocer y hacer objetos en los principales tipos de datos
\item
  Transformar tipos de datos de un tipo a otro
\end{itemize}

Este capítulo es la base para el próximo capítulo sobre contenedores de
datos.

\chapter{Contenedores de datos}\label{dataType2}

Hasta ahora hemos creado objetos simples que contienen solo un valor.
Sin embargo, pudimos ver que un objeto tenía atributos diferentes, como
su valor, pero también el tipo de datos contenidos (e.g.,
\texttt{numeric}, \texttt{character}). Ahora vamos a ver que hay
diferentes tipos de contenedores para almacenar datos múltiples.

\hypertarget{l014vector}{\section{\texorpdfstring{El contenedor
\texttt{vector}}{El contenedor vector}}\label{l014vector}}

En R, un \texttt{vector} es una combinación de datos con la
particularidad de que todos los datos contenidos en un \texttt{vector}
son del mismo tipo. Podemos almacenar por ejemplo múltiples elementos
del tipo \texttt{character} o \texttt{numeric} en un \texttt{vector},
pero no ambos. El contenedor \texttt{vector} es importante porque es el
elemento básico de R.

\subsection{\texorpdfstring{Crear un
\texttt{vector}}{Crear un vector}}\label{crear-un-vector}

Para crear un \texttt{vector} utilizaremos la función \texttt{c()} que
permite combinar elementos en un \texttt{vector}. Los elementos para
combinar deben estar separados por comas.

\begin{Shaded}
\begin{Highlighting}[]
\NormalTok{miVec01 <-}\StringTok{ }\KeywordTok{c}\NormalTok{(}\DecValTok{1}\NormalTok{, }\DecValTok{2}\NormalTok{, }\DecValTok{3}\NormalTok{, }\DecValTok{4}\NormalTok{) }\CommentTok{# un vector de 4 elementos de tipo numeric ; double}
\KeywordTok{print}\NormalTok{(miVec01)}
\end{Highlighting}
\end{Shaded}

\begin{verbatim}
## [1] 1 2 3 4
\end{verbatim}

\begin{Shaded}
\begin{Highlighting}[]
\KeywordTok{typeof}\NormalTok{(miVec01)}
\end{Highlighting}
\end{Shaded}

\begin{verbatim}
## [1] "double"
\end{verbatim}

\begin{Shaded}
\begin{Highlighting}[]
\KeywordTok{is.vector}\NormalTok{(miVec01)}
\end{Highlighting}
\end{Shaded}

\begin{verbatim}
## [1] TRUE
\end{verbatim}

La funcion \texttt{is.vector()} permite verificar el tipo de contenedor.

\begin{Shaded}
\begin{Highlighting}[]
\NormalTok{miVec02 <-}\StringTok{ }\KeywordTok{c}\NormalTok{(}\StringTok{"a"}\NormalTok{, }\StringTok{"b"}\NormalTok{, }\StringTok{"c"}\NormalTok{) }
\KeywordTok{print}\NormalTok{(miVec02)}
\end{Highlighting}
\end{Shaded}

\begin{verbatim}
## [1] "a" "b" "c"
\end{verbatim}

\begin{Shaded}
\begin{Highlighting}[]
\KeywordTok{typeof}\NormalTok{(miVec02)}
\end{Highlighting}
\end{Shaded}

\begin{verbatim}
## [1] "character"
\end{verbatim}

\begin{Shaded}
\begin{Highlighting}[]
\KeywordTok{is.vector}\NormalTok{(miVec02)}
\end{Highlighting}
\end{Shaded}

\begin{verbatim}
## [1] TRUE
\end{verbatim}

\begin{Shaded}
\begin{Highlighting}[]
\NormalTok{miVec03 <-}\StringTok{ }\KeywordTok{c}\NormalTok{(}\OtherTok{TRUE}\NormalTok{, }\OtherTok{FALSE}\NormalTok{, }\OtherTok{FALSE}\NormalTok{, }\OtherTok{TRUE}\NormalTok{)}
\KeywordTok{print}\NormalTok{(miVec03)}
\end{Highlighting}
\end{Shaded}

\begin{verbatim}
## [1]  TRUE FALSE FALSE  TRUE
\end{verbatim}

\begin{Shaded}
\begin{Highlighting}[]
\KeywordTok{typeof}\NormalTok{(miVec03)}
\end{Highlighting}
\end{Shaded}

\begin{verbatim}
## [1] "logical"
\end{verbatim}

\begin{Shaded}
\begin{Highlighting}[]
\KeywordTok{is.vector}\NormalTok{(miVec03)}
\end{Highlighting}
\end{Shaded}

\begin{verbatim}
## [1] TRUE
\end{verbatim}

\begin{Shaded}
\begin{Highlighting}[]
\NormalTok{miVecNA <-}\StringTok{ }\KeywordTok{c}\NormalTok{(}\DecValTok{1}\NormalTok{, }\OtherTok{NA}\NormalTok{, }\DecValTok{3}\NormalTok{, }\OtherTok{NA}\NormalTok{, }\DecValTok{5}\NormalTok{)}
\KeywordTok{print}\NormalTok{(miVecNA)}
\end{Highlighting}
\end{Shaded}

\begin{verbatim}
## [1]  1 NA  3 NA  5
\end{verbatim}

\begin{Shaded}
\begin{Highlighting}[]
\KeywordTok{typeof}\NormalTok{(miVecNA)}
\end{Highlighting}
\end{Shaded}

\begin{verbatim}
## [1] "double"
\end{verbatim}

\begin{Shaded}
\begin{Highlighting}[]
\KeywordTok{is.vector}\NormalTok{(miVecNA)}
\end{Highlighting}
\end{Shaded}

\begin{verbatim}
## [1] TRUE
\end{verbatim}

\begin{Shaded}
\begin{Highlighting}[]
\NormalTok{miVec04 <-}\StringTok{ }\KeywordTok{c}\NormalTok{(}\DecValTok{1}\NormalTok{, }\StringTok{"a"}\NormalTok{)}
\KeywordTok{print}\NormalTok{(miVec04)}
\end{Highlighting}
\end{Shaded}

\begin{verbatim}
## [1] "1" "a"
\end{verbatim}

\begin{Shaded}
\begin{Highlighting}[]
\KeywordTok{typeof}\NormalTok{(miVec04)}
\end{Highlighting}
\end{Shaded}

\begin{verbatim}
## [1] "character"
\end{verbatim}

\begin{Shaded}
\begin{Highlighting}[]
\KeywordTok{is.vector}\NormalTok{(miVec04)}
\end{Highlighting}
\end{Shaded}

\begin{verbatim}
## [1] TRUE
\end{verbatim}

Si combinamos diferentes tipos de datos, R intentará transformar los
elementos en un tipo de forma predeterminada. Si como aquí en el objeto
\texttt{miVec03} tenemos los tipos \texttt{character} y
\texttt{numeric}, R convertirá todos los elementos en
\texttt{character}.

\begin{Shaded}
\begin{Highlighting}[]
\NormalTok{miVec05 <-}\StringTok{ }\KeywordTok{c}\NormalTok{(}\KeywordTok{factor}\NormalTok{(}\StringTok{"abc"}\NormalTok{), }\StringTok{"def"}\NormalTok{)}
\KeywordTok{print}\NormalTok{(miVec05)}
\end{Highlighting}
\end{Shaded}

\begin{verbatim}
## [1] "1"   "def"
\end{verbatim}

\begin{Shaded}
\begin{Highlighting}[]
\KeywordTok{typeof}\NormalTok{(miVec05)}
\end{Highlighting}
\end{Shaded}

\begin{verbatim}
## [1] "character"
\end{verbatim}

\begin{Shaded}
\begin{Highlighting}[]
\NormalTok{miVec06 <-}\StringTok{ }\KeywordTok{c}\NormalTok{(}\OtherTok{TRUE}\NormalTok{, }\StringTok{"def"}\NormalTok{)}
\KeywordTok{print}\NormalTok{(miVec06)}
\end{Highlighting}
\end{Shaded}

\begin{verbatim}
## [1] "TRUE" "def"
\end{verbatim}

\begin{Shaded}
\begin{Highlighting}[]
\KeywordTok{typeof}\NormalTok{(miVec06)}
\end{Highlighting}
\end{Shaded}

\begin{verbatim}
## [1] "character"
\end{verbatim}

\begin{Shaded}
\begin{Highlighting}[]
\NormalTok{miVec07 <-}\StringTok{ }\KeywordTok{c}\NormalTok{(}\KeywordTok{factor}\NormalTok{(}\StringTok{"abc"}\NormalTok{), }\DecValTok{55}\NormalTok{)}
\KeywordTok{print}\NormalTok{(miVec07)}
\end{Highlighting}
\end{Shaded}

\begin{verbatim}
## [1]  1 55
\end{verbatim}

\begin{Shaded}
\begin{Highlighting}[]
\KeywordTok{typeof}\NormalTok{(miVec07)}
\end{Highlighting}
\end{Shaded}

\begin{verbatim}
## [1] "double"
\end{verbatim}

\begin{Shaded}
\begin{Highlighting}[]
\NormalTok{miVec08 <-}\StringTok{ }\KeywordTok{c}\NormalTok{(}\OtherTok{TRUE}\NormalTok{, }\DecValTok{55}\NormalTok{)}
\KeywordTok{print}\NormalTok{(miVec08)}
\end{Highlighting}
\end{Shaded}

\begin{verbatim}
## [1]  1 55
\end{verbatim}

\begin{Shaded}
\begin{Highlighting}[]
\KeywordTok{typeof}\NormalTok{(miVec08)}
\end{Highlighting}
\end{Shaded}

\begin{verbatim}
## [1] "double"
\end{verbatim}

También podemos combinar objetos existentes dentro de un
\texttt{vector}.

\begin{Shaded}
\begin{Highlighting}[]
\NormalTok{miVec09 <-}\StringTok{ }\KeywordTok{c}\NormalTok{(miVec02, }\StringTok{"d"}\NormalTok{, }\StringTok{"e"}\NormalTok{, }\StringTok{"f"}\NormalTok{)}
\KeywordTok{print}\NormalTok{(miVec09)}
\end{Highlighting}
\end{Shaded}

\begin{verbatim}
## [1] "a" "b" "c" "d" "e" "f"
\end{verbatim}

\begin{Shaded}
\begin{Highlighting}[]
\NormalTok{miVec10 <-}\StringTok{ }\KeywordTok{c}\NormalTok{(}\StringTok{"aaa"}\NormalTok{, }\StringTok{"aa"}\NormalTok{, miVec09, }\StringTok{"d"}\NormalTok{, }\StringTok{"e"}\NormalTok{, }\StringTok{"f"}\NormalTok{)}
\KeywordTok{print}\NormalTok{(miVec10)}
\end{Highlighting}
\end{Shaded}

\begin{verbatim}
##  [1] "aaa" "aa"  "a"   "b"   "c"   "d"   "e"   "f"   "d"   "e"   "f"
\end{verbatim}

\begin{Shaded}
\begin{Highlighting}[]
\NormalTok{miVec11 <-}\StringTok{ }\KeywordTok{c}\NormalTok{(}\DecValTok{789}\NormalTok{, miVec01 , }\DecValTok{564}\NormalTok{)}
\KeywordTok{print}\NormalTok{(miVec11)}
\end{Highlighting}
\end{Shaded}

\begin{verbatim}
## [1] 789   1   2   3   4 564
\end{verbatim}

\subsection{\texorpdfstring{Hacer operaciones con un
\texttt{vector}}{Hacer operaciones con un vector}}\label{hacer-operaciones-con-un-vector}

También podemos realizar operaciones en un \texttt{vector}.

\begin{Shaded}
\begin{Highlighting}[]
\KeywordTok{print}\NormalTok{(miVec01)}
\end{Highlighting}
\end{Shaded}

\begin{verbatim}
## [1] 1 2 3 4
\end{verbatim}

\begin{Shaded}
\begin{Highlighting}[]
\NormalTok{miVec01 }\OperatorTok{+}\StringTok{ }\DecValTok{1}
\end{Highlighting}
\end{Shaded}

\begin{verbatim}
## [1] 2 3 4 5
\end{verbatim}

\begin{Shaded}
\begin{Highlighting}[]
\NormalTok{miVec01 }\OperatorTok{-}\StringTok{ }\DecValTok{1}
\end{Highlighting}
\end{Shaded}

\begin{verbatim}
## [1] 0 1 2 3
\end{verbatim}

\begin{Shaded}
\begin{Highlighting}[]
\NormalTok{miVec01 }\OperatorTok{*}\StringTok{ }\DecValTok{2}
\end{Highlighting}
\end{Shaded}

\begin{verbatim}
## [1] 2 4 6 8
\end{verbatim}

\begin{Shaded}
\begin{Highlighting}[]
\NormalTok{miVec01 }\OperatorTok{/}\DecValTok{10}
\end{Highlighting}
\end{Shaded}

\begin{verbatim}
## [1] 0.1 0.2 0.3 0.4
\end{verbatim}

Las operaciones de un \texttt{vector} a otro también son posibles, pero
se debe tener cuidado para asegurar que el número de elementos en un
\texttt{vector} sea el mismo que el otro, de lo contrario R realizará el
cálculo comenzando desde el inicio del \texttt{vector} mas pequeño. Aquí
hay un ejemplo para ilustrar lo que R hace:

\begin{Shaded}
\begin{Highlighting}[]
\NormalTok{miVec12 <-}\StringTok{ }\KeywordTok{c}\NormalTok{(}\DecValTok{1}\NormalTok{, }\DecValTok{1}\NormalTok{, }\DecValTok{1}\NormalTok{, }\DecValTok{1}\NormalTok{, }\DecValTok{1}\NormalTok{, }\DecValTok{1}\NormalTok{, }\DecValTok{1}\NormalTok{, }\DecValTok{1}\NormalTok{, }\DecValTok{1}\NormalTok{)}
\KeywordTok{print}\NormalTok{(miVec12)}
\end{Highlighting}
\end{Shaded}

\begin{verbatim}
## [1] 1 1 1 1 1 1 1 1 1
\end{verbatim}

\begin{Shaded}
\begin{Highlighting}[]
\NormalTok{miVec13 <-}\StringTok{ }\KeywordTok{c}\NormalTok{(}\DecValTok{10}\NormalTok{, }\DecValTok{20}\NormalTok{, }\DecValTok{30}\NormalTok{)}
\KeywordTok{print}\NormalTok{(miVec13)}
\end{Highlighting}
\end{Shaded}

\begin{verbatim}
## [1] 10 20 30
\end{verbatim}

\begin{Shaded}
\begin{Highlighting}[]
\NormalTok{miVec12 }\OperatorTok{+}\StringTok{ }\NormalTok{miVec13 }\CommentTok{# vectores de diferentes tamaños: atención al resultado}
\end{Highlighting}
\end{Shaded}

\begin{verbatim}
## [1] 11 21 31 11 21 31 11 21 31
\end{verbatim}

\begin{Shaded}
\begin{Highlighting}[]
\NormalTok{miVec14 <-}\StringTok{ }\KeywordTok{c}\NormalTok{(}\DecValTok{10}\NormalTok{, }\DecValTok{20}\NormalTok{, }\DecValTok{30}\NormalTok{, }\DecValTok{40}\NormalTok{, }\DecValTok{50}\NormalTok{, }\DecValTok{60}\NormalTok{, }\DecValTok{70}\NormalTok{, }\DecValTok{80}\NormalTok{, }\DecValTok{90}\NormalTok{)}
\KeywordTok{print}\NormalTok{(miVec14)}
\end{Highlighting}
\end{Shaded}

\begin{verbatim}
## [1] 10 20 30 40 50 60 70 80 90
\end{verbatim}

\begin{Shaded}
\begin{Highlighting}[]
\NormalTok{miVec12 }\OperatorTok{+}\StringTok{ }\NormalTok{miVec14 }\CommentTok{# los vectores tienen el mismo tamaño}
\end{Highlighting}
\end{Shaded}

\begin{verbatim}
## [1] 11 21 31 41 51 61 71 81 91
\end{verbatim}

\begin{Shaded}
\begin{Highlighting}[]
\NormalTok{miVec15 <-}\StringTok{ }\KeywordTok{c}\NormalTok{(}\DecValTok{1}\NormalTok{, }\DecValTok{1}\NormalTok{, }\DecValTok{1}\NormalTok{, }\DecValTok{1}\NormalTok{)}
\KeywordTok{print}\NormalTok{(miVec15)}
\end{Highlighting}
\end{Shaded}

\begin{verbatim}
## [1] 1 1 1 1
\end{verbatim}

\begin{Shaded}
\begin{Highlighting}[]
\NormalTok{miVec15 }\OperatorTok{+}\StringTok{ }\NormalTok{miVec13 }\CommentTok{# vectores de diferentes tamaños y no múltiples}
\end{Highlighting}
\end{Shaded}

\begin{verbatim}
## Warning in miVec15 + miVec13: la taille d'un objet plus long n'est pas
## multiple de la taille d'un objet plus court
\end{verbatim}

\begin{verbatim}
## [1] 11 21 31 11
\end{verbatim}

\subsection{\texorpdfstring{Acceder a los valores de un
\texttt{vector}}{Acceder a los valores de un vector}}\label{acceder-a-los-valores-de-un-vector}

Suele pasar que sea necesario poder acceder a los valores de un
\texttt{vector}, es decir, recuperar un valor o un grupo de valores
dentro de un \texttt{vector}. Para acceder a un elemento de un
\texttt{vector} usamos los corchetes \texttt{{[}{]}}. Entre los
corchetes, podemos usar un número correspondiente al número del elemento
en el \texttt{vector}.

\begin{Shaded}
\begin{Highlighting}[]
\NormalTok{miVec20 <-}\StringTok{ }\KeywordTok{c}\NormalTok{(}\DecValTok{10}\NormalTok{, }\DecValTok{20}\NormalTok{, }\DecValTok{30}\NormalTok{, }\DecValTok{40}\NormalTok{, }\DecValTok{50}\NormalTok{, }\DecValTok{60}\NormalTok{, }\DecValTok{70}\NormalTok{, }\DecValTok{80}\NormalTok{, }\DecValTok{90}\NormalTok{)}
\NormalTok{miVec21 <-}\StringTok{ }\KeywordTok{c}\NormalTok{(}\StringTok{"a"}\NormalTok{, }\StringTok{"b"}\NormalTok{, }\StringTok{"c"}\NormalTok{, }\StringTok{"d"}\NormalTok{, }\StringTok{"e"}\NormalTok{, }\StringTok{"f"}\NormalTok{, }\StringTok{"g"}\NormalTok{, }\StringTok{"h"}\NormalTok{, }\StringTok{"i"}\NormalTok{)}
\KeywordTok{print}\NormalTok{(miVec20)}
\end{Highlighting}
\end{Shaded}

\begin{verbatim}
## [1] 10 20 30 40 50 60 70 80 90
\end{verbatim}

\begin{Shaded}
\begin{Highlighting}[]
\KeywordTok{print}\NormalTok{(miVec21)}
\end{Highlighting}
\end{Shaded}

\begin{verbatim}
## [1] "a" "b" "c" "d" "e" "f" "g" "h" "i"
\end{verbatim}

\begin{Shaded}
\begin{Highlighting}[]
\KeywordTok{print}\NormalTok{(miVec20[}\DecValTok{1}\NormalTok{])}
\end{Highlighting}
\end{Shaded}

\begin{verbatim}
## [1] 10
\end{verbatim}

\begin{Shaded}
\begin{Highlighting}[]
\KeywordTok{print}\NormalTok{(miVec21[}\DecValTok{3}\NormalTok{])}
\end{Highlighting}
\end{Shaded}

\begin{verbatim}
## [1] "c"
\end{verbatim}

También podemos usar la combinación de diferentes elementos (otro
\texttt{vector}).

\begin{Shaded}
\begin{Highlighting}[]
\KeywordTok{print}\NormalTok{(miVec20[}\KeywordTok{c}\NormalTok{(}\DecValTok{1}\NormalTok{, }\DecValTok{5}\NormalTok{, }\DecValTok{9}\NormalTok{)])}
\end{Highlighting}
\end{Shaded}

\begin{verbatim}
## [1] 10 50 90
\end{verbatim}

\begin{Shaded}
\begin{Highlighting}[]
\KeywordTok{print}\NormalTok{(miVec21[}\KeywordTok{c}\NormalTok{(}\DecValTok{4}\NormalTok{, }\DecValTok{3}\NormalTok{, }\DecValTok{1}\NormalTok{)])}
\end{Highlighting}
\end{Shaded}

\begin{verbatim}
## [1] "d" "c" "a"
\end{verbatim}

\begin{Shaded}
\begin{Highlighting}[]
\KeywordTok{print}\NormalTok{(miVec21[}\KeywordTok{c}\NormalTok{(}\DecValTok{4}\NormalTok{, }\DecValTok{4}\NormalTok{, }\DecValTok{3}\NormalTok{, }\DecValTok{4}\NormalTok{, }\DecValTok{3}\NormalTok{, }\DecValTok{2}\NormalTok{, }\DecValTok{5}\NormalTok{)])}
\end{Highlighting}
\end{Shaded}

\begin{verbatim}
## [1] "d" "d" "c" "d" "c" "b" "e"
\end{verbatim}

También podemos seleccionar elementos usando un operador de comparación
o un operador lógico.

\begin{Shaded}
\begin{Highlighting}[]
\KeywordTok{print}\NormalTok{(miVec20[miVec20 }\OperatorTok{>=}\StringTok{ }\DecValTok{50}\NormalTok{])}
\end{Highlighting}
\end{Shaded}

\begin{verbatim}
## [1] 50 60 70 80 90
\end{verbatim}

\begin{Shaded}
\begin{Highlighting}[]
\KeywordTok{print}\NormalTok{(miVec20[(miVec20 }\OperatorTok{>=}\StringTok{ }\DecValTok{50}\NormalTok{) }\OperatorTok{&}\StringTok{ }\NormalTok{((miVec20 }\OperatorTok{<}\StringTok{ }\DecValTok{80}\NormalTok{))])}
\end{Highlighting}
\end{Shaded}

\begin{verbatim}
## [1] 50 60 70
\end{verbatim}

\begin{Shaded}
\begin{Highlighting}[]
\KeywordTok{print}\NormalTok{(miVec20[miVec20 }\OperatorTok{!=}\StringTok{ }\DecValTok{50}\NormalTok{])}
\end{Highlighting}
\end{Shaded}

\begin{verbatim}
## [1] 10 20 30 40 60 70 80 90
\end{verbatim}

\begin{Shaded}
\begin{Highlighting}[]
\KeywordTok{print}\NormalTok{(miVec20[miVec20 }\OperatorTok{==}\StringTok{ }\DecValTok{30}\NormalTok{])}
\end{Highlighting}
\end{Shaded}

\begin{verbatim}
## [1] 30
\end{verbatim}

\begin{Shaded}
\begin{Highlighting}[]
\KeywordTok{print}\NormalTok{(miVec20[(miVec20 }\OperatorTok{==}\StringTok{ }\DecValTok{30}\NormalTok{) }\OperatorTok{|}\StringTok{ }\NormalTok{(miVec20 }\OperatorTok{==}\StringTok{ }\DecValTok{50}\NormalTok{)])}
\end{Highlighting}
\end{Shaded}

\begin{verbatim}
## [1] 30 50
\end{verbatim}

\begin{Shaded}
\begin{Highlighting}[]
\KeywordTok{print}\NormalTok{(miVec21[miVec21 }\OperatorTok{==}\StringTok{ "a"}\NormalTok{])}
\end{Highlighting}
\end{Shaded}

\begin{verbatim}
## [1] "a"
\end{verbatim}

Otra característica interesante es la posibilidad de condicionar los
elementos a seleccionar en base a otro \texttt{vector}.

\begin{Shaded}
\begin{Highlighting}[]
\KeywordTok{print}\NormalTok{(miVec21[miVec20 }\OperatorTok{>=}\StringTok{ }\DecValTok{50}\NormalTok{])}
\end{Highlighting}
\end{Shaded}

\begin{verbatim}
## [1] "e" "f" "g" "h" "i"
\end{verbatim}

\begin{Shaded}
\begin{Highlighting}[]
\KeywordTok{print}\NormalTok{(miVec21[(miVec20 }\OperatorTok{>=}\StringTok{ }\DecValTok{50}\NormalTok{) }\OperatorTok{&}\StringTok{ }\NormalTok{((miVec20 }\OperatorTok{<}\StringTok{ }\DecValTok{80}\NormalTok{))])}
\end{Highlighting}
\end{Shaded}

\begin{verbatim}
## [1] "e" "f" "g"
\end{verbatim}

\begin{Shaded}
\begin{Highlighting}[]
\KeywordTok{print}\NormalTok{(miVec21[miVec20 }\OperatorTok{!=}\StringTok{ }\DecValTok{50}\NormalTok{])}
\end{Highlighting}
\end{Shaded}

\begin{verbatim}
## [1] "a" "b" "c" "d" "f" "g" "h" "i"
\end{verbatim}

\begin{Shaded}
\begin{Highlighting}[]
\KeywordTok{print}\NormalTok{(miVec21[miVec20 }\OperatorTok{==}\StringTok{ }\DecValTok{30}\NormalTok{])}
\end{Highlighting}
\end{Shaded}

\begin{verbatim}
## [1] "c"
\end{verbatim}

\begin{Shaded}
\begin{Highlighting}[]
\KeywordTok{print}\NormalTok{(miVec21[(miVec20 }\OperatorTok{==}\StringTok{ }\DecValTok{30}\NormalTok{) }\OperatorTok{|}\StringTok{ }\NormalTok{(miVec20 }\OperatorTok{==}\StringTok{ }\DecValTok{50}\NormalTok{)])}
\end{Highlighting}
\end{Shaded}

\begin{verbatim}
## [1] "c" "e"
\end{verbatim}

\begin{Shaded}
\begin{Highlighting}[]
\KeywordTok{print}\NormalTok{(miVec21[(miVec20 }\OperatorTok{==}\StringTok{ }\DecValTok{30}\NormalTok{) }\OperatorTok{|}\StringTok{ }\NormalTok{(miVec21 }\OperatorTok{==}\StringTok{ "h"}\NormalTok{)])}
\end{Highlighting}
\end{Shaded}

\begin{verbatim}
## [1] "c" "h"
\end{verbatim}

También es posible excluir ciertos elementos en lugar de seleccionarlos.

\begin{Shaded}
\begin{Highlighting}[]
\KeywordTok{print}\NormalTok{(miVec20[}\OperatorTok{-}\DecValTok{1}\NormalTok{])}
\end{Highlighting}
\end{Shaded}

\begin{verbatim}
## [1] 20 30 40 50 60 70 80 90
\end{verbatim}

\begin{Shaded}
\begin{Highlighting}[]
\KeywordTok{print}\NormalTok{(miVec21[}\OperatorTok{-}\DecValTok{5}\NormalTok{])}
\end{Highlighting}
\end{Shaded}

\begin{verbatim}
## [1] "a" "b" "c" "d" "f" "g" "h" "i"
\end{verbatim}

\begin{Shaded}
\begin{Highlighting}[]
\KeywordTok{print}\NormalTok{(miVec20[}\OperatorTok{-}\KeywordTok{c}\NormalTok{(}\DecValTok{1}\NormalTok{, }\DecValTok{2}\NormalTok{, }\DecValTok{5}\NormalTok{)])}
\end{Highlighting}
\end{Shaded}

\begin{verbatim}
## [1] 30 40 60 70 80 90
\end{verbatim}

\begin{Shaded}
\begin{Highlighting}[]
\KeywordTok{print}\NormalTok{(miVec21[}\OperatorTok{-}\KeywordTok{c}\NormalTok{(}\DecValTok{1}\NormalTok{, }\DecValTok{2}\NormalTok{, }\DecValTok{5}\NormalTok{)])}
\end{Highlighting}
\end{Shaded}

\begin{verbatim}
## [1] "c" "d" "f" "g" "h" "i"
\end{verbatim}

Los elementos de un \texttt{vector} también se pueden seleccionar sobre
la base de un \texttt{vector} tipo \texttt{logical}. En este caso, solo
se seleccionarán elementos con un valor \texttt{TRUE}.

\begin{Shaded}
\begin{Highlighting}[]
\NormalTok{miVec22 <-}\StringTok{ }\KeywordTok{c}\NormalTok{(}\OtherTok{TRUE}\NormalTok{, }\OtherTok{TRUE}\NormalTok{, }\OtherTok{FALSE}\NormalTok{, }\OtherTok{TRUE}\NormalTok{, }\OtherTok{FALSE}\NormalTok{, }\OtherTok{TRUE}\NormalTok{, }\OtherTok{FALSE}\NormalTok{, }\OtherTok{TRUE}\NormalTok{, }\OtherTok{TRUE}\NormalTok{)}
\KeywordTok{print}\NormalTok{(miVec21[miVec22])}
\end{Highlighting}
\end{Shaded}

\begin{verbatim}
## [1] "a" "b" "d" "f" "h" "i"
\end{verbatim}

\subsection{\texorpdfstring{Dar nombres a los elementos de un
\texttt{vector}}{Dar nombres a los elementos de un vector}}\label{dar-nombres-a-los-elementos-de-un-vector}

Los elementos de un \texttt{vector} se pueden nombrar para
referenciarlos y luego selectionarlos. La función \texttt{names()}
recupera los nombres de los elementos de un vector.

\begin{Shaded}
\begin{Highlighting}[]
\NormalTok{miVec23 <-}\StringTok{ }\KeywordTok{c}\NormalTok{(}\DataTypeTok{aaa =} \DecValTok{10}\NormalTok{, }\DataTypeTok{bbb =} \DecValTok{20}\NormalTok{, }\DataTypeTok{ccc =} \DecValTok{30}\NormalTok{, }\DataTypeTok{ddd =} \DecValTok{40}\NormalTok{, }\DataTypeTok{eee =} \DecValTok{50}\NormalTok{)}
\KeywordTok{print}\NormalTok{(miVec23)}
\end{Highlighting}
\end{Shaded}

\begin{verbatim}
## aaa bbb ccc ddd eee 
##  10  20  30  40  50
\end{verbatim}

\begin{Shaded}
\begin{Highlighting}[]
\KeywordTok{print}\NormalTok{(miVec23[}\StringTok{"bbb"}\NormalTok{])}
\end{Highlighting}
\end{Shaded}

\begin{verbatim}
## bbb 
##  20
\end{verbatim}

\begin{Shaded}
\begin{Highlighting}[]
\KeywordTok{print}\NormalTok{(miVec23[}\KeywordTok{c}\NormalTok{(}\StringTok{"bbb"}\NormalTok{, }\StringTok{"ccc"}\NormalTok{, }\StringTok{"bbb"}\NormalTok{)])}
\end{Highlighting}
\end{Shaded}

\begin{verbatim}
## bbb ccc bbb 
##  20  30  20
\end{verbatim}

\begin{Shaded}
\begin{Highlighting}[]
\KeywordTok{names}\NormalTok{(miVec23)}
\end{Highlighting}
\end{Shaded}

\begin{verbatim}
## [1] "aaa" "bbb" "ccc" "ddd" "eee"
\end{verbatim}

\subsection{\texorpdfstring{Editar los elementos de un
\texttt{vector}}{Editar los elementos de un vector}}\label{editar-los-elementos-de-un-vector}

Para modificar un \texttt{vector}, operamos de la misma manera que para
modificar un objeto simple, con el signo \texttt{\textless{}-} y el
elemento o los elementos a modificar entre corchetes.

\begin{Shaded}
\begin{Highlighting}[]
\KeywordTok{print}\NormalTok{(miVec21)}
\end{Highlighting}
\end{Shaded}

\begin{verbatim}
## [1] "a" "b" "c" "d" "e" "f" "g" "h" "i"
\end{verbatim}

\begin{Shaded}
\begin{Highlighting}[]
\NormalTok{miVec21[}\DecValTok{3}\NormalTok{] <-}\StringTok{ "zzz"}
\KeywordTok{print}\NormalTok{(miVec21)}
\end{Highlighting}
\end{Shaded}

\begin{verbatim}
## [1] "a"   "b"   "zzz" "d"   "e"   "f"   "g"   "h"   "i"
\end{verbatim}

\begin{Shaded}
\begin{Highlighting}[]
\NormalTok{miVec21[(miVec20 }\OperatorTok{>=}\StringTok{ }\DecValTok{50}\NormalTok{) }\OperatorTok{&}\StringTok{ }\NormalTok{((miVec20 }\OperatorTok{<}\StringTok{ }\DecValTok{80}\NormalTok{))] <-}\StringTok{ "qwerty"}
\KeywordTok{print}\NormalTok{(miVec21)}
\end{Highlighting}
\end{Shaded}

\begin{verbatim}
## [1] "a"      "b"      "zzz"    "d"      "qwerty" "qwerty" "qwerty" "h"     
## [9] "i"
\end{verbatim}

\begin{Shaded}
\begin{Highlighting}[]
\KeywordTok{print}\NormalTok{(miVec23)}
\end{Highlighting}
\end{Shaded}

\begin{verbatim}
## aaa bbb ccc ddd eee 
##  10  20  30  40  50
\end{verbatim}

\begin{Shaded}
\begin{Highlighting}[]
\NormalTok{miVec23[}\StringTok{"ccc"}\NormalTok{] <-}\StringTok{ }\NormalTok{miVec23[}\StringTok{"ccc"}\NormalTok{] }\OperatorTok{+}\StringTok{ }\DecValTok{100}
\KeywordTok{print}\NormalTok{(miVec23)}
\end{Highlighting}
\end{Shaded}

\begin{verbatim}
## aaa bbb ccc ddd eee 
##  10  20 130  40  50
\end{verbatim}

También podemos cambiar los nombres asociados con los elementos de un
\texttt{vector}.

\begin{Shaded}
\begin{Highlighting}[]
\KeywordTok{print}\NormalTok{(miVec23)}
\end{Highlighting}
\end{Shaded}

\begin{verbatim}
## aaa bbb ccc ddd eee 
##  10  20 130  40  50
\end{verbatim}

\begin{Shaded}
\begin{Highlighting}[]
\KeywordTok{names}\NormalTok{(miVec23)[}\DecValTok{2}\NormalTok{] <-}\StringTok{ "bb_bb"}
\KeywordTok{print}\NormalTok{(miVec23)}
\end{Highlighting}
\end{Shaded}

\begin{verbatim}
##   aaa bb_bb   ccc   ddd   eee 
##    10    20   130    40    50
\end{verbatim}

Podemos hacer mucho más con un \texttt{vector} y volveremos a su manejo
y operaciones posibles en el capítulo sobre funciones.

\hypertarget{l014list}{\section{\texorpdfstring{El contenedor
\texttt{list}}{El contenedor list}}\label{l014list}}

El segundo tipo de contenedor que vamos a presentar es el contenedor
\texttt{list}, que es también el segundo contenedor después del
tipo\texttt{vector} debido a su importancia en la programación con R. El
contenedor de tipo \texttt{list} le permite almacenar \textbf{listas} de
elementos. Contrariamente a lo que vimos antes con el tipo
\texttt{vector}, los elementos del tipo \texttt{list} pueden ser
diferentes (por ejemplo, un \texttt{vector} de tipo \texttt{numeric},
luego un \texttt{vector} de tipo \texttt{character}). Los elementos del
tipo \texttt{list} también pueden ser contenedores diferentes (por
ejemplo, un \texttt{vector}, luego una \texttt{list}). El tipo de
contenedor \texttt{list} tendrá mas sentido cuando hayamos estudiado los
\textbf{bucles} y \textbf{funciones} de la familia \texttt{apply}.

\subsection{\texorpdfstring{Crear una
\texttt{list}}{Crear una list}}\label{crear-una-list}

Para crear una \texttt{list} usaremos la función \texttt{list()}, que
toma elementos (objetos) como argumentos.

\begin{Shaded}
\begin{Highlighting}[]
\NormalTok{miList01 <-}\StringTok{ }\KeywordTok{list}\NormalTok{()}
\KeywordTok{print}\NormalTok{(miList01)}
\end{Highlighting}
\end{Shaded}

\begin{verbatim}
## list()
\end{verbatim}

\begin{Shaded}
\begin{Highlighting}[]
\NormalTok{miList02 <-}\StringTok{ }\KeywordTok{list}\NormalTok{(}\DecValTok{5}\NormalTok{, }\StringTok{"qwerty"}\NormalTok{, }\KeywordTok{c}\NormalTok{(}\DecValTok{4}\NormalTok{, }\DecValTok{5}\NormalTok{, }\DecValTok{6}\NormalTok{), }\KeywordTok{c}\NormalTok{(}\StringTok{"a"}\NormalTok{, }\StringTok{"b"}\NormalTok{, }\StringTok{"c"}\NormalTok{))}
\KeywordTok{print}\NormalTok{(miList02)}
\end{Highlighting}
\end{Shaded}

\begin{verbatim}
## [[1]]
## [1] 5
## 
## [[2]]
## [1] "qwerty"
## 
## [[3]]
## [1] 4 5 6
## 
## [[4]]
## [1] "a" "b" "c"
\end{verbatim}

\begin{Shaded}
\begin{Highlighting}[]
\NormalTok{miList03 <-}\StringTok{ }\KeywordTok{list}\NormalTok{(}\DecValTok{5}\NormalTok{, }\StringTok{"qwerty"}\NormalTok{, }\KeywordTok{list}\NormalTok{(}\KeywordTok{c}\NormalTok{(}\DecValTok{4}\NormalTok{, }\DecValTok{5}\NormalTok{, }\DecValTok{6}\NormalTok{), }\KeywordTok{c}\NormalTok{(}\StringTok{"a"}\NormalTok{, }\StringTok{"b"}\NormalTok{, }\StringTok{"c"}\NormalTok{)))}
\KeywordTok{print}\NormalTok{(miList03)}
\end{Highlighting}
\end{Shaded}

\begin{verbatim}
## [[1]]
## [1] 5
## 
## [[2]]
## [1] "qwerty"
## 
## [[3]]
## [[3]][[1]]
## [1] 4 5 6
## 
## [[3]][[2]]
## [1] "a" "b" "c"
\end{verbatim}

La función \texttt{is.list()} se usa para probar si hemos creado un
objeto de tipo \texttt{list}.

\begin{Shaded}
\begin{Highlighting}[]
\KeywordTok{is.list}\NormalTok{(miList02)}
\end{Highlighting}
\end{Shaded}

\begin{verbatim}
## [1] TRUE
\end{verbatim}

\begin{Shaded}
\begin{Highlighting}[]
\KeywordTok{typeof}\NormalTok{(miList02)}
\end{Highlighting}
\end{Shaded}

\begin{verbatim}
## [1] "list"
\end{verbatim}

\subsection{\texorpdfstring{Acceder a los valores de una
\texttt{list}}{Acceder a los valores de una list}}\label{acceder-a-los-valores-de-una-list}

Los elementos del contenedor \texttt{list} son identificables por los
corchetes dobles \texttt{{[}{[}\ {]}{]}}.

\begin{Shaded}
\begin{Highlighting}[]
\KeywordTok{print}\NormalTok{(miList02)}
\end{Highlighting}
\end{Shaded}

\begin{verbatim}
## [[1]]
## [1] 5
## 
## [[2]]
## [1] "qwerty"
## 
## [[3]]
## [1] 4 5 6
## 
## [[4]]
## [1] "a" "b" "c"
\end{verbatim}

En el objeto de tipo \texttt{list} \texttt{miList02}, hay cuatro
elementos identificables con \texttt{{[}{[}1{]}{]}},
\texttt{{[}{[}2{]}{]}}, \texttt{{[}{[}3{]}{]}} y \texttt{{[}{[}4{]}{]}}.
Cada uno de los elementos es de tipo \texttt{vector}. El primer elemento
tiene un tamaño de 1 con elementos del tipo \texttt{double}, el segundo
elemento tiene un tamaño de 1 con elementos del tipo \texttt{character},
el tercero elemento tiene un tamaño de 3 con elementos del tipo
\texttt{double}, y el cuarto elemento tiene un tamaño de 3 con elementos
del tipo \texttt{character}.

\begin{Shaded}
\begin{Highlighting}[]
\KeywordTok{typeof}\NormalTok{(miList02)}
\end{Highlighting}
\end{Shaded}

\begin{verbatim}
## [1] "list"
\end{verbatim}

\begin{Shaded}
\begin{Highlighting}[]
\KeywordTok{print}\NormalTok{(miList02[[}\DecValTok{1}\NormalTok{]])}
\end{Highlighting}
\end{Shaded}

\begin{verbatim}
## [1] 5
\end{verbatim}

\begin{Shaded}
\begin{Highlighting}[]
\KeywordTok{typeof}\NormalTok{(miList02[[}\DecValTok{1}\NormalTok{]])}
\end{Highlighting}
\end{Shaded}

\begin{verbatim}
## [1] "double"
\end{verbatim}

\begin{Shaded}
\begin{Highlighting}[]
\KeywordTok{print}\NormalTok{(miList02[[}\DecValTok{2}\NormalTok{]])}
\end{Highlighting}
\end{Shaded}

\begin{verbatim}
## [1] "qwerty"
\end{verbatim}

\begin{Shaded}
\begin{Highlighting}[]
\KeywordTok{typeof}\NormalTok{(miList02[[}\DecValTok{2}\NormalTok{]])}
\end{Highlighting}
\end{Shaded}

\begin{verbatim}
## [1] "character"
\end{verbatim}

\begin{Shaded}
\begin{Highlighting}[]
\KeywordTok{print}\NormalTok{(miList02[[}\DecValTok{3}\NormalTok{]])}
\end{Highlighting}
\end{Shaded}

\begin{verbatim}
## [1] 4 5 6
\end{verbatim}

\begin{Shaded}
\begin{Highlighting}[]
\KeywordTok{typeof}\NormalTok{(miList02[[}\DecValTok{3}\NormalTok{]])}
\end{Highlighting}
\end{Shaded}

\begin{verbatim}
## [1] "double"
\end{verbatim}

\begin{Shaded}
\begin{Highlighting}[]
\KeywordTok{print}\NormalTok{(miList02[[}\DecValTok{4}\NormalTok{]])}
\end{Highlighting}
\end{Shaded}

\begin{verbatim}
## [1] "a" "b" "c"
\end{verbatim}

\begin{Shaded}
\begin{Highlighting}[]
\KeywordTok{typeof}\NormalTok{(miList02[[}\DecValTok{4}\NormalTok{]])}
\end{Highlighting}
\end{Shaded}

\begin{verbatim}
## [1] "character"
\end{verbatim}

El acceso al segundo elemento del \texttt{vector} ubicado en la cuarta
posición de la \texttt{list} se hace con
\texttt{miList02{[}{[}4{]}{]}{[}2{]}}. Usamos doble corchetes para el
cuarto elemento de la \texttt{list}, luego corchetes simples para el
segundo elemento del \texttt{vector}.

\begin{Shaded}
\begin{Highlighting}[]
\KeywordTok{print}\NormalTok{(miList02[[}\DecValTok{4}\NormalTok{]][}\DecValTok{2}\NormalTok{])}
\end{Highlighting}
\end{Shaded}

\begin{verbatim}
## [1] "b"
\end{verbatim}

Como una \texttt{list} puede contener una o más \texttt{list}, podemos
acceder a la información buscada combinando corchetes dobles. El objeto
\texttt{miList04} es una \texttt{list} de dos elementos: la
\texttt{list} \texttt{miList02} y la \texttt{list} \texttt{miList03}. El
objeto \texttt{miList03} en sí contiene una \texttt{list} como tercer
elemento. Para acceder al primer elemento del \texttt{vector} en la
primera posición del elemento en la tercera posición del segundo
elemento del \texttt{list} \texttt{miList04}, podemos usar
\texttt{miList04{[}{[}2{]}{]}{[}{[}3{]}{]}{[}{[}1{]}{]}{[}1{]}}. No hay
límite en cuanto a la profundidad de \texttt{list} pero en la práctica
raramente hay necesidad de hacer \texttt{list} de \texttt{list} de
\texttt{list}.

\begin{Shaded}
\begin{Highlighting}[]
\NormalTok{miList04 <-}\StringTok{ }\KeywordTok{list}\NormalTok{(miList02, miList03)}
\KeywordTok{print}\NormalTok{(miList04)}
\end{Highlighting}
\end{Shaded}

\begin{verbatim}
## [[1]]
## [[1]][[1]]
## [1] 5
## 
## [[1]][[2]]
## [1] "qwerty"
## 
## [[1]][[3]]
## [1] 4 5 6
## 
## [[1]][[4]]
## [1] "a" "b" "c"
## 
## 
## [[2]]
## [[2]][[1]]
## [1] 5
## 
## [[2]][[2]]
## [1] "qwerty"
## 
## [[2]][[3]]
## [[2]][[3]][[1]]
## [1] 4 5 6
## 
## [[2]][[3]][[2]]
## [1] "a" "b" "c"
\end{verbatim}

\begin{Shaded}
\begin{Highlighting}[]
\KeywordTok{print}\NormalTok{(miList04[[}\DecValTok{2}\NormalTok{]][[}\DecValTok{3}\NormalTok{]][[}\DecValTok{1}\NormalTok{]][}\DecValTok{1}\NormalTok{])}
\end{Highlighting}
\end{Shaded}

\begin{verbatim}
## [1] 4
\end{verbatim}

Para concretar el ejemplo anterior, podemos imaginar especies de
barrenadores del maíz (\emph{Sesamia nonagrioides} y \emph{Ostrinia
nubilalis}), muestreados en diferentes sitios, con diferentes
abundancias en cuatro fechas. Aquí daremos nombres a los elementos de
las \texttt{list}.

\begin{Shaded}
\begin{Highlighting}[]
\NormalTok{bddInsect <-}\StringTok{ }\KeywordTok{list}\NormalTok{(}\DataTypeTok{Snonagrioides =} \KeywordTok{list}\NormalTok{(}\DataTypeTok{site01 =} \KeywordTok{c}\NormalTok{(}\DecValTok{12}\NormalTok{, }\DecValTok{5}\NormalTok{, }\DecValTok{8}\NormalTok{, }\DecValTok{7}\NormalTok{), }\DataTypeTok{site02 =} \KeywordTok{c}\NormalTok{(}\DecValTok{5}\NormalTok{, }\DecValTok{23}\NormalTok{, }\DecValTok{4}\NormalTok{, }\DecValTok{41}\NormalTok{), }\DataTypeTok{site03 =} \KeywordTok{c}\NormalTok{(}\DecValTok{12}\NormalTok{, }\DecValTok{0}\NormalTok{, }\DecValTok{0}\NormalTok{, }\DecValTok{0}\NormalTok{)), }\DataTypeTok{Onubilalis =} \KeywordTok{list}\NormalTok{(}\DataTypeTok{site01 =} \KeywordTok{c}\NormalTok{(}\DecValTok{12}\NormalTok{, }\DecValTok{1}\NormalTok{, }\DecValTok{2}\NormalTok{, }\DecValTok{3}\NormalTok{), }\DataTypeTok{site02 =} \KeywordTok{c}\NormalTok{(}\DecValTok{0}\NormalTok{, }\DecValTok{0}\NormalTok{, }\DecValTok{0}\NormalTok{, }\DecValTok{1}\NormalTok{), }\DataTypeTok{site03 =} \KeywordTok{c}\NormalTok{(}\DecValTok{1}\NormalTok{, }\DecValTok{1}\NormalTok{, }\DecValTok{2}\NormalTok{, }\DecValTok{3}\NormalTok{)))}
\KeywordTok{print}\NormalTok{(bddInsect)}
\end{Highlighting}
\end{Shaded}

\begin{verbatim}
## $Snonagrioides
## $Snonagrioides$site01
## [1] 12  5  8  7
## 
## $Snonagrioides$site02
## [1]  5 23  4 41
## 
## $Snonagrioides$site03
## [1] 12  0  0  0
## 
## 
## $Onubilalis
## $Onubilalis$site01
## [1] 12  1  2  3
## 
## $Onubilalis$site02
## [1] 0 0 0 1
## 
## $Onubilalis$site03
## [1] 1 1 2 3
\end{verbatim}

\begin{rmdstyle}
Leer una larga línea de código como la línea para crear el objeto
\texttt{bddInsect} resulta difícil porque la profundidad de los
elementos solo se puede deducir de los paréntesis. Es por eso que vamos
a reorganizar el código para que sea más legible mediante el
\textbf{margen adicional}. El margen adicional implica poner información
en diferentes niveles para que podamos identificar rápidamente los
diferentes niveles de un código. Para aplicar el margen adicional se
presiona la tecla de tabulación. Volveremos al margen adicional con más
detalles en el capítulo sobre \textbf{bucles}. Recordemos por el momento
que si una línea de código es demasiado larga, podemos saltar de línea y
usar el margen adicional. R leerá todo como una sola línea de código.
\end{rmdstyle}

\begin{Shaded}
\begin{Highlighting}[]
\NormalTok{bddInsect <-}\StringTok{ }\KeywordTok{list}\NormalTok{(}
  \DataTypeTok{Snonagrioides =} \KeywordTok{list}\NormalTok{(}
    \DataTypeTok{site01 =} \KeywordTok{c}\NormalTok{(}\DecValTok{12}\NormalTok{, }\DecValTok{5}\NormalTok{, }\DecValTok{8}\NormalTok{, }\DecValTok{7}\NormalTok{), }
    \DataTypeTok{site02 =} \KeywordTok{c}\NormalTok{(}\DecValTok{5}\NormalTok{, }\DecValTok{23}\NormalTok{, }\DecValTok{4}\NormalTok{, }\DecValTok{41}\NormalTok{), }
    \DataTypeTok{site03 =} \KeywordTok{c}\NormalTok{(}\DecValTok{12}\NormalTok{, }\DecValTok{0}\NormalTok{, }\DecValTok{0}\NormalTok{, }\DecValTok{0}\NormalTok{)}
\NormalTok{  ), }
  \DataTypeTok{Onubilalis =} \KeywordTok{list}\NormalTok{(}
    \DataTypeTok{site01 =} \KeywordTok{c}\NormalTok{(}\DecValTok{12}\NormalTok{, }\DecValTok{1}\NormalTok{, }\DecValTok{2}\NormalTok{, }\DecValTok{3}\NormalTok{), }
    \DataTypeTok{site02 =} \KeywordTok{c}\NormalTok{(}\DecValTok{0}\NormalTok{, }\DecValTok{0}\NormalTok{, }\DecValTok{0}\NormalTok{, }\DecValTok{1}\NormalTok{), }
    \DataTypeTok{site03 =} \KeywordTok{c}\NormalTok{(}\DecValTok{1}\NormalTok{, }\DecValTok{1}\NormalTok{, }\DecValTok{2}\NormalTok{, }\DecValTok{3}\NormalTok{)}
\NormalTok{  )}
\NormalTok{)}
\end{Highlighting}
\end{Shaded}

Podemos seleccionar los datos de abundancia del segundo sitio de la
primera especie como previamente
\texttt{bddInsect{[}{[}1{]}{]}{[}{[}2{]}{]}}, o alternativamente usando
los nombres de los elementos \texttt{bddInsect\$Snonagrioides\$site02}.
Para hacer esto usamos el signo \texttt{\$}, o como alternativa el
nombre de los elementos con comillas simples o dobles
\texttt{bddInsect{[}{[}\textquotesingle{}Snonagrioides\textquotesingle{}{]}{]}{[}{[}\textquotesingle{}sitio02\textquotesingle{}{]}{]}}.

\begin{Shaded}
\begin{Highlighting}[]
\KeywordTok{print}\NormalTok{(bddInsect[[}\DecValTok{1}\NormalTok{]][[}\DecValTok{2}\NormalTok{]])}
\end{Highlighting}
\end{Shaded}

\begin{verbatim}
## [1]  5 23  4 41
\end{verbatim}

\begin{Shaded}
\begin{Highlighting}[]
\KeywordTok{print}\NormalTok{(bddInsect}\OperatorTok{$}\NormalTok{Snonagrioides}\OperatorTok{$}\NormalTok{site02)}
\end{Highlighting}
\end{Shaded}

\begin{verbatim}
## [1]  5 23  4 41
\end{verbatim}

\begin{Shaded}
\begin{Highlighting}[]
\KeywordTok{print}\NormalTok{(bddInsect[[}\StringTok{'Snonagrioides'}\NormalTok{]][[}\StringTok{'site02'}\NormalTok{]])}
\end{Highlighting}
\end{Shaded}

\begin{verbatim}
## [1]  5 23  4 41
\end{verbatim}

En cuanto a los vectores, podemos recuperar los nombres de los elementos
con la función \texttt{names()}.

\begin{Shaded}
\begin{Highlighting}[]
\KeywordTok{names}\NormalTok{(bddInsect)}
\end{Highlighting}
\end{Shaded}

\begin{verbatim}
## [1] "Snonagrioides" "Onubilalis"
\end{verbatim}

\begin{Shaded}
\begin{Highlighting}[]
\KeywordTok{names}\NormalTok{(bddInsect[[}\DecValTok{1}\NormalTok{]])}
\end{Highlighting}
\end{Shaded}

\begin{verbatim}
## [1] "site01" "site02" "site03"
\end{verbatim}

Cuando usamos los corchetes dobles \texttt{{[}{[}{]}{]}} o el signo
\texttt{\$}, R devuelve el contenido del elemento seleccionado. En
nuestro ejemplo, los datos de abundancia están contenidos como un
\texttt{vector}, por lo que R devuelve un elemento del tipo
\texttt{vector}. Si queremos seleccionar un elemento de una
\texttt{list} pero manteniendo el formato \texttt{list}, entonces
podemos usar corchetes simples \texttt{{[}{]}}.

\begin{Shaded}
\begin{Highlighting}[]
\KeywordTok{print}\NormalTok{(bddInsect[[}\DecValTok{1}\NormalTok{]][[}\DecValTok{2}\NormalTok{]])}
\end{Highlighting}
\end{Shaded}

\begin{verbatim}
## [1]  5 23  4 41
\end{verbatim}

\begin{Shaded}
\begin{Highlighting}[]
\KeywordTok{typeof}\NormalTok{(bddInsect[[}\DecValTok{1}\NormalTok{]][[}\DecValTok{2}\NormalTok{]])}
\end{Highlighting}
\end{Shaded}

\begin{verbatim}
## [1] "double"
\end{verbatim}

\begin{Shaded}
\begin{Highlighting}[]
\KeywordTok{is.list}\NormalTok{(bddInsect[[}\DecValTok{1}\NormalTok{]][[}\DecValTok{2}\NormalTok{]])}
\end{Highlighting}
\end{Shaded}

\begin{verbatim}
## [1] FALSE
\end{verbatim}

\begin{Shaded}
\begin{Highlighting}[]
\KeywordTok{print}\NormalTok{(bddInsect[[}\DecValTok{1}\NormalTok{]][}\DecValTok{2}\NormalTok{])}
\end{Highlighting}
\end{Shaded}

\begin{verbatim}
## $site02
## [1]  5 23  4 41
\end{verbatim}

\begin{Shaded}
\begin{Highlighting}[]
\KeywordTok{typeof}\NormalTok{(bddInsect[[}\DecValTok{1}\NormalTok{]][}\DecValTok{2}\NormalTok{])}
\end{Highlighting}
\end{Shaded}

\begin{verbatim}
## [1] "list"
\end{verbatim}

\begin{Shaded}
\begin{Highlighting}[]
\KeywordTok{is.list}\NormalTok{(bddInsect[[}\DecValTok{1}\NormalTok{]][}\DecValTok{2}\NormalTok{])}
\end{Highlighting}
\end{Shaded}

\begin{verbatim}
## [1] TRUE
\end{verbatim}

El uso de corchetes simples \texttt{{[}{]}} es útil cuando queremos
recuperar varios elementos de una \texttt{list}. Por ejemplo, para
seleccionar las abundancias de insectos de los primeros dos sitios de la
primera especie, usaremos
\texttt{bddInsect\ {[}{[}1{]}{]}{[}c(1,\ 2){]}} o alternativamente
\texttt{bddInsect{[}{[}1{]}{]}{[}c("site01",\ "sitio02"){]}}.

\begin{Shaded}
\begin{Highlighting}[]
\KeywordTok{print}\NormalTok{(bddInsect[[}\DecValTok{1}\NormalTok{]][}\KeywordTok{c}\NormalTok{(}\DecValTok{1}\NormalTok{, }\DecValTok{2}\NormalTok{)])}
\end{Highlighting}
\end{Shaded}

\begin{verbatim}
## $site01
## [1] 12  5  8  7
## 
## $site02
## [1]  5 23  4 41
\end{verbatim}

\begin{Shaded}
\begin{Highlighting}[]
\KeywordTok{print}\NormalTok{(bddInsect[[}\DecValTok{1}\NormalTok{]][}\KeywordTok{c}\NormalTok{(}\StringTok{"site01"}\NormalTok{, }\StringTok{"site02"}\NormalTok{)])}
\end{Highlighting}
\end{Shaded}

\begin{verbatim}
## $site01
## [1] 12  5  8  7
## 
## $site02
## [1]  5 23  4 41
\end{verbatim}

\subsection{\texorpdfstring{Editar una
\texttt{list}}{Editar una list}}\label{editar-una-list}

Una \texttt{list} se puede modificar de la misma manera que para el
contenedor \texttt{vector}, es decir, haciendo referencia con corchetes
al elemento que queremos modificar.

\begin{Shaded}
\begin{Highlighting}[]
\KeywordTok{print}\NormalTok{(miList02)}
\end{Highlighting}
\end{Shaded}

\begin{verbatim}
## [[1]]
## [1] 5
## 
## [[2]]
## [1] "qwerty"
## 
## [[3]]
## [1] 4 5 6
## 
## [[4]]
## [1] "a" "b" "c"
\end{verbatim}

\begin{Shaded}
\begin{Highlighting}[]
\NormalTok{miList02[[}\DecValTok{1}\NormalTok{]] <-}\StringTok{ }\DecValTok{12}
\KeywordTok{print}\NormalTok{(miList02)}
\end{Highlighting}
\end{Shaded}

\begin{verbatim}
## [[1]]
## [1] 12
## 
## [[2]]
## [1] "qwerty"
## 
## [[3]]
## [1] 4 5 6
## 
## [[4]]
## [1] "a" "b" "c"
\end{verbatim}

\begin{Shaded}
\begin{Highlighting}[]
\NormalTok{miList02[[}\DecValTok{4}\NormalTok{]] <-}\StringTok{ }\KeywordTok{c}\NormalTok{(}\StringTok{"d"}\NormalTok{, }\StringTok{"e"}\NormalTok{, }\StringTok{"f"}\NormalTok{)}
\KeywordTok{print}\NormalTok{(miList02)}
\end{Highlighting}
\end{Shaded}

\begin{verbatim}
## [[1]]
## [1] 12
## 
## [[2]]
## [1] "qwerty"
## 
## [[3]]
## [1] 4 5 6
## 
## [[4]]
## [1] "d" "e" "f"
\end{verbatim}

\begin{Shaded}
\begin{Highlighting}[]
\NormalTok{miList02[[}\DecValTok{4}\NormalTok{]] <-}\StringTok{ }\KeywordTok{c}\NormalTok{(}\StringTok{"a"}\NormalTok{, }\StringTok{"b"}\NormalTok{, }\StringTok{"c"}\NormalTok{, miList02[[}\DecValTok{4}\NormalTok{]], }\StringTok{"g"}\NormalTok{, }\StringTok{"h"}\NormalTok{, }\StringTok{"i"}\NormalTok{)}
\KeywordTok{print}\NormalTok{(miList02)}
\end{Highlighting}
\end{Shaded}

\begin{verbatim}
## [[1]]
## [1] 12
## 
## [[2]]
## [1] "qwerty"
## 
## [[3]]
## [1] 4 5 6
## 
## [[4]]
## [1] "a" "b" "c" "d" "e" "f" "g" "h" "i"
\end{verbatim}

\begin{Shaded}
\begin{Highlighting}[]
\NormalTok{miList02[[}\DecValTok{4}\NormalTok{]][}\DecValTok{5}\NormalTok{] <-}\StringTok{ "eee"}
\KeywordTok{print}\NormalTok{(miList02)}
\end{Highlighting}
\end{Shaded}

\begin{verbatim}
## [[1]]
## [1] 12
## 
## [[2]]
## [1] "qwerty"
## 
## [[3]]
## [1] 4 5 6
## 
## [[4]]
## [1] "a"   "b"   "c"   "d"   "eee" "f"   "g"   "h"   "i"
\end{verbatim}

\begin{Shaded}
\begin{Highlighting}[]
\NormalTok{miList02[[}\DecValTok{3}\NormalTok{]] <-}\StringTok{ }\NormalTok{miList02[[}\DecValTok{3}\NormalTok{]] }\OperatorTok{*}\StringTok{ }\DecValTok{10} \OperatorTok{-}\StringTok{ }\DecValTok{1}
\KeywordTok{print}\NormalTok{(miList02)}
\end{Highlighting}
\end{Shaded}

\begin{verbatim}
## [[1]]
## [1] 12
## 
## [[2]]
## [1] "qwerty"
## 
## [[3]]
## [1] 39 49 59
## 
## [[4]]
## [1] "a"   "b"   "c"   "d"   "eee" "f"   "g"   "h"   "i"
\end{verbatim}

\begin{Shaded}
\begin{Highlighting}[]
\NormalTok{miList02[[}\DecValTok{3}\NormalTok{]][}\DecValTok{2}\NormalTok{] <-}\StringTok{ }\NormalTok{miList02[[}\DecValTok{1}\NormalTok{]] }\OperatorTok{*}\StringTok{ }\DecValTok{100}
\KeywordTok{print}\NormalTok{(miList02)}
\end{Highlighting}
\end{Shaded}

\begin{verbatim}
## [[1]]
## [1] 12
## 
## [[2]]
## [1] "qwerty"
## 
## [[3]]
## [1]   39 1200   59
## 
## [[4]]
## [1] "a"   "b"   "c"   "d"   "eee" "f"   "g"   "h"   "i"
\end{verbatim}

\begin{Shaded}
\begin{Highlighting}[]
\KeywordTok{print}\NormalTok{(bddInsect)}
\end{Highlighting}
\end{Shaded}

\begin{verbatim}
## $Snonagrioides
## $Snonagrioides$site01
## [1] 12  5  8  7
## 
## $Snonagrioides$site02
## [1]  5 23  4 41
## 
## $Snonagrioides$site03
## [1] 12  0  0  0
## 
## 
## $Onubilalis
## $Onubilalis$site01
## [1] 12  1  2  3
## 
## $Onubilalis$site02
## [1] 0 0 0 1
## 
## $Onubilalis$site03
## [1] 1 1 2 3
\end{verbatim}

\begin{Shaded}
\begin{Highlighting}[]
\NormalTok{bddInsect[[}\StringTok{'Snonagrioides'}\NormalTok{]][[}\StringTok{'site02'}\NormalTok{]] <-}\StringTok{ }\KeywordTok{c}\NormalTok{(}\DecValTok{2}\NormalTok{, }\DecValTok{4}\NormalTok{, }\DecValTok{6}\NormalTok{, }\DecValTok{8}\NormalTok{)}
\KeywordTok{print}\NormalTok{(bddInsect)}
\end{Highlighting}
\end{Shaded}

\begin{verbatim}
## $Snonagrioides
## $Snonagrioides$site01
## [1] 12  5  8  7
## 
## $Snonagrioides$site02
## [1] 2 4 6 8
## 
## $Snonagrioides$site03
## [1] 12  0  0  0
## 
## 
## $Onubilalis
## $Onubilalis$site01
## [1] 12  1  2  3
## 
## $Onubilalis$site02
## [1] 0 0 0 1
## 
## $Onubilalis$site03
## [1] 1 1 2 3
\end{verbatim}

Para combinar dos \texttt{list}, simplemente usamos la función
\texttt{c()} que hemos usado para crear un \texttt{vector}.

\begin{Shaded}
\begin{Highlighting}[]
\NormalTok{miList0203 <-}\StringTok{ }\KeywordTok{c}\NormalTok{(miList02, miList03)}
\KeywordTok{print}\NormalTok{(miList0203)}
\end{Highlighting}
\end{Shaded}

\begin{verbatim}
## [[1]]
## [1] 12
## 
## [[2]]
## [1] "qwerty"
## 
## [[3]]
## [1]   39 1200   59
## 
## [[4]]
## [1] "a"   "b"   "c"   "d"   "eee" "f"   "g"   "h"   "i"  
## 
## [[5]]
## [1] 5
## 
## [[6]]
## [1] "qwerty"
## 
## [[7]]
## [[7]][[1]]
## [1] 4 5 6
## 
## [[7]][[2]]
## [1] "a" "b" "c"
\end{verbatim}

Un objeto de tipo \texttt{list} se puede transformar en \texttt{vector}
con la función \texttt{unlist()} si el formato de los elementos de la
lista lo permite (un \texttt{vector} solo puede contener elementos del
mismo tipo).

\begin{Shaded}
\begin{Highlighting}[]
\NormalTok{miList05 <-}\StringTok{ }\KeywordTok{list}\NormalTok{(}\StringTok{"a"}\NormalTok{, }\KeywordTok{c}\NormalTok{(}\StringTok{"b"}\NormalTok{, }\StringTok{"c"}\NormalTok{), }\StringTok{"d"}\NormalTok{)}
\KeywordTok{print}\NormalTok{(miList05)}
\end{Highlighting}
\end{Shaded}

\begin{verbatim}
## [[1]]
## [1] "a"
## 
## [[2]]
## [1] "b" "c"
## 
## [[3]]
## [1] "d"
\end{verbatim}

\begin{Shaded}
\begin{Highlighting}[]
\NormalTok{miVec24 <-}\StringTok{ }\KeywordTok{unlist}\NormalTok{(miList05)}
\KeywordTok{print}\NormalTok{(miVec24)}
\end{Highlighting}
\end{Shaded}

\begin{verbatim}
## [1] "a" "b" "c" "d"
\end{verbatim}

\begin{Shaded}
\begin{Highlighting}[]
\NormalTok{miList06 <-}\StringTok{ }\KeywordTok{list}\NormalTok{(}\KeywordTok{c}\NormalTok{(}\DecValTok{1}\NormalTok{, }\DecValTok{2}\NormalTok{, }\DecValTok{3}\NormalTok{), }\KeywordTok{c}\NormalTok{(}\DecValTok{4}\NormalTok{, }\DecValTok{5}\NormalTok{, }\DecValTok{6}\NormalTok{, }\DecValTok{7}\NormalTok{), }\DecValTok{8}\NormalTok{, }\DecValTok{9}\NormalTok{, }\KeywordTok{c}\NormalTok{(}\DecValTok{10}\NormalTok{, }\DecValTok{11}\NormalTok{))}
\KeywordTok{print}\NormalTok{(miList06)}
\end{Highlighting}
\end{Shaded}

\begin{verbatim}
## [[1]]
## [1] 1 2 3
## 
## [[2]]
## [1] 4 5 6 7
## 
## [[3]]
## [1] 8
## 
## [[4]]
## [1] 9
## 
## [[5]]
## [1] 10 11
\end{verbatim}

\begin{Shaded}
\begin{Highlighting}[]
\NormalTok{miVec25 <-}\StringTok{ }\KeywordTok{unlist}\NormalTok{(miList06)}
\KeywordTok{print}\NormalTok{(miVec25)}
\end{Highlighting}
\end{Shaded}

\begin{verbatim}
##  [1]  1  2  3  4  5  6  7  8  9 10 11
\end{verbatim}

Para agregar un elemento a una \texttt{list}, podemos usar la función
\texttt{c()} o los corchetes dobles \texttt{{[}{[}\ {]}{]}}.

\begin{Shaded}
\begin{Highlighting}[]
\KeywordTok{print}\NormalTok{(miList05)}
\end{Highlighting}
\end{Shaded}

\begin{verbatim}
## [[1]]
## [1] "a"
## 
## [[2]]
## [1] "b" "c"
## 
## [[3]]
## [1] "d"
\end{verbatim}

\begin{Shaded}
\begin{Highlighting}[]
\NormalTok{miList05 <-}\StringTok{ }\KeywordTok{c}\NormalTok{(miList05, }\StringTok{"e"}\NormalTok{)}
\KeywordTok{print}\NormalTok{(miList05)}
\end{Highlighting}
\end{Shaded}

\begin{verbatim}
## [[1]]
## [1] "a"
## 
## [[2]]
## [1] "b" "c"
## 
## [[3]]
## [1] "d"
## 
## [[4]]
## [1] "e"
\end{verbatim}

\begin{Shaded}
\begin{Highlighting}[]
\NormalTok{miList05[[}\DecValTok{5}\NormalTok{]] <-}\StringTok{ }\KeywordTok{c}\NormalTok{(}\StringTok{"fgh"}\NormalTok{, }\StringTok{"ijk"}\NormalTok{)}
\KeywordTok{print}\NormalTok{(miList05)}
\end{Highlighting}
\end{Shaded}

\begin{verbatim}
## [[1]]
## [1] "a"
## 
## [[2]]
## [1] "b" "c"
## 
## [[3]]
## [1] "d"
## 
## [[4]]
## [1] "e"
## 
## [[5]]
## [1] "fgh" "ijk"
\end{verbatim}

Para eliminar un elemento de una \texttt{list}, la técnica más rápida es
establecer \texttt{NULL} en el elemento que deseamos eliminar.

\begin{Shaded}
\begin{Highlighting}[]
\KeywordTok{print}\NormalTok{(miList05)}
\end{Highlighting}
\end{Shaded}

\begin{verbatim}
## [[1]]
## [1] "a"
## 
## [[2]]
## [1] "b" "c"
## 
## [[3]]
## [1] "d"
## 
## [[4]]
## [1] "e"
## 
## [[5]]
## [1] "fgh" "ijk"
\end{verbatim}

\begin{Shaded}
\begin{Highlighting}[]
\NormalTok{miList05[[}\DecValTok{2}\NormalTok{]] <-}\StringTok{ }\OtherTok{NULL}
\KeywordTok{print}\NormalTok{(miList05)}
\end{Highlighting}
\end{Shaded}

\begin{verbatim}
## [[1]]
## [1] "a"
## 
## [[2]]
## [1] "d"
## 
## [[3]]
## [1] "e"
## 
## [[4]]
## [1] "fgh" "ijk"
\end{verbatim}

\hypertarget{l014dataframe}{\section{\texorpdfstring{El contenedor
\texttt{data.frame}}{El contenedor data.frame}}\label{l014dataframe}}

El contenedor \texttt{data.frame} se puede comparar a una \emph{tabla}.
Este es en realidad un caso especial de \texttt{list} donde todos los
elementos de la \texttt{list} tienen el mismo tamaño.

\subsection{\texorpdfstring{Crear un
\texttt{data.frame}}{Crear un data.frame}}\label{crear-un-data.frame}

Para crear un \texttt{data.frame} usamos la función
\texttt{data.frame()} que toma como argumentos los elementos de la tabla
que queremos crear. Los elementos son del tipo \texttt{vector} y son
todos del mismo tamaño. Podemos dar un nombre a cada \emph{columna}
(\texttt{vector}) de nuestra \emph{tabla} (\texttt{data.frame}).

\begin{Shaded}
\begin{Highlighting}[]
\CommentTok{# crear un data.frame }
\NormalTok{miDf01 <-}\StringTok{ }\KeywordTok{data.frame}\NormalTok{(}
  \DataTypeTok{numbers =} \KeywordTok{c}\NormalTok{(}\DecValTok{1}\NormalTok{, }\DecValTok{2}\NormalTok{, }\DecValTok{3}\NormalTok{, }\DecValTok{4}\NormalTok{), }
  \DataTypeTok{logicals =} \KeywordTok{c}\NormalTok{(}\OtherTok{TRUE}\NormalTok{, }\OtherTok{TRUE}\NormalTok{, }\OtherTok{FALSE}\NormalTok{, }\OtherTok{TRUE}\NormalTok{), }
  \DataTypeTok{characters =} \KeywordTok{c}\NormalTok{(}\StringTok{"a"}\NormalTok{, }\StringTok{"b"}\NormalTok{, }\StringTok{"c"}\NormalTok{, }\StringTok{"d"}\NormalTok{)}
\NormalTok{)}
\KeywordTok{print}\NormalTok{(miDf01)}
\end{Highlighting}
\end{Shaded}

\begin{verbatim}
##   numbers logicals characters
## 1       1     TRUE          a
## 2       2     TRUE          b
## 3       3    FALSE          c
## 4       4     TRUE          d
\end{verbatim}

\begin{Shaded}
\begin{Highlighting}[]
\CommentTok{# crear vectores, y el data.frame}
\NormalTok{numbers <-}\StringTok{ }\KeywordTok{c}\NormalTok{(}\DecValTok{1}\NormalTok{, }\DecValTok{2}\NormalTok{, }\DecValTok{3}\NormalTok{, }\DecValTok{4}\NormalTok{)}
\NormalTok{logicals <-}\StringTok{ }\KeywordTok{c}\NormalTok{(}\OtherTok{TRUE}\NormalTok{, }\OtherTok{TRUE}\NormalTok{, }\OtherTok{FALSE}\NormalTok{, }\OtherTok{TRUE}\NormalTok{)}
\NormalTok{characters <-}\StringTok{ }\KeywordTok{c}\NormalTok{(}\StringTok{"a"}\NormalTok{, }\StringTok{"b"}\NormalTok{, }\StringTok{"c"}\NormalTok{, }\StringTok{"d"}\NormalTok{)}
\NormalTok{miDf01 <-}\StringTok{ }\KeywordTok{data.frame}\NormalTok{(numbers, logicals, characters)}
\KeywordTok{print}\NormalTok{(miDf01)}
\end{Highlighting}
\end{Shaded}

\begin{verbatim}
##   numbers logicals characters
## 1       1     TRUE          a
## 2       2     TRUE          b
## 3       3    FALSE          c
## 4       4     TRUE          d
\end{verbatim}

\subsection{\texorpdfstring{Acceder a los elementos de un
\texttt{data.frame}}{Acceder a los elementos de un data.frame}}\label{acceder-a-los-elementos-de-un-data.frame}

El acceso a los diferentes valores de un \texttt{data.frame} se puede
hacer de la misma manera que para un contenedor de tipo \texttt{list}.

\begin{Shaded}
\begin{Highlighting}[]
\KeywordTok{print}\NormalTok{(miDf01}\OperatorTok{$}\NormalTok{numbers) }\CommentTok{# vector}
\end{Highlighting}
\end{Shaded}

\begin{verbatim}
## [1] 1 2 3 4
\end{verbatim}

\begin{Shaded}
\begin{Highlighting}[]
\KeywordTok{print}\NormalTok{(miDf01[[}\DecValTok{1}\NormalTok{]]) }\CommentTok{# vector}
\end{Highlighting}
\end{Shaded}

\begin{verbatim}
## [1] 1 2 3 4
\end{verbatim}

\begin{Shaded}
\begin{Highlighting}[]
\KeywordTok{print}\NormalTok{(miDf01[}\DecValTok{1}\NormalTok{]) }\CommentTok{# list}
\end{Highlighting}
\end{Shaded}

\begin{verbatim}
##   numbers
## 1       1
## 2       2
## 3       3
## 4       4
\end{verbatim}

\begin{Shaded}
\begin{Highlighting}[]
\KeywordTok{print}\NormalTok{(miDf01[}\StringTok{"numbers"}\NormalTok{]) }\CommentTok{# list}
\end{Highlighting}
\end{Shaded}

\begin{verbatim}
##   numbers
## 1       1
## 2       2
## 3       3
## 4       4
\end{verbatim}

\begin{Shaded}
\begin{Highlighting}[]
\KeywordTok{print}\NormalTok{(miDf01[[}\StringTok{"numbers"}\NormalTok{]]) }\CommentTok{# vector}
\end{Highlighting}
\end{Shaded}

\begin{verbatim}
## [1] 1 2 3 4
\end{verbatim}

También podemos usar otra forma que consiste en especificar la línea o
las líneas seguidas de una coma (con un espacio después de la coma), y
luego la columna o columnas entre corchetes. Si se omite la información
de línea o columna, R mostrará todas las líneas o columnas. Nuevamente
podemos usar el número correspondiente a un elemento o el nombre del
elemento que queremos seleccionar.

\begin{Shaded}
\begin{Highlighting}[]
\NormalTok{myRow <-}\StringTok{ }\DecValTok{2}
\NormalTok{myCol <-}\StringTok{ }\DecValTok{1}
\KeywordTok{print}\NormalTok{(miDf01[myRow, myCol])}
\end{Highlighting}
\end{Shaded}

\begin{verbatim}
## [1] 2
\end{verbatim}

\begin{Shaded}
\begin{Highlighting}[]
\KeywordTok{print}\NormalTok{(miDf01[myRow, ])}
\end{Highlighting}
\end{Shaded}

\begin{verbatim}
##   numbers logicals characters
## 2       2     TRUE          b
\end{verbatim}

\begin{Shaded}
\begin{Highlighting}[]
\KeywordTok{print}\NormalTok{(miDf01[, myCol])}
\end{Highlighting}
\end{Shaded}

\begin{verbatim}
## [1] 1 2 3 4
\end{verbatim}

\begin{Shaded}
\begin{Highlighting}[]
\NormalTok{myCol <-}\StringTok{ "numbers"}
\KeywordTok{print}\NormalTok{(miDf01[, myCol])}
\end{Highlighting}
\end{Shaded}

\begin{verbatim}
## [1] 1 2 3 4
\end{verbatim}

Es posible seleccionar múltiples líneas o columnas.

\begin{Shaded}
\begin{Highlighting}[]
\KeywordTok{print}\NormalTok{(miDf01[, }\KeywordTok{c}\NormalTok{(}\DecValTok{1}\NormalTok{, }\DecValTok{2}\NormalTok{)])}
\end{Highlighting}
\end{Shaded}

\begin{verbatim}
##   numbers logicals
## 1       1     TRUE
## 2       2     TRUE
## 3       3    FALSE
## 4       4     TRUE
\end{verbatim}

\begin{Shaded}
\begin{Highlighting}[]
\KeywordTok{print}\NormalTok{(miDf01[}\KeywordTok{c}\NormalTok{(}\DecValTok{2}\NormalTok{, }\DecValTok{1}\NormalTok{), ])}
\end{Highlighting}
\end{Shaded}

\begin{verbatim}
##   numbers logicals characters
## 2       2     TRUE          b
## 1       1     TRUE          a
\end{verbatim}

Como cada columna está en formato \texttt{vector}, también podemos hacer
una selección que depende del contenido con operadores de comparación y
operadores lógicos.

\begin{Shaded}
\begin{Highlighting}[]
\NormalTok{miDfSub01 <-}\StringTok{ }\NormalTok{miDf01[miDf01}\OperatorTok{$}\NormalTok{numbers }\OperatorTok{>}\StringTok{ }\DecValTok{2}\NormalTok{, ]}
\KeywordTok{print}\NormalTok{(miDfSub01)}
\end{Highlighting}
\end{Shaded}

\begin{verbatim}
##   numbers logicals characters
## 3       3    FALSE          c
## 4       4     TRUE          d
\end{verbatim}

\begin{Shaded}
\begin{Highlighting}[]
\NormalTok{miDfSub02 <-}\StringTok{ }\NormalTok{miDf01[(miDf01}\OperatorTok{$}\NormalTok{logicals }\OperatorTok{==}\StringTok{ }\OtherTok{TRUE}\NormalTok{) }\OperatorTok{&}\StringTok{ }\NormalTok{(miDf01}\OperatorTok{$}\NormalTok{numbers }\OperatorTok{<}\StringTok{ }\DecValTok{2}\NormalTok{), ]}
\KeywordTok{print}\NormalTok{(miDfSub02)}
\end{Highlighting}
\end{Shaded}

\begin{verbatim}
##   numbers logicals characters
## 1       1     TRUE          a
\end{verbatim}

\begin{Shaded}
\begin{Highlighting}[]
\NormalTok{miDfSub03 <-}\StringTok{ }\NormalTok{miDf01[(miDf01}\OperatorTok{$}\NormalTok{numbers }\OperatorTok\StringTok{ }\DecValTok{2}\NormalTok{) }\OperatorTok{==}\StringTok{ }\DecValTok{0}\NormalTok{, ]}
\KeywordTok{print}\NormalTok{(miDfSub03)}
\end{Highlighting}
\end{Shaded}

\begin{verbatim}
##   numbers logicals characters
## 2       2     TRUE          b
## 4       4     TRUE          d
\end{verbatim}

\begin{Shaded}
\begin{Highlighting}[]
\NormalTok{miDfSub04 <-}\StringTok{ }\NormalTok{miDf01[((miDf01}\OperatorTok{$}\NormalTok{numbers }\OperatorTok\StringTok{ }\DecValTok{2}\NormalTok{) }\OperatorTok{==}\StringTok{ }\DecValTok{0}\NormalTok{) }\OperatorTok{|}\StringTok{ }\NormalTok{(miDf01}\OperatorTok{$}\NormalTok{logicals }\OperatorTok{==}\StringTok{ }\OtherTok{TRUE}\NormalTok{), ]}
\KeywordTok{print}\NormalTok{(miDfSub04)}
\end{Highlighting}
\end{Shaded}

\begin{verbatim}
##   numbers logicals characters
## 1       1     TRUE          a
## 2       2     TRUE          b
## 4       4     TRUE          d
\end{verbatim}

\subsection{\texorpdfstring{Modificar un
\texttt{data.frame}}{Modificar un data.frame}}\label{modificar-un-data.frame}

Para agregar un elemento a un \texttt{data.frame}, procedemos como para
un contenedor de tipo \texttt{list}. Es necesario asegurarse de que el
nuevo elemento sea del mismo tamaño que los otros elementos de nuestro
\texttt{data.frame}. Por defecto, un nuevo elemento en
\texttt{data.frame} toma el nombre de la letra \emph{V} seguido del
número de la columna. Podemos cambiar los nombres de las columnas con la
función \texttt{colnames()}. Podemos nombrar las líneas con la función
\texttt{rownames()}.

\begin{Shaded}
\begin{Highlighting}[]
\NormalTok{newVec <-}\StringTok{ }\KeywordTok{c}\NormalTok{(}\DecValTok{4}\NormalTok{, }\DecValTok{5}\NormalTok{, }\DecValTok{6}\NormalTok{, }\DecValTok{7}\NormalTok{)}
\NormalTok{miDf01[[}\DecValTok{4}\NormalTok{]] <-}\StringTok{ }\NormalTok{newVec}
\KeywordTok{print}\NormalTok{(miDf01)}
\end{Highlighting}
\end{Shaded}

\begin{verbatim}
##   numbers logicals characters V4
## 1       1     TRUE          a  4
## 2       2     TRUE          b  5
## 3       3    FALSE          c  6
## 4       4     TRUE          d  7
\end{verbatim}

\begin{Shaded}
\begin{Highlighting}[]
\KeywordTok{print}\NormalTok{(}\KeywordTok{colnames}\NormalTok{(miDf01))}
\end{Highlighting}
\end{Shaded}

\begin{verbatim}
## [1] "numbers"    "logicals"   "characters" "V4"
\end{verbatim}

\begin{Shaded}
\begin{Highlighting}[]
\KeywordTok{colnames}\NormalTok{(miDf01)[}\DecValTok{4}\NormalTok{] <-}\StringTok{ "newVec"}
\KeywordTok{print}\NormalTok{(miDf01)}
\end{Highlighting}
\end{Shaded}

\begin{verbatim}
##   numbers logicals characters newVec
## 1       1     TRUE          a      4
## 2       2     TRUE          b      5
## 3       3    FALSE          c      6
## 4       4     TRUE          d      7
\end{verbatim}

\begin{Shaded}
\begin{Highlighting}[]
\KeywordTok{print}\NormalTok{(}\KeywordTok{rownames}\NormalTok{(miDf01))}
\end{Highlighting}
\end{Shaded}

\begin{verbatim}
## [1] "1" "2" "3" "4"
\end{verbatim}

\begin{Shaded}
\begin{Highlighting}[]
\KeywordTok{rownames}\NormalTok{(miDf01) <-}\StringTok{ }\KeywordTok{c}\NormalTok{(}\StringTok{"row1"}\NormalTok{, }\StringTok{"row2"}\NormalTok{, }\StringTok{"row3"}\NormalTok{, }\StringTok{"row4"}\NormalTok{)}
\KeywordTok{print}\NormalTok{(miDf01)}
\end{Highlighting}
\end{Shaded}

\begin{verbatim}
##      numbers logicals characters newVec
## row1       1     TRUE          a      4
## row2       2     TRUE          b      5
## row3       3    FALSE          c      6
## row4       4     TRUE          d      7
\end{verbatim}

\begin{Shaded}
\begin{Highlighting}[]
\NormalTok{newVec2 <-}\StringTok{ }\KeywordTok{c}\NormalTok{(}\DecValTok{40}\NormalTok{, }\DecValTok{50}\NormalTok{, }\DecValTok{60}\NormalTok{, }\DecValTok{70}\NormalTok{)}
\NormalTok{miDf01}\OperatorTok{$}\NormalTok{newVec2 <-}\StringTok{ }\NormalTok{newVec2}
\KeywordTok{print}\NormalTok{(miDf01)}
\end{Highlighting}
\end{Shaded}

\begin{verbatim}
##      numbers logicals characters newVec newVec2
## row1       1     TRUE          a      4      40
## row2       2     TRUE          b      5      50
## row3       3    FALSE          c      6      60
## row4       4     TRUE          d      7      70
\end{verbatim}

Como el contenedor de tipo \texttt{data.frame} es un caso especial de
\texttt{list}, la selección y modificación se realiza como un contenedor
de tipo \texttt{list}. Dado que los elementos de un \texttt{data.frame}
son del tipo \texttt{vector}, la selección y la modificación de los
elementos de un \texttt{data.frame} se hace como para un contenedor
\texttt{vector}.

\begin{Shaded}
\begin{Highlighting}[]
\NormalTok{miDf01}\OperatorTok{$}\NormalTok{newVec2 <-}\StringTok{ }\NormalTok{miDf01}\OperatorTok{$}\NormalTok{newVec2 }\OperatorTok{*}\StringTok{ }\DecValTok{2}
\KeywordTok{print}\NormalTok{(miDf01)}
\end{Highlighting}
\end{Shaded}

\begin{verbatim}
##      numbers logicals characters newVec newVec2
## row1       1     TRUE          a      4      80
## row2       2     TRUE          b      5     100
## row3       3    FALSE          c      6     120
## row4       4     TRUE          d      7     140
\end{verbatim}

\begin{Shaded}
\begin{Highlighting}[]
\NormalTok{miDf01}\OperatorTok{$}\NormalTok{newVec2 }\OperatorTok{+}\StringTok{ }\NormalTok{miDf01}\OperatorTok{$}\NormalTok{newVec}
\end{Highlighting}
\end{Shaded}

\begin{verbatim}
## [1]  84 105 126 147
\end{verbatim}

\begin{Shaded}
\begin{Highlighting}[]
\NormalTok{miDf01}\OperatorTok{$}\NormalTok{newVec2[}\DecValTok{2}\NormalTok{] <-}\StringTok{ }\DecValTok{0}
\KeywordTok{print}\NormalTok{(miDf01)}
\end{Highlighting}
\end{Shaded}

\begin{verbatim}
##      numbers logicals characters newVec newVec2
## row1       1     TRUE          a      4      80
## row2       2     TRUE          b      5       0
## row3       3    FALSE          c      6     120
## row4       4     TRUE          d      7     140
\end{verbatim}

Un \texttt{vector} se puede transformar en \texttt{data.frame} con la
función \texttt{as.data.frame()}.

\begin{Shaded}
\begin{Highlighting}[]
\KeywordTok{print}\NormalTok{(newVec2)}
\end{Highlighting}
\end{Shaded}

\begin{verbatim}
## [1] 40 50 60 70
\end{verbatim}

\begin{Shaded}
\begin{Highlighting}[]
\KeywordTok{print}\NormalTok{(}\KeywordTok{as.data.frame}\NormalTok{(newVec2))}
\end{Highlighting}
\end{Shaded}

\begin{verbatim}
##   newVec2
## 1      40
## 2      50
## 3      60
## 4      70
\end{verbatim}

\begin{Shaded}
\begin{Highlighting}[]
\KeywordTok{is.data.frame}\NormalTok{(newVec2)}
\end{Highlighting}
\end{Shaded}

\begin{verbatim}
## [1] FALSE
\end{verbatim}

\begin{Shaded}
\begin{Highlighting}[]
\KeywordTok{is.data.frame}\NormalTok{(}\KeywordTok{as.data.frame}\NormalTok{(newVec2))}
\end{Highlighting}
\end{Shaded}

\begin{verbatim}
## [1] TRUE
\end{verbatim}

\hypertarget{l014matrix}{\section{\texorpdfstring{El contenedor
\texttt{matrix}}{El contenedor matrix}}\label{l014matrix}}

El contenedor \texttt{matrix} se puede ver como un \texttt{vector} de
dos dimensiones: líneas y columnas. Corresponde a una matriz en
matemáticas, y puede contener solo un tipo de datos (\texttt{logical},
\texttt{numeric}, \texttt{character}, \ldots{}).

\subsection{\texorpdfstring{Crear una
\texttt{matrix}}{Crear una matrix}}\label{crear-una-matrix}

Para crear una \texttt{matrix} primero creamos un \texttt{vector}, luego
especificamos el número deseado de líneas y columnas en la función
\texttt{matrix()}.

\begin{Shaded}
\begin{Highlighting}[]
\NormalTok{vecForMatrix <-}\StringTok{ }\KeywordTok{c}\NormalTok{(}\DecValTok{1}\NormalTok{, }\DecValTok{2}\NormalTok{, }\DecValTok{3}\NormalTok{, }\DecValTok{4}\NormalTok{, }\DecValTok{5}\NormalTok{, }\DecValTok{6}\NormalTok{, }\DecValTok{7}\NormalTok{, }\DecValTok{8}\NormalTok{, }\DecValTok{9}\NormalTok{, }\DecValTok{10}\NormalTok{, }\DecValTok{11}\NormalTok{, }\DecValTok{12}\NormalTok{)}
\NormalTok{miMat <-}\StringTok{ }\KeywordTok{matrix}\NormalTok{(vecForMatrix, }\DataTypeTok{nrow =} \DecValTok{3}\NormalTok{, }\DataTypeTok{ncol =} \DecValTok{4}\NormalTok{)}
\KeywordTok{print}\NormalTok{(miMat)}
\end{Highlighting}
\end{Shaded}

\begin{verbatim}
##      [,1] [,2] [,3] [,4]
## [1,]    1    4    7   10
## [2,]    2    5    8   11
## [3,]    3    6    9   12
\end{verbatim}

No tenemos que especificar el número de líneas \texttt{nrow} y el número
de columnas \texttt{ncol}. Si usamos uno u otro de estos argumentos, R
calculará automáticamente el número correspondiente.

\begin{Shaded}
\begin{Highlighting}[]
\NormalTok{miMat <-}\StringTok{ }\KeywordTok{matrix}\NormalTok{(vecForMatrix, }\DataTypeTok{nrow =} \DecValTok{3}\NormalTok{)}
\KeywordTok{print}\NormalTok{(miMat)}
\end{Highlighting}
\end{Shaded}

\begin{verbatim}
##      [,1] [,2] [,3] [,4]
## [1,]    1    4    7   10
## [2,]    2    5    8   11
## [3,]    3    6    9   12
\end{verbatim}

\begin{Shaded}
\begin{Highlighting}[]
\NormalTok{miMat <-}\StringTok{ }\KeywordTok{matrix}\NormalTok{(vecForMatrix, }\DataTypeTok{ncol =} \DecValTok{4}\NormalTok{)}
\KeywordTok{print}\NormalTok{(miMat)}
\end{Highlighting}
\end{Shaded}

\begin{verbatim}
##      [,1] [,2] [,3] [,4]
## [1,]    1    4    7   10
## [2,]    2    5    8   11
## [3,]    3    6    9   12
\end{verbatim}

Observamos que los diferentes elementos del \texttt{vector} inicial
aparecen por columna. Si queremos llenar la \texttt{matrix} empezando
por línea, entonces tenemos que dar como valor \texttt{TRUE} al
argumento \texttt{byrow}.

\begin{Shaded}
\begin{Highlighting}[]
\NormalTok{miMat <-}\StringTok{ }\KeywordTok{matrix}\NormalTok{(vecForMatrix, }\DataTypeTok{nrow =} \DecValTok{3}\NormalTok{, }\DataTypeTok{byrow =} \OtherTok{TRUE}\NormalTok{)}
\KeywordTok{print}\NormalTok{(miMat)}
\end{Highlighting}
\end{Shaded}

\begin{verbatim}
##      [,1] [,2] [,3] [,4]
## [1,]    1    2    3    4
## [2,]    5    6    7    8
## [3,]    9   10   11   12
\end{verbatim}

También podemos dar un nombre a las líneas y columnas de nuestra
\texttt{matrix} cuando se crea con el argumento \texttt{dimnames} que
toma como valor una \texttt{list} de dos elementos: el nombre de las
líneas y luego el nombre de las columnas. También podemos cambiar el
nombre de las líneas y columnas a posteriori con las funciones
\texttt{rownames()} y \texttt{colnames()}.

\begin{Shaded}
\begin{Highlighting}[]
\NormalTok{miMat <-}\StringTok{ }\KeywordTok{matrix}\NormalTok{(}
\NormalTok{  vecForMatrix, }
  \DataTypeTok{nrow =} \DecValTok{3}\NormalTok{, }
  \DataTypeTok{byrow =} \OtherTok{TRUE}\NormalTok{, }
  \DataTypeTok{dimnames =} \KeywordTok{list}\NormalTok{(}\KeywordTok{c}\NormalTok{(}\StringTok{"r1"}\NormalTok{, }\StringTok{"r2"}\NormalTok{, }\StringTok{"r3"}\NormalTok{), }\KeywordTok{c}\NormalTok{(}\StringTok{"c1"}\NormalTok{, }\StringTok{"c2"}\NormalTok{, }\StringTok{"c3"}\NormalTok{, }\StringTok{"c4"}\NormalTok{))}
\NormalTok{)}
\KeywordTok{print}\NormalTok{(miMat)}
\end{Highlighting}
\end{Shaded}

\begin{verbatim}
##    c1 c2 c3 c4
## r1  1  2  3  4
## r2  5  6  7  8
## r3  9 10 11 12
\end{verbatim}

\begin{Shaded}
\begin{Highlighting}[]
\KeywordTok{colnames}\NormalTok{(miMat) <-}\StringTok{ }\KeywordTok{c}\NormalTok{(}\StringTok{"col1"}\NormalTok{, }\StringTok{"col2"}\NormalTok{, }\StringTok{"col3"}\NormalTok{, }\StringTok{"col4"}\NormalTok{)}
\KeywordTok{rownames}\NormalTok{(miMat) <-}\StringTok{ }\KeywordTok{c}\NormalTok{(}\StringTok{"row1"}\NormalTok{, }\StringTok{"row2"}\NormalTok{, }\StringTok{"row3"}\NormalTok{)}
\KeywordTok{print}\NormalTok{(miMat)}
\end{Highlighting}
\end{Shaded}

\begin{verbatim}
##      col1 col2 col3 col4
## row1    1    2    3    4
## row2    5    6    7    8
## row3    9   10   11   12
\end{verbatim}

Es posible crear una \texttt{matrix} desde un \texttt{data.frame} con la
función \texttt{as.matrix()}. Tenemos que verificar que nuestra
\texttt{data.frame} contenga solo elementos del mismo tipo (por ejemplo,
elementos de tipo \texttt{numeric}).

\begin{Shaded}
\begin{Highlighting}[]
\NormalTok{vecForMat01 <-}\StringTok{ }\KeywordTok{c}\NormalTok{(}\DecValTok{1}\NormalTok{, }\DecValTok{2}\NormalTok{, }\DecValTok{3}\NormalTok{, }\DecValTok{4}\NormalTok{, }\DecValTok{5}\NormalTok{, }\DecValTok{6}\NormalTok{, }\DecValTok{7}\NormalTok{, }\DecValTok{8}\NormalTok{, }\DecValTok{9}\NormalTok{, }\DecValTok{10}\NormalTok{, }\DecValTok{11}\NormalTok{, }\DecValTok{12}\NormalTok{)}
\NormalTok{vecForMat02 <-}\StringTok{ }\NormalTok{vecForMat01 }\OperatorTok{*}\StringTok{ }\DecValTok{10}
\NormalTok{vecForMat03 <-}\StringTok{ }\NormalTok{vecForMat01 }\OperatorTok{/}\StringTok{ }\DecValTok{10}
\NormalTok{dfForMat <-}\StringTok{ }\KeywordTok{data.frame}\NormalTok{(vecForMat01, vecForMat02, vecForMat03)}
\KeywordTok{print}\NormalTok{(dfForMat)}
\end{Highlighting}
\end{Shaded}

\begin{verbatim}
##    vecForMat01 vecForMat02 vecForMat03
## 1            1          10         0.1
## 2            2          20         0.2
## 3            3          30         0.3
## 4            4          40         0.4
## 5            5          50         0.5
## 6            6          60         0.6
## 7            7          70         0.7
## 8            8          80         0.8
## 9            9          90         0.9
## 10          10         100         1.0
## 11          11         110         1.1
## 12          12         120         1.2
\end{verbatim}

\begin{Shaded}
\begin{Highlighting}[]
\KeywordTok{is.matrix}\NormalTok{(dfForMat)}
\end{Highlighting}
\end{Shaded}

\begin{verbatim}
## [1] FALSE
\end{verbatim}

\begin{Shaded}
\begin{Highlighting}[]
\KeywordTok{as.matrix}\NormalTok{(dfForMat)}
\end{Highlighting}
\end{Shaded}

\begin{verbatim}
##       vecForMat01 vecForMat02 vecForMat03
##  [1,]           1          10         0.1
##  [2,]           2          20         0.2
##  [3,]           3          30         0.3
##  [4,]           4          40         0.4
##  [5,]           5          50         0.5
##  [6,]           6          60         0.6
##  [7,]           7          70         0.7
##  [8,]           8          80         0.8
##  [9,]           9          90         0.9
## [10,]          10         100         1.0
## [11,]          11         110         1.1
## [12,]          12         120         1.2
\end{verbatim}

\begin{Shaded}
\begin{Highlighting}[]
\KeywordTok{is.matrix}\NormalTok{(}\KeywordTok{as.matrix}\NormalTok{(dfForMat))}
\end{Highlighting}
\end{Shaded}

\begin{verbatim}
## [1] TRUE
\end{verbatim}

También podemos crear una \texttt{matrix} desde un \texttt{vector} con
la función \texttt{as.matrix()} (matriz de una sola columna).

\begin{Shaded}
\begin{Highlighting}[]
\KeywordTok{as.matrix}\NormalTok{(vecForMat01)}
\end{Highlighting}
\end{Shaded}

\begin{verbatim}
##       [,1]
##  [1,]    1
##  [2,]    2
##  [3,]    3
##  [4,]    4
##  [5,]    5
##  [6,]    6
##  [7,]    7
##  [8,]    8
##  [9,]    9
## [10,]   10
## [11,]   11
## [12,]   12
\end{verbatim}

\subsection{\texorpdfstring{Manipular y hacer operaciones en una
\texttt{matrix}}{Manipular y hacer operaciones en una matrix}}\label{manipular-y-hacer-operaciones-en-una-matrix}

Todas las operaciones término a término son posibles con una
\texttt{matrix}.

\begin{Shaded}
\begin{Highlighting}[]
\CommentTok{# operaciones término a término}
\NormalTok{miMat01 <-}\StringTok{ }\KeywordTok{matrix}\NormalTok{(vecForMat01, }\DataTypeTok{ncol =} \DecValTok{3}\NormalTok{)}
\NormalTok{miVecOp <-}\StringTok{ }\KeywordTok{c}\NormalTok{(}\DecValTok{1}\NormalTok{, }\DecValTok{10}\NormalTok{, }\DecValTok{100}\NormalTok{, }\DecValTok{1000}\NormalTok{)}
\NormalTok{miMat01 }\OperatorTok{*}\StringTok{ }\NormalTok{miVecOp}
\end{Highlighting}
\end{Shaded}

\begin{verbatim}
##      [,1] [,2]  [,3]
## [1,]    1    5     9
## [2,]   20   60   100
## [3,]  300  700  1100
## [4,] 4000 8000 12000
\end{verbatim}

\begin{Shaded}
\begin{Highlighting}[]
\NormalTok{miMat01 }\OperatorTok{+}\StringTok{ }\NormalTok{miVecOp}
\end{Highlighting}
\end{Shaded}

\begin{verbatim}
##      [,1] [,2] [,3]
## [1,]    2    6   10
## [2,]   12   16   20
## [3,]  103  107  111
## [4,] 1004 1008 1012
\end{verbatim}

\begin{Shaded}
\begin{Highlighting}[]
\NormalTok{miMat01 }\OperatorTok{/}\StringTok{ }\NormalTok{miMat01}
\end{Highlighting}
\end{Shaded}

\begin{verbatim}
##      [,1] [,2] [,3]
## [1,]    1    1    1
## [2,]    1    1    1
## [3,]    1    1    1
## [4,]    1    1    1
\end{verbatim}

\begin{Shaded}
\begin{Highlighting}[]
\NormalTok{miMat01 }\OperatorTok{-}\StringTok{ }\DecValTok{10}
\end{Highlighting}
\end{Shaded}

\begin{verbatim}
##      [,1] [,2] [,3]
## [1,]   -9   -5   -1
## [2,]   -8   -4    0
## [3,]   -7   -3    1
## [4,]   -6   -2    2
\end{verbatim}

Para realizar operaciones algebraicas, podemos usar la función
\texttt{\%*\%}.

\begin{Shaded}
\begin{Highlighting}[]
\CommentTok{# operaciones algebraicas}
\NormalTok{miVecConf <-}\StringTok{ }\KeywordTok{c}\NormalTok{(}\DecValTok{1}\NormalTok{, }\DecValTok{10}\NormalTok{, }\DecValTok{100}\NormalTok{)}
\NormalTok{miMat01 }\OperatorTok\StringTok{ }\NormalTok{miVecConf}
\end{Highlighting}
\end{Shaded}

\begin{verbatim}
##      [,1]
## [1,]  951
## [2,] 1062
## [3,] 1173
## [4,] 1284
\end{verbatim}

\begin{Shaded}
\begin{Highlighting}[]
\NormalTok{miMat02 <-}\StringTok{ }\KeywordTok{matrix}\NormalTok{(}\KeywordTok{c}\NormalTok{(}\DecValTok{1}\NormalTok{, }\DecValTok{2}\NormalTok{, }\DecValTok{3}\NormalTok{, }\DecValTok{4}\NormalTok{, }\DecValTok{5}\NormalTok{, }\DecValTok{6}\NormalTok{, }\DecValTok{7}\NormalTok{, }\DecValTok{8}\NormalTok{, }\DecValTok{9}\NormalTok{), }\DataTypeTok{ncol =} \DecValTok{3}\NormalTok{)}
\KeywordTok{print}\NormalTok{(miMat02)}
\end{Highlighting}
\end{Shaded}

\begin{verbatim}
##      [,1] [,2] [,3]
## [1,]    1    4    7
## [2,]    2    5    8
## [3,]    3    6    9
\end{verbatim}

\begin{Shaded}
\begin{Highlighting}[]
\NormalTok{miMat02 }\OperatorTok\StringTok{ }\NormalTok{miMat02}
\end{Highlighting}
\end{Shaded}

\begin{verbatim}
##      [,1] [,2] [,3]
## [1,]   30   66  102
## [2,]   36   81  126
## [3,]   42   96  150
\end{verbatim}

La diagonal de una \texttt{matrix} se puede obtener con la función
\texttt{diag()} y el determinante de una \texttt{matrix} con la función
\texttt{det()}.

\begin{Shaded}
\begin{Highlighting}[]
\KeywordTok{print}\NormalTok{(miMat02)}
\end{Highlighting}
\end{Shaded}

\begin{verbatim}
##      [,1] [,2] [,3]
## [1,]    1    4    7
## [2,]    2    5    8
## [3,]    3    6    9
\end{verbatim}

\begin{Shaded}
\begin{Highlighting}[]
\KeywordTok{diag}\NormalTok{(miMat02)}
\end{Highlighting}
\end{Shaded}

\begin{verbatim}
## [1] 1 5 9
\end{verbatim}

\begin{Shaded}
\begin{Highlighting}[]
\KeywordTok{det}\NormalTok{(miMat02)}
\end{Highlighting}
\end{Shaded}

\begin{verbatim}
## [1] 0
\end{verbatim}

Suele ser útil poder hacer una transposición de \texttt{matrix}
(columnas en líneas o líneas en columnas). Para eso, están las funciones
\texttt{aperm()} o \texttt{t()}. la función \texttt{t()} es más genérica
y también funciona con \texttt{data.frame}.

\begin{Shaded}
\begin{Highlighting}[]
\KeywordTok{aperm}\NormalTok{(miMat01)}
\end{Highlighting}
\end{Shaded}

\begin{verbatim}
##      [,1] [,2] [,3] [,4]
## [1,]    1    2    3    4
## [2,]    5    6    7    8
## [3,]    9   10   11   12
\end{verbatim}

\begin{Shaded}
\begin{Highlighting}[]
\KeywordTok{t}\NormalTok{(miMat01)}
\end{Highlighting}
\end{Shaded}

\begin{verbatim}
##      [,1] [,2] [,3] [,4]
## [1,]    1    2    3    4
## [2,]    5    6    7    8
## [3,]    9   10   11   12
\end{verbatim}

\subsection{\texorpdfstring{Acceder a los elementos de una
\texttt{matrix}}{Acceder a los elementos de una matrix}}\label{acceder-a-los-elementos-de-una-matrix}

Tal como hemos hecho con los \texttt{data.frame}, podemos acceder a los
elementos de una \texttt{matrix} especificando un número de línea y un
número de columna entre corchetes simples \texttt{{[}\ {]}}, y separados
por una coma. Si \texttt{i} es el número de línea y \texttt{j} es el
número de columna, entonces \texttt{miMat01{[}i,\ j{]}} devuelve el
elemento en la línea \texttt{i} y en la columna\texttt{j}.
\texttt{miMat01{[}i,{]}} devuelve todos los elementos de la línea
\texttt{i}, y \texttt{miMat01{[},\ j{]}} todos los elementos de la
columna \texttt{j}. Múltiples selecciones son posibles. También podemos
acceder a un elemento de acuerdo con su posición en la \texttt{matrix}
entre corchetes simples \texttt{{[}\ {]}} contando por columna y luego
por línea. En nuestro ejemplo, el valor del décimo elemento es 10.

\begin{Shaded}
\begin{Highlighting}[]
\NormalTok{i <-}\StringTok{ }\DecValTok{2}
\NormalTok{j <-}\StringTok{ }\DecValTok{1}
\KeywordTok{print}\NormalTok{(miMat01[i, j])}
\end{Highlighting}
\end{Shaded}

\begin{verbatim}
## [1] 2
\end{verbatim}

\begin{Shaded}
\begin{Highlighting}[]
\KeywordTok{print}\NormalTok{(miMat01[i, ])}
\end{Highlighting}
\end{Shaded}

\begin{verbatim}
## [1]  2  6 10
\end{verbatim}

\begin{Shaded}
\begin{Highlighting}[]
\KeywordTok{print}\NormalTok{(miMat01[, j])}
\end{Highlighting}
\end{Shaded}

\begin{verbatim}
## [1] 1 2 3 4
\end{verbatim}

\begin{Shaded}
\begin{Highlighting}[]
\KeywordTok{print}\NormalTok{(miMat01[}\KeywordTok{c}\NormalTok{(}\DecValTok{1}\NormalTok{, }\DecValTok{2}\NormalTok{), }\KeywordTok{c}\NormalTok{(}\DecValTok{2}\NormalTok{, }\DecValTok{3}\NormalTok{)])}
\end{Highlighting}
\end{Shaded}

\begin{verbatim}
##      [,1] [,2]
## [1,]    5    9
## [2,]    6   10
\end{verbatim}

\begin{Shaded}
\begin{Highlighting}[]
\KeywordTok{print}\NormalTok{(miMat01[}\DecValTok{10}\NormalTok{])}
\end{Highlighting}
\end{Shaded}

\begin{verbatim}
## [1] 10
\end{verbatim}

\section{\texorpdfstring{El contenedor
\texttt{array}}{El contenedor array}}\label{l014array}

El contenedor \texttt{array} es una generalización del contenedor de
tipo \texttt{matrix}. Donde el tipo \texttt{matrix} tiene dos
dimensiones (líneas y columnas), el tipo \texttt{array} tiene un número
indefinido de dimensiones. Podemos saber el número de dimensiones de un
\texttt{array} (y por lo tanto una \texttt{matrix}) con la función
\texttt{dim()}.

\begin{Shaded}
\begin{Highlighting}[]
\KeywordTok{dim}\NormalTok{(miMat01)}
\end{Highlighting}
\end{Shaded}

\begin{verbatim}
## [1] 4 3
\end{verbatim}

\subsection{\texorpdfstring{Crear un
\texttt{array}}{Crear un array}}\label{crear-un-array}

La creación de una \texttt{array} es similar a la de una \texttt{matrix}
con una dimensión extra.

\begin{Shaded}
\begin{Highlighting}[]
\NormalTok{miVecArr <-}\StringTok{ }\KeywordTok{c}\NormalTok{(}\DecValTok{1}\NormalTok{, }\DecValTok{2}\NormalTok{, }\DecValTok{3}\NormalTok{, }\DecValTok{4}\NormalTok{, }\DecValTok{5}\NormalTok{, }\DecValTok{6}\NormalTok{, }\DecValTok{7}\NormalTok{, }\DecValTok{8}\NormalTok{, }\DecValTok{9}\NormalTok{)}
\NormalTok{miArray <-}\StringTok{ }\KeywordTok{array}\NormalTok{(miVecArr, }\DataTypeTok{dim =} \KeywordTok{c}\NormalTok{(}\DecValTok{3}\NormalTok{, }\DecValTok{3}\NormalTok{, }\DecValTok{2}\NormalTok{))}
\KeywordTok{print}\NormalTok{(miArray)}
\end{Highlighting}
\end{Shaded}

\begin{verbatim}
## , , 1
## 
##      [,1] [,2] [,3]
## [1,]    1    4    7
## [2,]    2    5    8
## [3,]    3    6    9
## 
## , , 2
## 
##      [,1] [,2] [,3]
## [1,]    1    4    7
## [2,]    2    5    8
## [3,]    3    6    9
\end{verbatim}

\begin{Shaded}
\begin{Highlighting}[]
\KeywordTok{dim}\NormalTok{(miArray)}
\end{Highlighting}
\end{Shaded}

\begin{verbatim}
## [1] 3 3 2
\end{verbatim}

\begin{Shaded}
\begin{Highlighting}[]
\KeywordTok{is.array}\NormalTok{(miArray)}
\end{Highlighting}
\end{Shaded}

\begin{verbatim}
## [1] TRUE
\end{verbatim}

\begin{Shaded}
\begin{Highlighting}[]
\NormalTok{miVecArr02 <-}\StringTok{ }\DecValTok{10} \OperatorTok{*}\StringTok{ }\NormalTok{miVecArr}
\NormalTok{miArray02 <-}\StringTok{ }\KeywordTok{array}\NormalTok{(}\KeywordTok{c}\NormalTok{(miVecArr, miVecArr02), }\DataTypeTok{dim =} \KeywordTok{c}\NormalTok{(}\DecValTok{3}\NormalTok{, }\DecValTok{3}\NormalTok{, }\DecValTok{2}\NormalTok{))}
\KeywordTok{print}\NormalTok{(miArray02)}
\end{Highlighting}
\end{Shaded}

\begin{verbatim}
## , , 1
## 
##      [,1] [,2] [,3]
## [1,]    1    4    7
## [2,]    2    5    8
## [3,]    3    6    9
## 
## , , 2
## 
##      [,1] [,2] [,3]
## [1,]   10   40   70
## [2,]   20   50   80
## [3,]   30   60   90
\end{verbatim}

\begin{Shaded}
\begin{Highlighting}[]
\KeywordTok{dim}\NormalTok{(miArray02)}
\end{Highlighting}
\end{Shaded}

\begin{verbatim}
## [1] 3 3 2
\end{verbatim}

\begin{Shaded}
\begin{Highlighting}[]
\KeywordTok{is.array}\NormalTok{(miArray02)}
\end{Highlighting}
\end{Shaded}

\begin{verbatim}
## [1] TRUE
\end{verbatim}

Podemos dar nombres a líneas y columnas, pero también a elementos.

\begin{Shaded}
\begin{Highlighting}[]
\NormalTok{miArray02 <-}\StringTok{ }\KeywordTok{array}\NormalTok{(}
  \KeywordTok{c}\NormalTok{(miVecArr, miVecArr02), }
  \DataTypeTok{dim =} \KeywordTok{c}\NormalTok{(}\DecValTok{3}\NormalTok{, }\DecValTok{3}\NormalTok{, }\DecValTok{2}\NormalTok{), }
  \DataTypeTok{dimnames =} \KeywordTok{list}\NormalTok{(}
    \KeywordTok{c}\NormalTok{(}\StringTok{"r1"}\NormalTok{, }\StringTok{"r2"}\NormalTok{, }\StringTok{"r3"}\NormalTok{), }
    \KeywordTok{c}\NormalTok{(}\StringTok{"c1"}\NormalTok{, }\StringTok{"c2"}\NormalTok{, }\StringTok{"c3"}\NormalTok{), }
    \KeywordTok{c}\NormalTok{(}\StringTok{"matrix1"}\NormalTok{, }\StringTok{"matrix2"}\NormalTok{)}
\NormalTok{  )}
\NormalTok{)}
\KeywordTok{print}\NormalTok{(miArray02)}
\end{Highlighting}
\end{Shaded}

\begin{verbatim}
## , , matrix1
## 
##    c1 c2 c3
## r1  1  4  7
## r2  2  5  8
## r3  3  6  9
## 
## , , matrix2
## 
##    c1 c2 c3
## r1 10 40 70
## r2 20 50 80
## r3 30 60 90
\end{verbatim}

\subsection{\texorpdfstring{Manipular un
\texttt{array}}{Manipular un array}}\label{manipular-un-array}

La manipulación de un \texttt{array} se hace de la misma manera que para
una \texttt{matrix}. Para acceder a los diferentes elementos de un
\texttt{array}, simplemente hay que especificar la línea \texttt{i}, la
columna \texttt{j}, y la \texttt{matrix} \texttt{k}.

\begin{Shaded}
\begin{Highlighting}[]
\NormalTok{i <-}\StringTok{ }\DecValTok{2}
\NormalTok{j <-}\StringTok{ }\DecValTok{1}
\NormalTok{k <-}\StringTok{ }\DecValTok{1}
\KeywordTok{print}\NormalTok{(miArray02[i, j, k])}
\end{Highlighting}
\end{Shaded}

\begin{verbatim}
## [1] 2
\end{verbatim}

\begin{Shaded}
\begin{Highlighting}[]
\KeywordTok{print}\NormalTok{(miArray02[, j, k])}
\end{Highlighting}
\end{Shaded}

\begin{verbatim}
## r1 r2 r3 
##  1  2  3
\end{verbatim}

\begin{Shaded}
\begin{Highlighting}[]
\KeywordTok{print}\NormalTok{(miArray02[i, , k])}
\end{Highlighting}
\end{Shaded}

\begin{verbatim}
## c1 c2 c3 
##  2  5  8
\end{verbatim}

\begin{Shaded}
\begin{Highlighting}[]
\KeywordTok{print}\NormalTok{(miArray02[i, j, ])}
\end{Highlighting}
\end{Shaded}

\begin{verbatim}
## matrix1 matrix2 
##       2      20
\end{verbatim}

\section{Conclusión}\label{conclusion-3}

Felicitaciones! Ahora conocemos los principales tipos de objetos que
usaremos con R. Un objeto se caracteriza por sus atributos:

\begin{itemize}
\tightlist
\item
  el tipo de contenedor (\texttt{vector}, \texttt{data.frame},
  \texttt{matrix}, \texttt{array})
\item
  el tipo de contenido de cada elemento (\texttt{numeric},
  \texttt{logical}, \texttt{character}, \ldots{})
\item
  el valor de cada uno de los elementos (5, ``qwerty'', TRUE, \ldots{})
\end{itemize}

Todos estos objetos se almacenan temporalmente en el entorno global de R
(en la memoria RAM de nuestra computadora). El siguiente capítulo
tratará las funciones y resaltará uno de los aspectos que hace que R sea
tan poderoso para analizar y administrar nuestros datos.

\chapter{Las funciones}\label{fonctions}

\section{¿Qué es una función?}\label{que-es-una-funcion}

Con este capítulo vamos a echar un primer vistazo al poder de R a través
de las funciones. Una función es un conjunto de líneas de código para
realizar una tarea en particular. Hemos visto muchas funciones en
capítulos anteriores, unas simples como la función \texttt{+} para
añadir números, o otras más complejas como \texttt{c()} o
\texttt{data.frame()} que permiten crear un \texttt{vector} o
\texttt{data.frame}. En cualquier caso, se puede reconocer una función
gracias a los paréntesis que la siguen en los cuales vamos a ingresar
\textbf{argumentos}. Los argumentos corresponden a la información que
queremos transmitir a nuestra función para que realice la tarea que
queremos lograr.

\begin{rmdnote}
Para funciones simples como \texttt{+}, los paréntesis han sido
eliminados para que el código sea más fácil de leer, pero es una función
que puede usarse con paréntesis si usamos el signo \texttt{+} entre
comillas. Los argumentos son los números que queremos agregar.
\end{rmdnote}

\begin{Shaded}
\begin{Highlighting}[]
\DecValTok{5} \OperatorTok{+}\StringTok{ }\DecValTok{2}
\end{Highlighting}
\end{Shaded}

\begin{verbatim}
## [1] 7
\end{verbatim}

\begin{Shaded}
\begin{Highlighting}[]
\StringTok{'+'}\NormalTok{(}\DecValTok{5}\NormalTok{, }\DecValTok{2}\NormalTok{)}
\end{Highlighting}
\end{Shaded}

\begin{verbatim}
## [1] 7
\end{verbatim}

En este capítulo nos enfocaremos en las funciones más comunes. No se
trata de aprender todo de memoria, sino de saber que existen estas
funciones y de poder consultar más adelante este capítulo como
referencia. ¡Con tiempo y práctica eventualmente los sabremos de
memoria! Hay más de 1000 funciones en la versión básica de R, y más de
10000 paquetes adicionales que se pueden instalar, cada uno con docenas
de funciones. Antes de comenzar a escribir una nueva función, siempre
debemos verificar que ya no exista.

\section{Las funciones más comunes}\label{las-funciones-mas-comunes}

Para trabajar con las funciones, vamos a usar los datos \texttt{iris}
que están incluidos con la versión básica de R y que corresponden a la
longitud y el ancho de los sépalos y pétalos de diferentes especies de
iris. Los datos \texttt{iris} estan en una \texttt{data.frame} de 5
columnas y 150 líneas. Para obtener más información sobre los datos
\texttt{iris}, podemos consultar la documentación R con la función
\texttt{help(iris)}. El acceso a la documentación es el tema de la
siguiente sección.

\subsection{El acceso a la
documentación}\label{el-acceso-a-la-documentacion}

\subsubsection{\texorpdfstring{\texttt{help()}}{help()}}\label{l015help}

La función esencial de R es acceder a la documentación (en ingles).
\textbf{Todas las funciones R tienen documentación}. Podemos acceder a
la documentación con la función \texttt{help()} o usando el atajo
\texttt{?}.

\begin{Shaded}
\begin{Highlighting}[]
\KeywordTok{help}\NormalTok{(matrix) }\CommentTok{# equivalente a ?matrix}
\end{Highlighting}
\end{Shaded}

La documentación siempre está estructurada de la misma manera. Primero
tenemos el nombre de la función buscada \texttt{matrix}, seguida entre
llaves por el nombre del paquete R cuya función depende. Veremos cómo
instalar paquetes adicionales más adelante. Por ahora tenemos los que
vienen con la versión básica de R. Aquí podemos ver que la función
\texttt{matrix()} depende del paquete \texttt{base}.

Podemos ver la etiqueta de la función (\texttt{Matrices}), seguida de
los parafos \texttt{Description}, \texttt{Usage}, y \texttt{Arguments}.
Algunas veces se agregan los párrafos \texttt{Details}, \texttt{Note},
\texttt{References} y \texttt{See\ also}. El último párrafo es
\texttt{Ejemplos}. La última línea de la documentación permite volver al
índice del paquete del que depende la función consultada.

Al copiar \texttt{help(matrix)} en nuestra consola R, podemos ver que el
párrafo \texttt{Description} indica lo que hace la función. En el caso
de \texttt{help(matrix)}, hay tres funciones: \texttt{matrix()},
\texttt{as.matrix()} y \texttt{is.matrix()}.

\begin{Shaded}
\begin{Highlighting}[]
\CommentTok{# Description}
\CommentTok{# matrix creates a matrix from the given set of values.}
\CommentTok{# as.matrix attempts to turn its argument into a matrix.}
\CommentTok{# is.matrix tests if its argument is a (strict) matrix.}
\end{Highlighting}
\end{Shaded}

El párrafo \texttt{Usage} explica cómo usar la función y cuáles son los
valores predeterminados para cada parámetro.

\begin{Shaded}
\begin{Highlighting}[]
\CommentTok{# Usage}
\CommentTok{# matrix(data = NA, nrow = 1, ncol = 1, byrow = FALSE,}
\CommentTok{#        dimnames = NULL)}
\end{Highlighting}
\end{Shaded}

La función \texttt{matrix()} puede tomar 5 argumentos: \texttt{data},
\texttt{nrow}, \texttt{ncol}, \texttt{byrow}, y \texttt{dimnames}.
Podemos ver que por defecto una \texttt{matrix} consistirá de una sola
línea y una sola columna, y que la información se completará por
columna.

El párrafo \texttt{Arguments} detalla los valores y el tipo de
contenedor de cada argumento de nuestra función. Por ejemplo, podemos
ver que el argumento \texttt{dimnames} debe ser del tipo \texttt{list}.
Es por eso que hemos usado este formato en la sección
\protect\hyperlink{l014matrix}{\texttt{matrix}}.

\begin{Shaded}
\begin{Highlighting}[]
\CommentTok{# Arguments}
\CommentTok{# data      an optional data vector (including a list or expression vector). }
\CommentTok{#           Non-atomic classed R objects are coerced by as.vector and all }
\CommentTok{#           attributes discarded.}
\CommentTok{# nrow      the desired number of rows.}
\CommentTok{# ncol      the desired number of columns.}
\CommentTok{# byrow     logical. If FALSE (the default) the matrix is filled by columns, }
\CommentTok{#           otherwise the matrix is filled by rows.}
\CommentTok{# dimnames  A dimnames attribute for the matrix: NULL or a list of length 2 }
\CommentTok{#           giving the row and column names respectively. An empty list is }
\CommentTok{#           treated as NULL, and a list of length one as row names. The }
\CommentTok{#           list can be named, and the list names will be used as names for }
\CommentTok{#           the dimensions.}
\end{Highlighting}
\end{Shaded}

El párrafo \texttt{Details} proporciona elementos adicionales en la
función. El párrafo \texttt{Examples} proporciona ejemplos reproducibles
en la consola.

\begin{Shaded}
\begin{Highlighting}[]
\NormalTok{## Example of setting row and column names}
\NormalTok{mdat <-}\StringTok{ }\KeywordTok{matrix}\NormalTok{(}\KeywordTok{c}\NormalTok{(}\DecValTok{1}\NormalTok{,}\DecValTok{2}\NormalTok{,}\DecValTok{3}\NormalTok{, }\DecValTok{11}\NormalTok{,}\DecValTok{12}\NormalTok{,}\DecValTok{13}\NormalTok{), }\DataTypeTok{nrow =} \DecValTok{2}\NormalTok{, }\DataTypeTok{ncol =} \DecValTok{3}\NormalTok{, }\DataTypeTok{byrow =} \OtherTok{TRUE}\NormalTok{,}
               \DataTypeTok{dimnames =} \KeywordTok{list}\NormalTok{(}\KeywordTok{c}\NormalTok{(}\StringTok{"row1"}\NormalTok{, }\StringTok{"row2"}\NormalTok{),}
                               \KeywordTok{c}\NormalTok{(}\StringTok{"C.1"}\NormalTok{, }\StringTok{"C.2"}\NormalTok{, }\StringTok{"C.3"}\NormalTok{)))}
\NormalTok{mdat}
\end{Highlighting}
\end{Shaded}

\begin{verbatim}
##      C.1 C.2 C.3
## row1   1   2   3
## row2  11  12  13
\end{verbatim}

\begin{rmdstyle}
El nombre de los argumentos no es necesario para que una función sea
interpretada correctamente por R. Sin embargo, es mejor usar
explicitamente el nombre de los argumentos seguidos por el signo
\texttt{=} para que el código sea más legible.
\end{rmdstyle}

\begin{Shaded}
\begin{Highlighting}[]
\CommentTok{# buen ejemplo}
\NormalTok{mdat <-}\StringTok{ }\KeywordTok{matrix}\NormalTok{(}\KeywordTok{c}\NormalTok{(}\DecValTok{1}\NormalTok{, }\DecValTok{2}\NormalTok{, }\DecValTok{3}\NormalTok{, }\DecValTok{11}\NormalTok{, }\DecValTok{12}\NormalTok{, }\DecValTok{13}\NormalTok{), }\DataTypeTok{nrow =} \DecValTok{2}\NormalTok{, }\DataTypeTok{ncol =} \DecValTok{3}\NormalTok{, }\DataTypeTok{byrow =} \OtherTok{TRUE}\NormalTok{)}
\CommentTok{# mal ejemplo}
\NormalTok{mdat <-}\StringTok{ }\KeywordTok{matrix}\NormalTok{(}\KeywordTok{c}\NormalTok{(}\DecValTok{1}\NormalTok{, }\DecValTok{2}\NormalTok{, }\DecValTok{3}\NormalTok{, }\DecValTok{11}\NormalTok{, }\DecValTok{12}\NormalTok{, }\DecValTok{13}\NormalTok{), }\DecValTok{2}\NormalTok{, }\DecValTok{3}\NormalTok{, }\OtherTok{TRUE}\NormalTok{)}
\end{Highlighting}
\end{Shaded}

\subsubsection{\texorpdfstring{\texttt{help.search()}}{help.search()}}\label{l015helpsearch}

La función \texttt{help.search()} o \texttt{??} permite buscar una
expresión en toda la documentación. Es útil cuando buscamos una función
sin saber el nombre exacto de la función en R.

\begin{Shaded}
\begin{Highlighting}[]
\KeywordTok{help.search}\NormalTok{(}\StringTok{"average"}\NormalTok{)}
\end{Highlighting}
\end{Shaded}

La función \texttt{help.search()} devuelve una página que contiene la
lista de páginas donde se encontró la expresión en la forma
\texttt{package-name::function-name}.

\subsection{Ver los datos}\label{ver-los-datos}

\subsubsection{\texorpdfstring{\texttt{str()}}{str()}}\label{l015str}

La función \texttt{str()} permite visualizar la estructura interna de un
objeto, como se indica en la documentación que podemos consultar con
\texttt{help(str)}.

\begin{Shaded}
\begin{Highlighting}[]
\KeywordTok{str}\NormalTok{(iris)}
\end{Highlighting}
\end{Shaded}

\begin{verbatim}
## 'data.frame':    150 obs. of  5 variables:
##  $ Sepal.Length: num  5.1 4.9 4.7 4.6 5 5.4 4.6 5 4.4 4.9 ...
##  $ Sepal.Width : num  3.5 3 3.2 3.1 3.6 3.9 3.4 3.4 2.9 3.1 ...
##  $ Petal.Length: num  1.4 1.4 1.3 1.5 1.4 1.7 1.4 1.5 1.4 1.5 ...
##  $ Petal.Width : num  0.2 0.2 0.2 0.2 0.2 0.4 0.3 0.2 0.2 0.1 ...
##  $ Species     : Factor w/ 3 levels "setosa","versicolor",..: 1 1 1 1 1 1 1 1 1 1 ...
\end{verbatim}

La función \texttt{str()} devuelve el tipo de objeto
(\texttt{data.frame}), el número de observaciones (150), el número de
variables (5), el nombre de cada variable (\texttt{Sepal.Length} ,
\texttt{Sepal.Width}, \texttt{Petal.Length}, \texttt{Petal.Width}, y
\texttt{Species}), el tipo de cada variable
(\texttt{num},\texttt{Factor}), y los primeros valores de cada una de
las variables. Es una función útil para echar un vistazo a un conjunto
de datos, pero también para verificar que los datos sean del tipo
requirido antes de realizar un análisis estadístico.

\subsubsection{\texorpdfstring{\texttt{head()} y
\texttt{tail()}}{head() y tail()}}\label{l015head}

La función \texttt{head()} devuelve los primeros valores de un objeto, y
la función \texttt{tail()} devuelve los últimos valores de un objeto.
Por defecto, se devuelven seis valores, el argumento \texttt{n} controla
el número de valores a devolver.

\begin{Shaded}
\begin{Highlighting}[]
\KeywordTok{head}\NormalTok{(iris)}
\end{Highlighting}
\end{Shaded}

\begin{verbatim}
##   Sepal.Length Sepal.Width Petal.Length Petal.Width Species
## 1          5.1         3.5          1.4         0.2  setosa
## 2          4.9         3.0          1.4         0.2  setosa
## 3          4.7         3.2          1.3         0.2  setosa
## 4          4.6         3.1          1.5         0.2  setosa
## 5          5.0         3.6          1.4         0.2  setosa
## 6          5.4         3.9          1.7         0.4  setosa
\end{verbatim}

\begin{Shaded}
\begin{Highlighting}[]
\KeywordTok{tail}\NormalTok{(iris)}
\end{Highlighting}
\end{Shaded}

\begin{verbatim}
##     Sepal.Length Sepal.Width Petal.Length Petal.Width   Species
## 145          6.7         3.3          5.7         2.5 virginica
## 146          6.7         3.0          5.2         2.3 virginica
## 147          6.3         2.5          5.0         1.9 virginica
## 148          6.5         3.0          5.2         2.0 virginica
## 149          6.2         3.4          5.4         2.3 virginica
## 150          5.9         3.0          5.1         1.8 virginica
\end{verbatim}

\begin{Shaded}
\begin{Highlighting}[]
\KeywordTok{head}\NormalTok{(iris, }\DataTypeTok{n =} \DecValTok{2}\NormalTok{)}
\end{Highlighting}
\end{Shaded}

\begin{verbatim}
##   Sepal.Length Sepal.Width Petal.Length Petal.Width Species
## 1          5.1         3.5          1.4         0.2  setosa
## 2          4.9         3.0          1.4         0.2  setosa
\end{verbatim}

\subsubsection{\texorpdfstring{\texttt{names()}}{names()}}\label{l015names}

Ya hemos visto la función \texttt{names()}, que permite conocer los
nombres de los elementos de un objeto, pero también asignar nombres a
los elementos de un objeto como a un
\protect\hyperlink{l014matrix}{\texttt{matrix}}, a una
\protect\hyperlink{l014list}{\texttt{list}} o a un
\protect\hyperlink{l014dataframe}{\texttt{data.frame}}.

\begin{Shaded}
\begin{Highlighting}[]
\KeywordTok{names}\NormalTok{(iris)}
\end{Highlighting}
\end{Shaded}

\begin{verbatim}
## [1] "Sepal.Length" "Sepal.Width"  "Petal.Length" "Petal.Width" 
## [5] "Species"
\end{verbatim}

\begin{Shaded}
\begin{Highlighting}[]
\NormalTok{irisCopy <-}\StringTok{ }\NormalTok{iris}
\KeywordTok{names}\NormalTok{(irisCopy) <-}\StringTok{ }\KeywordTok{c}\NormalTok{(}\StringTok{"a"}\NormalTok{, }\StringTok{"b"}\NormalTok{, }\StringTok{"c"}\NormalTok{, }\StringTok{"d"}\NormalTok{, }\StringTok{"e"}\NormalTok{)}
\KeywordTok{names}\NormalTok{(irisCopy)}
\end{Highlighting}
\end{Shaded}

\begin{verbatim}
## [1] "a" "b" "c" "d" "e"
\end{verbatim}

\subsubsection{\texorpdfstring{\texttt{cat()} et
\texttt{print()}}{cat() et print()}}\label{l015print}

La función \texttt{cat()} se usa para mostrar el contenido de un objeto
mientras que la función \texttt{print()} devuelve el valor de un objeto
con la capacidad de realizar conversiones.

\begin{Shaded}
\begin{Highlighting}[]
\KeywordTok{cat}\NormalTok{(}\KeywordTok{names}\NormalTok{(iris))}
\end{Highlighting}
\end{Shaded}

\begin{verbatim}
## Sepal.Length Sepal.Width Petal.Length Petal.Width Species
\end{verbatim}

\begin{Shaded}
\begin{Highlighting}[]
\KeywordTok{print}\NormalTok{(}\KeywordTok{names}\NormalTok{(iris))}
\end{Highlighting}
\end{Shaded}

\begin{verbatim}
## [1] "Sepal.Length" "Sepal.Width"  "Petal.Length" "Petal.Width" 
## [5] "Species"
\end{verbatim}

\begin{Shaded}
\begin{Highlighting}[]
\KeywordTok{cat}\NormalTok{(iris[}\DecValTok{1}\NormalTok{, }\DecValTok{1}\NormalTok{])}
\end{Highlighting}
\end{Shaded}

\begin{verbatim}
## 5.1
\end{verbatim}

\begin{Shaded}
\begin{Highlighting}[]
\KeywordTok{print}\NormalTok{(iris[}\DecValTok{1}\NormalTok{, }\DecValTok{1}\NormalTok{])}
\end{Highlighting}
\end{Shaded}

\begin{verbatim}
## [1] 5.1
\end{verbatim}

\begin{Shaded}
\begin{Highlighting}[]
\KeywordTok{print}\NormalTok{(iris[}\DecValTok{1}\NormalTok{, }\DecValTok{1}\NormalTok{], }\DataTypeTok{digits =} \DecValTok{0}\NormalTok{)}
\end{Highlighting}
\end{Shaded}

\begin{verbatim}
## [1] 5
\end{verbatim}

\subsection{Manipular los datos}\label{manipular-los-datos}

\subsubsection{\texorpdfstring{\texttt{rank()}}{rank()}}\label{l015rank}

La función \texttt{rank()} devuelve el número de la posición ordenada de
cada elemento de un conjunto de elementos. En el caso de elementos del
mismo valor, el argumento \texttt{ties.method} hace posible hacer una
elección sobre la clasificación. Como con todas las funciones, los
detalles están presentes en la documentación.

\begin{Shaded}
\begin{Highlighting}[]
\NormalTok{vecManip <-}\StringTok{ }\KeywordTok{c}\NormalTok{(}\DecValTok{10}\NormalTok{, }\DecValTok{20}\NormalTok{, }\DecValTok{30}\NormalTok{, }\DecValTok{70}\NormalTok{, }\DecValTok{60}\NormalTok{, }\DecValTok{50}\NormalTok{, }\DecValTok{40}\NormalTok{)}
\KeywordTok{rank}\NormalTok{(vecManip)}
\end{Highlighting}
\end{Shaded}

\begin{verbatim}
## [1] 1 2 3 7 6 5 4
\end{verbatim}

\begin{Shaded}
\begin{Highlighting}[]
\NormalTok{vecManip2 <-}\StringTok{ }\KeywordTok{c}\NormalTok{(}\DecValTok{10}\NormalTok{, }\DecValTok{20}\NormalTok{, }\DecValTok{30}\NormalTok{, }\DecValTok{10}\NormalTok{, }\DecValTok{50}\NormalTok{, }\DecValTok{10}\NormalTok{, }\DecValTok{40}\NormalTok{)}
\KeywordTok{rank}\NormalTok{(vecManip2)}
\end{Highlighting}
\end{Shaded}

\begin{verbatim}
## [1] 2 4 5 2 7 2 6
\end{verbatim}

\begin{Shaded}
\begin{Highlighting}[]
\KeywordTok{rank}\NormalTok{(vecManip2, }\DataTypeTok{ties.method =} \StringTok{"first"}\NormalTok{)}
\end{Highlighting}
\end{Shaded}

\begin{verbatim}
## [1] 1 4 5 2 7 3 6
\end{verbatim}

\begin{Shaded}
\begin{Highlighting}[]
\KeywordTok{rank}\NormalTok{(vecManip2, }\DataTypeTok{ties.method =} \StringTok{"min"}\NormalTok{)}
\end{Highlighting}
\end{Shaded}

\begin{verbatim}
## [1] 1 4 5 1 7 1 6
\end{verbatim}

\begin{Shaded}
\begin{Highlighting}[]
\KeywordTok{print}\NormalTok{(iris[, }\DecValTok{1}\NormalTok{])}
\end{Highlighting}
\end{Shaded}

\begin{verbatim}
##   [1] 5.1 4.9 4.7 4.6 5.0 5.4 4.6 5.0 4.4 4.9 5.4 4.8 4.8 4.3 5.8 5.7 5.4
##  [18] 5.1 5.7 5.1 5.4 5.1 4.6 5.1 4.8 5.0 5.0 5.2 5.2 4.7 4.8 5.4 5.2 5.5
##  [35] 4.9 5.0 5.5 4.9 4.4 5.1 5.0 4.5 4.4 5.0 5.1 4.8 5.1 4.6 5.3 5.0 7.0
##  [52] 6.4 6.9 5.5 6.5 5.7 6.3 4.9 6.6 5.2 5.0 5.9 6.0 6.1 5.6 6.7 5.6 5.8
##  [69] 6.2 5.6 5.9 6.1 6.3 6.1 6.4 6.6 6.8 6.7 6.0 5.7 5.5 5.5 5.8 6.0 5.4
##  [86] 6.0 6.7 6.3 5.6 5.5 5.5 6.1 5.8 5.0 5.6 5.7 5.7 6.2 5.1 5.7 6.3 5.8
## [103] 7.1 6.3 6.5 7.6 4.9 7.3 6.7 7.2 6.5 6.4 6.8 5.7 5.8 6.4 6.5 7.7 7.7
## [120] 6.0 6.9 5.6 7.7 6.3 6.7 7.2 6.2 6.1 6.4 7.2 7.4 7.9 6.4 6.3 6.1 7.7
## [137] 6.3 6.4 6.0 6.9 6.7 6.9 5.8 6.8 6.7 6.7 6.3 6.5 6.2 5.9
\end{verbatim}

\begin{Shaded}
\begin{Highlighting}[]
\KeywordTok{rank}\NormalTok{(iris[, }\DecValTok{1}\NormalTok{], }\DataTypeTok{ties.method =} \StringTok{"average"}\NormalTok{)}
\end{Highlighting}
\end{Shaded}

\begin{verbatim}
##   [1]  37.0  19.5  10.5   7.5  27.5  49.5   7.5  27.5   3.0  19.5  49.5
##  [12]  14.0  14.0   1.0  77.0  69.5  49.5  37.0  69.5  37.0  49.5  37.0
##  [23]   7.5  37.0  14.0  27.5  27.5  43.5  43.5  10.5  14.0  49.5  43.5
##  [34]  56.0  19.5  27.5  56.0  19.5   3.0  37.0  27.5   5.0   3.0  27.5
##  [45]  37.0  14.0  37.0   7.5  46.0  27.5 138.0 112.0 135.5  56.0 118.0
##  [56]  69.5 104.0  19.5 121.5  43.5  27.5  82.0  86.5  92.5  62.5 126.5
##  [67]  62.5  77.0  97.5  62.5  82.0  92.5 104.0  92.5 112.0 121.5 132.0
##  [78] 126.5  86.5  69.5  56.0  56.0  77.0  86.5  49.5  86.5 126.5 104.0
##  [89]  62.5  56.0  56.0  92.5  77.0  27.5  62.5  69.5  69.5  97.5  37.0
## [100]  69.5 104.0  77.0 139.0 104.0 118.0 145.0  19.5 143.0 126.5 141.0
## [111] 118.0 112.0 132.0  69.5  77.0 112.0 118.0 147.5 147.5  86.5 135.5
## [122]  62.5 147.5 104.0 126.5 141.0  97.5  92.5 112.0 141.0 144.0 150.0
## [133] 112.0 104.0  92.5 147.5 104.0 112.0  86.5 135.5 126.5 135.5  77.0
## [144] 132.0 126.5 126.5 104.0 118.0  97.5  82.0
\end{verbatim}

\begin{Shaded}
\begin{Highlighting}[]
\CommentTok{# help(rank)}
\CommentTok{# ...}
\CommentTok{# Usage}
\CommentTok{# rank(x, na.last = TRUE,}
\CommentTok{#     ties.method = c("average", "first", "last", }
\CommentTok{#       "random", "max", "min"))}
\end{Highlighting}
\end{Shaded}

\subsubsection{\texorpdfstring{\texttt{order()}}{order()}}\label{l015order}

La función \texttt{order()} devuelve el número de la reorganización de
los elementos en función de su posición. Es muy útil, por ejemplo, para
ordenar un \texttt{data.frame} en función de una columna.

\begin{Shaded}
\begin{Highlighting}[]
\KeywordTok{print}\NormalTok{(vecManip2)}
\end{Highlighting}
\end{Shaded}

\begin{verbatim}
## [1] 10 20 30 10 50 10 40
\end{verbatim}

\begin{Shaded}
\begin{Highlighting}[]
\KeywordTok{rank}\NormalTok{(vecManip2)}
\end{Highlighting}
\end{Shaded}

\begin{verbatim}
## [1] 2 4 5 2 7 2 6
\end{verbatim}

\begin{Shaded}
\begin{Highlighting}[]
\KeywordTok{order}\NormalTok{(vecManip2)}
\end{Highlighting}
\end{Shaded}

\begin{verbatim}
## [1] 1 4 6 2 3 7 5
\end{verbatim}

\begin{Shaded}
\begin{Highlighting}[]
\KeywordTok{print}\NormalTok{(iris[, }\DecValTok{1}\NormalTok{])}
\end{Highlighting}
\end{Shaded}

\begin{verbatim}
##   [1] 5.1 4.9 4.7 4.6 5.0 5.4 4.6 5.0 4.4 4.9 5.4 4.8 4.8 4.3 5.8 5.7 5.4
##  [18] 5.1 5.7 5.1 5.4 5.1 4.6 5.1 4.8 5.0 5.0 5.2 5.2 4.7 4.8 5.4 5.2 5.5
##  [35] 4.9 5.0 5.5 4.9 4.4 5.1 5.0 4.5 4.4 5.0 5.1 4.8 5.1 4.6 5.3 5.0 7.0
##  [52] 6.4 6.9 5.5 6.5 5.7 6.3 4.9 6.6 5.2 5.0 5.9 6.0 6.1 5.6 6.7 5.6 5.8
##  [69] 6.2 5.6 5.9 6.1 6.3 6.1 6.4 6.6 6.8 6.7 6.0 5.7 5.5 5.5 5.8 6.0 5.4
##  [86] 6.0 6.7 6.3 5.6 5.5 5.5 6.1 5.8 5.0 5.6 5.7 5.7 6.2 5.1 5.7 6.3 5.8
## [103] 7.1 6.3 6.5 7.6 4.9 7.3 6.7 7.2 6.5 6.4 6.8 5.7 5.8 6.4 6.5 7.7 7.7
## [120] 6.0 6.9 5.6 7.7 6.3 6.7 7.2 6.2 6.1 6.4 7.2 7.4 7.9 6.4 6.3 6.1 7.7
## [137] 6.3 6.4 6.0 6.9 6.7 6.9 5.8 6.8 6.7 6.7 6.3 6.5 6.2 5.9
\end{verbatim}

\begin{Shaded}
\begin{Highlighting}[]
\KeywordTok{rank}\NormalTok{(iris[, }\DecValTok{1}\NormalTok{])}
\end{Highlighting}
\end{Shaded}

\begin{verbatim}
##   [1]  37.0  19.5  10.5   7.5  27.5  49.5   7.5  27.5   3.0  19.5  49.5
##  [12]  14.0  14.0   1.0  77.0  69.5  49.5  37.0  69.5  37.0  49.5  37.0
##  [23]   7.5  37.0  14.0  27.5  27.5  43.5  43.5  10.5  14.0  49.5  43.5
##  [34]  56.0  19.5  27.5  56.0  19.5   3.0  37.0  27.5   5.0   3.0  27.5
##  [45]  37.0  14.0  37.0   7.5  46.0  27.5 138.0 112.0 135.5  56.0 118.0
##  [56]  69.5 104.0  19.5 121.5  43.5  27.5  82.0  86.5  92.5  62.5 126.5
##  [67]  62.5  77.0  97.5  62.5  82.0  92.5 104.0  92.5 112.0 121.5 132.0
##  [78] 126.5  86.5  69.5  56.0  56.0  77.0  86.5  49.5  86.5 126.5 104.0
##  [89]  62.5  56.0  56.0  92.5  77.0  27.5  62.5  69.5  69.5  97.5  37.0
## [100]  69.5 104.0  77.0 139.0 104.0 118.0 145.0  19.5 143.0 126.5 141.0
## [111] 118.0 112.0 132.0  69.5  77.0 112.0 118.0 147.5 147.5  86.5 135.5
## [122]  62.5 147.5 104.0 126.5 141.0  97.5  92.5 112.0 141.0 144.0 150.0
## [133] 112.0 104.0  92.5 147.5 104.0 112.0  86.5 135.5 126.5 135.5  77.0
## [144] 132.0 126.5 126.5 104.0 118.0  97.5  82.0
\end{verbatim}

\begin{Shaded}
\begin{Highlighting}[]
\KeywordTok{order}\NormalTok{(iris[, }\DecValTok{1}\NormalTok{])}
\end{Highlighting}
\end{Shaded}

\begin{verbatim}
##   [1]  14   9  39  43  42   4   7  23  48   3  30  12  13  25  31  46   2
##  [18]  10  35  38  58 107   5   8  26  27  36  41  44  50  61  94   1  18
##  [35]  20  22  24  40  45  47  99  28  29  33  60  49   6  11  17  21  32
##  [52]  85  34  37  54  81  82  90  91  65  67  70  89  95 122  16  19  56
##  [69]  80  96  97 100 114  15  68  83  93 102 115 143  62  71 150  63  79
##  [86]  84  86 120 139  64  72  74  92 128 135  69  98 127 149  57  73  88
## [103] 101 104 124 134 137 147  52  75 112 116 129 133 138  55 105 111 117
## [120] 148  59  76  66  78  87 109 125 141 145 146  77 113 144  53 121 140
## [137] 142  51 103 110 126 130 108 131 106 118 119 123 136 132
\end{verbatim}

\begin{Shaded}
\begin{Highlighting}[]
\KeywordTok{head}\NormalTok{(iris[}\KeywordTok{order}\NormalTok{(iris[, }\DecValTok{1}\NormalTok{]),], }\DataTypeTok{n =} \DecValTok{10}\NormalTok{)}
\end{Highlighting}
\end{Shaded}

\begin{verbatim}
##    Sepal.Length Sepal.Width Petal.Length Petal.Width Species
## 14          4.3         3.0          1.1         0.1  setosa
## 9           4.4         2.9          1.4         0.2  setosa
## 39          4.4         3.0          1.3         0.2  setosa
## 43          4.4         3.2          1.3         0.2  setosa
## 42          4.5         2.3          1.3         0.3  setosa
## 4           4.6         3.1          1.5         0.2  setosa
## 7           4.6         3.4          1.4         0.3  setosa
## 23          4.6         3.6          1.0         0.2  setosa
## 48          4.6         3.2          1.4         0.2  setosa
## 3           4.7         3.2          1.3         0.2  setosa
\end{verbatim}

\subsubsection{\texorpdfstring{\texttt{sort()}}{sort()}}\label{l015sort}

La función \texttt{sort()} se usa para ordenar los elementos de un
objeto. No permite la clasificación por más de una variable, como es el
caso de \texttt{order()}.

\begin{Shaded}
\begin{Highlighting}[]
\KeywordTok{print}\NormalTok{(vecManip2)}
\end{Highlighting}
\end{Shaded}

\begin{verbatim}
## [1] 10 20 30 10 50 10 40
\end{verbatim}

\begin{Shaded}
\begin{Highlighting}[]
\KeywordTok{sort}\NormalTok{(vecManip2)}
\end{Highlighting}
\end{Shaded}

\begin{verbatim}
## [1] 10 10 10 20 30 40 50
\end{verbatim}

\begin{Shaded}
\begin{Highlighting}[]
\KeywordTok{print}\NormalTok{(iris[, }\DecValTok{1}\NormalTok{])}
\end{Highlighting}
\end{Shaded}

\begin{verbatim}
##   [1] 5.1 4.9 4.7 4.6 5.0 5.4 4.6 5.0 4.4 4.9 5.4 4.8 4.8 4.3 5.8 5.7 5.4
##  [18] 5.1 5.7 5.1 5.4 5.1 4.6 5.1 4.8 5.0 5.0 5.2 5.2 4.7 4.8 5.4 5.2 5.5
##  [35] 4.9 5.0 5.5 4.9 4.4 5.1 5.0 4.5 4.4 5.0 5.1 4.8 5.1 4.6 5.3 5.0 7.0
##  [52] 6.4 6.9 5.5 6.5 5.7 6.3 4.9 6.6 5.2 5.0 5.9 6.0 6.1 5.6 6.7 5.6 5.8
##  [69] 6.2 5.6 5.9 6.1 6.3 6.1 6.4 6.6 6.8 6.7 6.0 5.7 5.5 5.5 5.8 6.0 5.4
##  [86] 6.0 6.7 6.3 5.6 5.5 5.5 6.1 5.8 5.0 5.6 5.7 5.7 6.2 5.1 5.7 6.3 5.8
## [103] 7.1 6.3 6.5 7.6 4.9 7.3 6.7 7.2 6.5 6.4 6.8 5.7 5.8 6.4 6.5 7.7 7.7
## [120] 6.0 6.9 5.6 7.7 6.3 6.7 7.2 6.2 6.1 6.4 7.2 7.4 7.9 6.4 6.3 6.1 7.7
## [137] 6.3 6.4 6.0 6.9 6.7 6.9 5.8 6.8 6.7 6.7 6.3 6.5 6.2 5.9
\end{verbatim}

\begin{Shaded}
\begin{Highlighting}[]
\KeywordTok{sort}\NormalTok{(iris[, }\DecValTok{1}\NormalTok{])}
\end{Highlighting}
\end{Shaded}

\begin{verbatim}
##   [1] 4.3 4.4 4.4 4.4 4.5 4.6 4.6 4.6 4.6 4.7 4.7 4.8 4.8 4.8 4.8 4.8 4.9
##  [18] 4.9 4.9 4.9 4.9 4.9 5.0 5.0 5.0 5.0 5.0 5.0 5.0 5.0 5.0 5.0 5.1 5.1
##  [35] 5.1 5.1 5.1 5.1 5.1 5.1 5.1 5.2 5.2 5.2 5.2 5.3 5.4 5.4 5.4 5.4 5.4
##  [52] 5.4 5.5 5.5 5.5 5.5 5.5 5.5 5.5 5.6 5.6 5.6 5.6 5.6 5.6 5.7 5.7 5.7
##  [69] 5.7 5.7 5.7 5.7 5.7 5.8 5.8 5.8 5.8 5.8 5.8 5.8 5.9 5.9 5.9 6.0 6.0
##  [86] 6.0 6.0 6.0 6.0 6.1 6.1 6.1 6.1 6.1 6.1 6.2 6.2 6.2 6.2 6.3 6.3 6.3
## [103] 6.3 6.3 6.3 6.3 6.3 6.3 6.4 6.4 6.4 6.4 6.4 6.4 6.4 6.5 6.5 6.5 6.5
## [120] 6.5 6.6 6.6 6.7 6.7 6.7 6.7 6.7 6.7 6.7 6.7 6.8 6.8 6.8 6.9 6.9 6.9
## [137] 6.9 7.0 7.1 7.2 7.2 7.2 7.3 7.4 7.6 7.7 7.7 7.7 7.7 7.9
\end{verbatim}

\subsubsection{\texorpdfstring{\texttt{append()}}{append()}}\label{l015append}

Esta función se usa para agregar un elemento a un \texttt{vector} en una
posición determinada por el argumento \texttt{after}. Esta función
también es más rápida que su alternativa
\protect\hyperlink{l014vector}{\texttt{c()}}.

\begin{Shaded}
\begin{Highlighting}[]
\KeywordTok{print}\NormalTok{(vecManip2)}
\end{Highlighting}
\end{Shaded}

\begin{verbatim}
## [1] 10 20 30 10 50 10 40
\end{verbatim}

\begin{Shaded}
\begin{Highlighting}[]
\KeywordTok{append}\NormalTok{(vecManip2, }\DecValTok{5}\NormalTok{)}
\end{Highlighting}
\end{Shaded}

\begin{verbatim}
## [1] 10 20 30 10 50 10 40  5
\end{verbatim}

\begin{Shaded}
\begin{Highlighting}[]
\KeywordTok{append}\NormalTok{(vecManip2, }\DecValTok{5}\NormalTok{, }\DataTypeTok{after =} \DecValTok{2}\NormalTok{)}
\end{Highlighting}
\end{Shaded}

\begin{verbatim}
## [1] 10 20  5 30 10 50 10 40
\end{verbatim}

\subsubsection{\texorpdfstring{\texttt{cbind()} et
\texttt{rbind()}}{cbind() et rbind()}}\label{l015cbind}

Las funciones \texttt{cbind()} y \texttt{rbind()} permiten combinar
elementos por columna o por línea.

\begin{Shaded}
\begin{Highlighting}[]
\KeywordTok{cbind}\NormalTok{(vecManip2, vecManip2)}
\end{Highlighting}
\end{Shaded}

\begin{verbatim}
##      vecManip2 vecManip2
## [1,]        10        10
## [2,]        20        20
## [3,]        30        30
## [4,]        10        10
## [5,]        50        50
## [6,]        10        10
## [7,]        40        40
\end{verbatim}

\begin{Shaded}
\begin{Highlighting}[]
\KeywordTok{rbind}\NormalTok{(vecManip2, vecManip2)}
\end{Highlighting}
\end{Shaded}

\begin{verbatim}
##           [,1] [,2] [,3] [,4] [,5] [,6] [,7]
## vecManip2   10   20   30   10   50   10   40
## vecManip2   10   20   30   10   50   10   40
\end{verbatim}

\subsubsection{\texorpdfstring{\texttt{paste()} et
\texttt{paste0()}}{paste() et paste0()}}\label{l015paste}

Estas son dos funciones que usaremos mucho a partir de ahora. Las
funciones \texttt{paste()} y \texttt{paste0()} se usan para concatenar
cadenas de texto. La función \texttt{paste0()} es equivalente a
\texttt{paste()} sin proponer un separador entre los elementos a
concatenar. La función \texttt{paste0()} también es más rápida.

\begin{Shaded}
\begin{Highlighting}[]
\KeywordTok{paste}\NormalTok{(}\DecValTok{1}\NormalTok{, }\StringTok{"a"}\NormalTok{)}
\end{Highlighting}
\end{Shaded}

\begin{verbatim}
## [1] "1 a"
\end{verbatim}

\begin{Shaded}
\begin{Highlighting}[]
\KeywordTok{paste0}\NormalTok{(}\DecValTok{1}\NormalTok{, }\StringTok{"a"}\NormalTok{)}
\end{Highlighting}
\end{Shaded}

\begin{verbatim}
## [1] "1a"
\end{verbatim}

\begin{Shaded}
\begin{Highlighting}[]
\KeywordTok{paste}\NormalTok{(}\DecValTok{1}\NormalTok{, }\StringTok{"a"}\NormalTok{, }\DataTypeTok{sep =} \StringTok{"_"}\NormalTok{)}
\end{Highlighting}
\end{Shaded}

\begin{verbatim}
## [1] "1_a"
\end{verbatim}

\begin{Shaded}
\begin{Highlighting}[]
\KeywordTok{paste0}\NormalTok{(}\StringTok{"prefix_"}\NormalTok{, vecManip2, }\StringTok{"_suffix"}\NormalTok{)}
\end{Highlighting}
\end{Shaded}

\begin{verbatim}
## [1] "prefix_10_suffix" "prefix_20_suffix" "prefix_30_suffix"
## [4] "prefix_10_suffix" "prefix_50_suffix" "prefix_10_suffix"
## [7] "prefix_40_suffix"
\end{verbatim}

\begin{Shaded}
\begin{Highlighting}[]
\KeywordTok{paste}\NormalTok{(vecManip2, }\KeywordTok{rank}\NormalTok{(vecManip2), }\DataTypeTok{sep =} \StringTok{"_"}\NormalTok{)}
\end{Highlighting}
\end{Shaded}

\begin{verbatim}
## [1] "10_2" "20_4" "30_5" "10_2" "50_7" "10_2" "40_6"
\end{verbatim}

\subsubsection{\texorpdfstring{\texttt{rev()}}{rev()}}\label{l015rev}

La función \texttt{rev\ ()} devuelve los elementos de un objeto en orden
inverso.

\begin{Shaded}
\begin{Highlighting}[]
\KeywordTok{print}\NormalTok{(vecManip2)}
\end{Highlighting}
\end{Shaded}

\begin{verbatim}
## [1] 10 20 30 10 50 10 40
\end{verbatim}

\begin{Shaded}
\begin{Highlighting}[]
\KeywordTok{rev}\NormalTok{(vecManip2)}
\end{Highlighting}
\end{Shaded}

\begin{verbatim}
## [1] 40 10 50 10 30 20 10
\end{verbatim}

\subsubsection{\texorpdfstring{\texttt{\%in\%()}}{\%in\%()}}\label{l015in}

La función \texttt{\%in\%()} se puede comparar con un
\protect\hyperlink{l011opcomp}{operador de comparación}. Esta función
toma dos objetos como argumentos y devuelve \texttt{TRUE} o
\texttt{FALSE} para cada elemento del primer objeto de acuerdo con su
presencia o ausencia en el segundo objeto. Para acceder a la
documentación de la función, use
\texttt{help(\textquotesingle{}\%in\%\textquotesingle{})} (con comillas
simples).

\begin{Shaded}
\begin{Highlighting}[]
\KeywordTok{print}\NormalTok{(vecManip)}
\end{Highlighting}
\end{Shaded}

\begin{verbatim}
## [1] 10 20 30 70 60 50 40
\end{verbatim}

\begin{Shaded}
\begin{Highlighting}[]
\KeywordTok{print}\NormalTok{(vecManip2)}
\end{Highlighting}
\end{Shaded}

\begin{verbatim}
## [1] 10 20 30 10 50 10 40
\end{verbatim}

\begin{Shaded}
\begin{Highlighting}[]
\NormalTok{vecManip }\OperatorTok\StringTok{ }\NormalTok{vecManip2}
\end{Highlighting}
\end{Shaded}

\begin{verbatim}
## [1]  TRUE  TRUE  TRUE FALSE FALSE  TRUE  TRUE
\end{verbatim}

\begin{Shaded}
\begin{Highlighting}[]
\NormalTok{vecManip2 }\OperatorTok\StringTok{ }\NormalTok{vecManip}
\end{Highlighting}
\end{Shaded}

\begin{verbatim}
## [1] TRUE TRUE TRUE TRUE TRUE TRUE TRUE
\end{verbatim}

\subsection{Funciones matemáticas}\label{l015maths}

Ya hemos visto las funciones \texttt{+}, \texttt{-}, \texttt{*},
\texttt{/}, \texttt{\^{}}, \texttt{\%\%} y otros
\protect\hyperlink{l011opari}{operadores aritméticos}. R también tiene
funciones matemáticas básicas como exponencial \texttt{exp()}, raíz
cuadrada \texttt{sqrt()}, valor absoluto \texttt{abs()}, sinus
\texttt{sin()}, coseno \texttt{cos()}, tangente \texttt{tan()},
logaritmo \texttt{log()}, logaritmo base 10 \texttt{log10()}, arco
coseno \texttt{acos()}, arco sinus \texttt{asin()}, y arco tangente
\texttt{atan()}.

\begin{Shaded}
\begin{Highlighting}[]
\KeywordTok{print}\NormalTok{(vecManip2)}
\end{Highlighting}
\end{Shaded}

\begin{verbatim}
## [1] 10 20 30 10 50 10 40
\end{verbatim}

\begin{Shaded}
\begin{Highlighting}[]
\KeywordTok{exp}\NormalTok{(vecManip2)}
\end{Highlighting}
\end{Shaded}

\begin{verbatim}
## [1] 2.202647e+04 4.851652e+08 1.068647e+13 2.202647e+04 5.184706e+21
## [6] 2.202647e+04 2.353853e+17
\end{verbatim}

\begin{Shaded}
\begin{Highlighting}[]
\KeywordTok{sqrt}\NormalTok{(vecManip2)}
\end{Highlighting}
\end{Shaded}

\begin{verbatim}
## [1] 3.162278 4.472136 5.477226 3.162278 7.071068 3.162278 6.324555
\end{verbatim}

\begin{Shaded}
\begin{Highlighting}[]
\KeywordTok{abs}\NormalTok{(}\OperatorTok{-}\NormalTok{vecManip2)}
\end{Highlighting}
\end{Shaded}

\begin{verbatim}
## [1] 10 20 30 10 50 10 40
\end{verbatim}

\begin{Shaded}
\begin{Highlighting}[]
\KeywordTok{sin}\NormalTok{(vecManip2)}
\end{Highlighting}
\end{Shaded}

\begin{verbatim}
## [1] -0.5440211  0.9129453 -0.9880316 -0.5440211 -0.2623749 -0.5440211
## [7]  0.7451132
\end{verbatim}

\begin{Shaded}
\begin{Highlighting}[]
\KeywordTok{cos}\NormalTok{(vecManip2)}
\end{Highlighting}
\end{Shaded}

\begin{verbatim}
## [1] -0.8390715  0.4080821  0.1542514 -0.8390715  0.9649660 -0.8390715
## [7] -0.6669381
\end{verbatim}

\begin{Shaded}
\begin{Highlighting}[]
\KeywordTok{tan}\NormalTok{(vecManip2)}
\end{Highlighting}
\end{Shaded}

\begin{verbatim}
## [1]  0.6483608  2.2371609 -6.4053312  0.6483608 -0.2719006  0.6483608
## [7] -1.1172149
\end{verbatim}

\begin{Shaded}
\begin{Highlighting}[]
\KeywordTok{log}\NormalTok{(vecManip2)}
\end{Highlighting}
\end{Shaded}

\begin{verbatim}
## [1] 2.302585 2.995732 3.401197 2.302585 3.912023 2.302585 3.688879
\end{verbatim}

\begin{Shaded}
\begin{Highlighting}[]
\KeywordTok{log10}\NormalTok{(vecManip2)}
\end{Highlighting}
\end{Shaded}

\begin{verbatim}
## [1] 1.000000 1.301030 1.477121 1.000000 1.698970 1.000000 1.602060
\end{verbatim}

\begin{Shaded}
\begin{Highlighting}[]
\KeywordTok{acos}\NormalTok{(vecManip2}\OperatorTok{/}\DecValTok{100}\NormalTok{)}
\end{Highlighting}
\end{Shaded}

\begin{verbatim}
## [1] 1.470629 1.369438 1.266104 1.470629 1.047198 1.470629 1.159279
\end{verbatim}

\begin{Shaded}
\begin{Highlighting}[]
\KeywordTok{asin}\NormalTok{(vecManip2}\OperatorTok{/}\DecValTok{100}\NormalTok{)}
\end{Highlighting}
\end{Shaded}

\begin{verbatim}
## [1] 0.1001674 0.2013579 0.3046927 0.1001674 0.5235988 0.1001674 0.4115168
\end{verbatim}

\begin{Shaded}
\begin{Highlighting}[]
\KeywordTok{atan}\NormalTok{(vecManip2}\OperatorTok{/}\DecValTok{100}\NormalTok{)}
\end{Highlighting}
\end{Shaded}

\begin{verbatim}
## [1] 0.09966865 0.19739556 0.29145679 0.09966865 0.46364761 0.09966865
## [7] 0.38050638
\end{verbatim}

\subsection{Estadísticas descriptivas}\label{estadisticas-descriptivas}

También podemos realizar estadísticas descriptivas de forma muy simple a
partir de un conjunto de datos.

\subsubsection{\texorpdfstring{\texttt{mean()}}{mean()}}\label{l015mean}

La función \texttt{mean()} devuelve la media. Para ignorar los valores
faltantes \texttt{NA}, hay que afectar el valor \texttt{TRUE} al
argumento \texttt{na.rm()}.

\begin{Shaded}
\begin{Highlighting}[]
\KeywordTok{mean}\NormalTok{(iris[, }\DecValTok{1}\NormalTok{])}
\end{Highlighting}
\end{Shaded}

\begin{verbatim}
## [1] 5.843333
\end{verbatim}

\begin{Shaded}
\begin{Highlighting}[]
\NormalTok{vecManip3 <-}\StringTok{ }\KeywordTok{c}\NormalTok{(}\DecValTok{1}\NormalTok{, }\DecValTok{5}\NormalTok{, }\DecValTok{6}\NormalTok{, }\DecValTok{8}\NormalTok{, }\OtherTok{NA}\NormalTok{, }\DecValTok{45}\NormalTok{, }\OtherTok{NA}\NormalTok{, }\DecValTok{14}\NormalTok{)}
\KeywordTok{mean}\NormalTok{(vecManip3)}
\end{Highlighting}
\end{Shaded}

\begin{verbatim}
## [1] NA
\end{verbatim}

\begin{Shaded}
\begin{Highlighting}[]
\KeywordTok{mean}\NormalTok{(vecManip3, }\DataTypeTok{na.rm =} \OtherTok{TRUE}\NormalTok{)}
\end{Highlighting}
\end{Shaded}

\begin{verbatim}
## [1] 13.16667
\end{verbatim}

\subsubsection{\texorpdfstring{\texttt{sd()}}{sd()}}\label{l015sd}

La función \texttt{sd()} devuelve la desviación estándar.

\begin{Shaded}
\begin{Highlighting}[]
\KeywordTok{sd}\NormalTok{(iris[, }\DecValTok{1}\NormalTok{])}
\end{Highlighting}
\end{Shaded}

\begin{verbatim}
## [1] 0.8280661
\end{verbatim}

\begin{Shaded}
\begin{Highlighting}[]
\KeywordTok{print}\NormalTok{(vecManip3)}
\end{Highlighting}
\end{Shaded}

\begin{verbatim}
## [1]  1  5  6  8 NA 45 NA 14
\end{verbatim}

\begin{Shaded}
\begin{Highlighting}[]
\KeywordTok{sd}\NormalTok{(vecManip3)}
\end{Highlighting}
\end{Shaded}

\begin{verbatim}
## [1] NA
\end{verbatim}

\begin{Shaded}
\begin{Highlighting}[]
\KeywordTok{sd}\NormalTok{(vecManip3, }\DataTypeTok{na.rm =} \OtherTok{TRUE}\NormalTok{)}
\end{Highlighting}
\end{Shaded}

\begin{verbatim}
## [1] 16.16684
\end{verbatim}

\subsubsection{\texorpdfstring{\texttt{max()} y
\texttt{min()}}{max() y min()}}\label{l015max}

La función \texttt{max()} devuelve el valor máximo y \texttt{min()} el
valor mínimo.

\begin{Shaded}
\begin{Highlighting}[]
\KeywordTok{max}\NormalTok{(iris[, }\DecValTok{1}\NormalTok{])}
\end{Highlighting}
\end{Shaded}

\begin{verbatim}
## [1] 7.9
\end{verbatim}

\begin{Shaded}
\begin{Highlighting}[]
\KeywordTok{print}\NormalTok{(vecManip3)}
\end{Highlighting}
\end{Shaded}

\begin{verbatim}
## [1]  1  5  6  8 NA 45 NA 14
\end{verbatim}

\begin{Shaded}
\begin{Highlighting}[]
\KeywordTok{max}\NormalTok{(vecManip3)}
\end{Highlighting}
\end{Shaded}

\begin{verbatim}
## [1] NA
\end{verbatim}

\begin{Shaded}
\begin{Highlighting}[]
\KeywordTok{max}\NormalTok{(vecManip3, }\DataTypeTok{na.rm =} \OtherTok{TRUE}\NormalTok{)}
\end{Highlighting}
\end{Shaded}

\begin{verbatim}
## [1] 45
\end{verbatim}

\begin{Shaded}
\begin{Highlighting}[]
\KeywordTok{min}\NormalTok{(iris[, }\DecValTok{1}\NormalTok{])}
\end{Highlighting}
\end{Shaded}

\begin{verbatim}
## [1] 4.3
\end{verbatim}

\begin{Shaded}
\begin{Highlighting}[]
\KeywordTok{min}\NormalTok{(vecManip3)}
\end{Highlighting}
\end{Shaded}

\begin{verbatim}
## [1] NA
\end{verbatim}

\begin{Shaded}
\begin{Highlighting}[]
\KeywordTok{min}\NormalTok{(vecManip3, }\DataTypeTok{na.rm =} \OtherTok{TRUE}\NormalTok{)}
\end{Highlighting}
\end{Shaded}

\begin{verbatim}
## [1] 1
\end{verbatim}

\subsubsection{\texorpdfstring{\texttt{quantile()}}{quantile()}}\label{l015quantile}

La función \texttt{quantile()} devuelve el cuantil definido por el
argumento \texttt{probs}.

\begin{Shaded}
\begin{Highlighting}[]
\KeywordTok{quantile}\NormalTok{(iris[, }\DecValTok{1}\NormalTok{])}
\end{Highlighting}
\end{Shaded}

\begin{verbatim}
##   0%  25%  50%  75% 100% 
##  4.3  5.1  5.8  6.4  7.9
\end{verbatim}

\begin{Shaded}
\begin{Highlighting}[]
\KeywordTok{quantile}\NormalTok{(iris[, }\DecValTok{1}\NormalTok{], }\DataTypeTok{probs =} \KeywordTok{c}\NormalTok{(}\DecValTok{0}\NormalTok{, }\FloatTok{0.25}\NormalTok{, }\FloatTok{0.5}\NormalTok{, }\FloatTok{0.75}\NormalTok{, }\DecValTok{1}\NormalTok{))}
\end{Highlighting}
\end{Shaded}

\begin{verbatim}
##   0%  25%  50%  75% 100% 
##  4.3  5.1  5.8  6.4  7.9
\end{verbatim}

\begin{Shaded}
\begin{Highlighting}[]
\KeywordTok{quantile}\NormalTok{(iris[, }\DecValTok{1}\NormalTok{], }\DataTypeTok{probs =} \KeywordTok{c}\NormalTok{(}\DecValTok{0}\NormalTok{, }\FloatTok{0.1}\NormalTok{, }\FloatTok{0.5}\NormalTok{, }\FloatTok{0.9}\NormalTok{, }\DecValTok{1}\NormalTok{))}
\end{Highlighting}
\end{Shaded}

\begin{verbatim}
##   0%  10%  50%  90% 100% 
##  4.3  4.8  5.8  6.9  7.9
\end{verbatim}

\subsubsection{\texorpdfstring{\texttt{summary()}}{summary()}}\label{l015summary}

La función \texttt{summary()} devuelve un resumen con el mínimo, primer
cuartil, mediana, promedio, tercer cuartil y máximo.

\begin{Shaded}
\begin{Highlighting}[]
\KeywordTok{summary}\NormalTok{(iris[, }\DecValTok{1}\NormalTok{])}
\end{Highlighting}
\end{Shaded}

\begin{verbatim}
##    Min. 1st Qu.  Median    Mean 3rd Qu.    Max. 
##   4.300   5.100   5.800   5.843   6.400   7.900
\end{verbatim}

\subsubsection{\texorpdfstring{\texttt{median()}}{median()}}\label{l015median}

La función \texttt{median()} devuelve la mediana.

\begin{Shaded}
\begin{Highlighting}[]
\KeywordTok{median}\NormalTok{(iris[, }\DecValTok{1}\NormalTok{])}
\end{Highlighting}
\end{Shaded}

\begin{verbatim}
## [1] 5.8
\end{verbatim}

\begin{Shaded}
\begin{Highlighting}[]
\KeywordTok{print}\NormalTok{(vecManip3)}
\end{Highlighting}
\end{Shaded}

\begin{verbatim}
## [1]  1  5  6  8 NA 45 NA 14
\end{verbatim}

\begin{Shaded}
\begin{Highlighting}[]
\KeywordTok{median}\NormalTok{(vecManip3)}
\end{Highlighting}
\end{Shaded}

\begin{verbatim}
## [1] NA
\end{verbatim}

\begin{Shaded}
\begin{Highlighting}[]
\KeywordTok{median}\NormalTok{(vecManip3, }\DataTypeTok{na.rm =} \OtherTok{TRUE}\NormalTok{)}
\end{Highlighting}
\end{Shaded}

\begin{verbatim}
## [1] 7
\end{verbatim}

\subsubsection{\texorpdfstring{\texttt{length()}}{length()}}\label{l015length}

La función \texttt{length()} devuelve el tamaño de un objeto (número de
elementos).

\begin{Shaded}
\begin{Highlighting}[]
\KeywordTok{length}\NormalTok{(iris[, }\DecValTok{1}\NormalTok{])}
\end{Highlighting}
\end{Shaded}

\begin{verbatim}
## [1] 150
\end{verbatim}

\begin{Shaded}
\begin{Highlighting}[]
\KeywordTok{print}\NormalTok{(vecManip3)}
\end{Highlighting}
\end{Shaded}

\begin{verbatim}
## [1]  1  5  6  8 NA 45 NA 14
\end{verbatim}

\begin{Shaded}
\begin{Highlighting}[]
\KeywordTok{length}\NormalTok{(vecManip3)}
\end{Highlighting}
\end{Shaded}

\begin{verbatim}
## [1] 8
\end{verbatim}

\subsubsection{\texorpdfstring{\texttt{nrow()} et
\texttt{ncol()}}{nrow() et ncol()}}\label{l015nrow}

La función \texttt{nrow()} devuelve el número de líneas y la función
\texttt{ncol()} el número de columnas en un objeto.

\begin{Shaded}
\begin{Highlighting}[]
\KeywordTok{nrow}\NormalTok{(iris)}
\end{Highlighting}
\end{Shaded}

\begin{verbatim}
## [1] 150
\end{verbatim}

\begin{Shaded}
\begin{Highlighting}[]
\KeywordTok{ncol}\NormalTok{(iris)}
\end{Highlighting}
\end{Shaded}

\begin{verbatim}
## [1] 5
\end{verbatim}

\subsubsection{\texorpdfstring{\texttt{round()}, \texttt{ceiling()},
\texttt{floor()}, et
\texttt{trunc()}}{round(), ceiling(), floor(), et trunc()}}\label{l015round}

La función \texttt{round()} le permite seleccionar una cierta cantidad
de decimales (0 por defecto)

\begin{Shaded}
\begin{Highlighting}[]
\KeywordTok{round}\NormalTok{(}\FloatTok{5.56874258564}\NormalTok{)}
\end{Highlighting}
\end{Shaded}

\begin{verbatim}
## [1] 6
\end{verbatim}

\begin{Shaded}
\begin{Highlighting}[]
\KeywordTok{round}\NormalTok{(}\FloatTok{5.56874258564}\NormalTok{, }\DataTypeTok{digits =} \DecValTok{2}\NormalTok{)}
\end{Highlighting}
\end{Shaded}

\begin{verbatim}
## [1] 5.57
\end{verbatim}

La función \texttt{ceiling()} devuelve el entero más pequeño que no es
inferior al valor especificado.

\begin{Shaded}
\begin{Highlighting}[]
\KeywordTok{ceiling}\NormalTok{(}\FloatTok{5.9999}\NormalTok{)}
\end{Highlighting}
\end{Shaded}

\begin{verbatim}
## [1] 6
\end{verbatim}

\begin{Shaded}
\begin{Highlighting}[]
\KeywordTok{ceiling}\NormalTok{(}\FloatTok{5.0001}\NormalTok{)}
\end{Highlighting}
\end{Shaded}

\begin{verbatim}
## [1] 6
\end{verbatim}

La función \texttt{floor()} devuelve el entero más grande que no excede
el valor especificado.

\begin{Shaded}
\begin{Highlighting}[]
\KeywordTok{floor}\NormalTok{(}\FloatTok{5.9999}\NormalTok{)}
\end{Highlighting}
\end{Shaded}

\begin{verbatim}
## [1] 5
\end{verbatim}

\begin{Shaded}
\begin{Highlighting}[]
\KeywordTok{floor}\NormalTok{(}\FloatTok{5.0001}\NormalTok{)}
\end{Highlighting}
\end{Shaded}

\begin{verbatim}
## [1] 5
\end{verbatim}

La función \texttt{trunc()} devuelve la parte entera del valor
especificado.

\begin{Shaded}
\begin{Highlighting}[]
\KeywordTok{trunc}\NormalTok{(}\FloatTok{5.9999}\NormalTok{)}
\end{Highlighting}
\end{Shaded}

\begin{verbatim}
## [1] 5
\end{verbatim}

\begin{Shaded}
\begin{Highlighting}[]
\KeywordTok{trunc}\NormalTok{(}\FloatTok{5.0001}\NormalTok{)}
\end{Highlighting}
\end{Shaded}

\begin{verbatim}
## [1] 5
\end{verbatim}

\subsubsection{\texorpdfstring{\texttt{rowSums()} et
\texttt{colSums()}}{rowSums() et colSums()}}\label{l015rowsums}

Las funciones \texttt{rowSums()} y \texttt{colSums()} calculan la suma
de filas y columnas.

\begin{Shaded}
\begin{Highlighting}[]
\KeywordTok{rowSums}\NormalTok{(iris[, }\KeywordTok{c}\NormalTok{(}\DecValTok{1}\NormalTok{, }\DecValTok{2}\NormalTok{, }\DecValTok{3}\NormalTok{, }\DecValTok{4}\NormalTok{)])}
\end{Highlighting}
\end{Shaded}

\begin{verbatim}
##   [1] 10.2  9.5  9.4  9.4 10.2 11.4  9.7 10.1  8.9  9.6 10.8 10.0  9.3  8.5
##  [15] 11.2 12.0 11.0 10.3 11.5 10.7 10.7 10.7  9.4 10.6 10.3  9.8 10.4 10.4
##  [29] 10.2  9.7  9.7 10.7 10.9 11.3  9.7  9.6 10.5 10.0  8.9 10.2 10.1  8.4
##  [43]  9.1 10.7 11.2  9.5 10.7  9.4 10.7  9.9 16.3 15.6 16.4 13.1 15.4 14.3
##  [57] 15.9 11.6 15.4 13.2 11.5 14.6 13.2 15.1 13.4 15.6 14.6 13.6 14.4 13.1
##  [71] 15.7 14.2 15.2 14.8 14.9 15.4 15.8 16.4 14.9 12.8 12.8 12.6 13.6 15.4
##  [85] 14.4 15.5 16.0 14.3 14.0 13.3 13.7 15.1 13.6 11.6 13.8 14.1 14.1 14.7
##  [99] 11.7 13.9 18.1 15.5 18.1 16.6 17.5 19.3 13.6 18.3 16.8 19.4 16.8 16.3
## [113] 17.4 15.2 16.1 17.2 16.8 20.4 19.5 14.7 18.1 15.3 19.2 15.7 17.8 18.2
## [127] 15.6 15.8 16.9 17.6 18.2 20.1 17.0 15.7 15.7 19.1 17.7 16.8 15.6 17.5
## [141] 17.8 17.4 15.5 18.2 18.2 17.2 15.7 16.7 17.3 15.8
\end{verbatim}

\begin{Shaded}
\begin{Highlighting}[]
\KeywordTok{colSums}\NormalTok{(iris[, }\KeywordTok{c}\NormalTok{(}\DecValTok{1}\NormalTok{, }\DecValTok{2}\NormalTok{, }\DecValTok{3}\NormalTok{, }\DecValTok{4}\NormalTok{)])}
\end{Highlighting}
\end{Shaded}

\begin{verbatim}
## Sepal.Length  Sepal.Width Petal.Length  Petal.Width 
##        876.5        458.6        563.7        179.9
\end{verbatim}

\subsubsection{\texorpdfstring{\texttt{rowMeans()} et
\texttt{colMeans()}}{rowMeans() et colMeans()}}\label{l015rowmeans}

Las funciones \texttt{rowMeans()} y \texttt{colMeans()} calculan el
promedio de filas y columnas.

\begin{Shaded}
\begin{Highlighting}[]
\KeywordTok{rowMeans}\NormalTok{(iris[, }\KeywordTok{c}\NormalTok{(}\DecValTok{1}\NormalTok{, }\DecValTok{2}\NormalTok{, }\DecValTok{3}\NormalTok{, }\DecValTok{4}\NormalTok{)])}
\end{Highlighting}
\end{Shaded}

\begin{verbatim}
##   [1] 2.550 2.375 2.350 2.350 2.550 2.850 2.425 2.525 2.225 2.400 2.700
##  [12] 2.500 2.325 2.125 2.800 3.000 2.750 2.575 2.875 2.675 2.675 2.675
##  [23] 2.350 2.650 2.575 2.450 2.600 2.600 2.550 2.425 2.425 2.675 2.725
##  [34] 2.825 2.425 2.400 2.625 2.500 2.225 2.550 2.525 2.100 2.275 2.675
##  [45] 2.800 2.375 2.675 2.350 2.675 2.475 4.075 3.900 4.100 3.275 3.850
##  [56] 3.575 3.975 2.900 3.850 3.300 2.875 3.650 3.300 3.775 3.350 3.900
##  [67] 3.650 3.400 3.600 3.275 3.925 3.550 3.800 3.700 3.725 3.850 3.950
##  [78] 4.100 3.725 3.200 3.200 3.150 3.400 3.850 3.600 3.875 4.000 3.575
##  [89] 3.500 3.325 3.425 3.775 3.400 2.900 3.450 3.525 3.525 3.675 2.925
## [100] 3.475 4.525 3.875 4.525 4.150 4.375 4.825 3.400 4.575 4.200 4.850
## [111] 4.200 4.075 4.350 3.800 4.025 4.300 4.200 5.100 4.875 3.675 4.525
## [122] 3.825 4.800 3.925 4.450 4.550 3.900 3.950 4.225 4.400 4.550 5.025
## [133] 4.250 3.925 3.925 4.775 4.425 4.200 3.900 4.375 4.450 4.350 3.875
## [144] 4.550 4.550 4.300 3.925 4.175 4.325 3.950
\end{verbatim}

\begin{Shaded}
\begin{Highlighting}[]
\KeywordTok{colMeans}\NormalTok{(iris[, }\KeywordTok{c}\NormalTok{(}\DecValTok{1}\NormalTok{, }\DecValTok{2}\NormalTok{, }\DecValTok{3}\NormalTok{, }\DecValTok{4}\NormalTok{)])}
\end{Highlighting}
\end{Shaded}

\begin{verbatim}
## Sepal.Length  Sepal.Width Petal.Length  Petal.Width 
##     5.843333     3.057333     3.758000     1.199333
\end{verbatim}

\subsubsection{\texorpdfstring{\texttt{aggregate()}}{aggregate()}}\label{l015aggregate}

La función \texttt{aggregate()} permite agrupar los elementos de un
objeto de acuerdo con un valor. El argumento \texttt{by} define el
elemento sobre el que se realiza la agrupación. Debe ser del tipo
\protect\hyperlink{l014list}{\texttt{list}}.

\begin{Shaded}
\begin{Highlighting}[]
\KeywordTok{aggregate}\NormalTok{(iris[, }\KeywordTok{c}\NormalTok{(}\DecValTok{1}\NormalTok{, }\DecValTok{2}\NormalTok{, }\DecValTok{3}\NormalTok{, }\DecValTok{4}\NormalTok{)], }\DataTypeTok{by =} \KeywordTok{list}\NormalTok{(iris}\OperatorTok{$}\NormalTok{Species), }\DataTypeTok{FUN =}\NormalTok{ mean)}
\end{Highlighting}
\end{Shaded}

\begin{verbatim}
##      Group.1 Sepal.Length Sepal.Width Petal.Length Petal.Width
## 1     setosa        5.006       3.428        1.462       0.246
## 2 versicolor        5.936       2.770        4.260       1.326
## 3  virginica        6.588       2.974        5.552       2.026
\end{verbatim}

\begin{Shaded}
\begin{Highlighting}[]
\KeywordTok{aggregate}\NormalTok{(iris[, }\KeywordTok{c}\NormalTok{(}\DecValTok{1}\NormalTok{, }\DecValTok{2}\NormalTok{)], }\DataTypeTok{by =} \KeywordTok{list}\NormalTok{(iris}\OperatorTok{$}\NormalTok{Species), }\DataTypeTok{FUN =}\NormalTok{ summary)}
\end{Highlighting}
\end{Shaded}

\begin{verbatim}
##      Group.1 Sepal.Length.Min. Sepal.Length.1st Qu. Sepal.Length.Median
## 1     setosa             4.300                4.800               5.000
## 2 versicolor             4.900                5.600               5.900
## 3  virginica             4.900                6.225               6.500
##   Sepal.Length.Mean Sepal.Length.3rd Qu. Sepal.Length.Max.
## 1             5.006                5.200             5.800
## 2             5.936                6.300             7.000
## 3             6.588                6.900             7.900
##   Sepal.Width.Min. Sepal.Width.1st Qu. Sepal.Width.Median Sepal.Width.Mean
## 1            2.300               3.200              3.400            3.428
## 2            2.000               2.525              2.800            2.770
## 3            2.200               2.800              3.000            2.974
##   Sepal.Width.3rd Qu. Sepal.Width.Max.
## 1               3.675            4.400
## 2               3.000            3.400
## 3               3.175            3.800
\end{verbatim}

\subsubsection{\texorpdfstring{\texttt{range()}}{range()}}\label{l015range}

La función \texttt{range()} devuelve el mínimo y el máximo.

\begin{Shaded}
\begin{Highlighting}[]
\KeywordTok{range}\NormalTok{(iris[, }\DecValTok{1}\NormalTok{])}
\end{Highlighting}
\end{Shaded}

\begin{verbatim}
## [1] 4.3 7.9
\end{verbatim}

\begin{Shaded}
\begin{Highlighting}[]
\KeywordTok{print}\NormalTok{(vecManip3)}
\end{Highlighting}
\end{Shaded}

\begin{verbatim}
## [1]  1  5  6  8 NA 45 NA 14
\end{verbatim}

\begin{Shaded}
\begin{Highlighting}[]
\KeywordTok{range}\NormalTok{(vecManip3)}
\end{Highlighting}
\end{Shaded}

\begin{verbatim}
## [1] NA NA
\end{verbatim}

\begin{Shaded}
\begin{Highlighting}[]
\KeywordTok{range}\NormalTok{(vecManip3, }\DataTypeTok{na.rm =} \OtherTok{TRUE}\NormalTok{)}
\end{Highlighting}
\end{Shaded}

\begin{verbatim}
## [1]  1 45
\end{verbatim}

\subsubsection{\texorpdfstring{\texttt{unique()}}{unique()}}\label{l015unique}

La función \texttt{unique()} devuelve los valores únicos de un objeto
(sin duplicados).

\begin{Shaded}
\begin{Highlighting}[]
\KeywordTok{unique}\NormalTok{(iris[, }\DecValTok{1}\NormalTok{])}
\end{Highlighting}
\end{Shaded}

\begin{verbatim}
##  [1] 5.1 4.9 4.7 4.6 5.0 5.4 4.4 4.8 4.3 5.8 5.7 5.2 5.5 4.5 5.3 7.0 6.4
## [18] 6.9 6.5 6.3 6.6 5.9 6.0 6.1 5.6 6.7 6.2 6.8 7.1 7.6 7.3 7.2 7.7 7.4
## [35] 7.9
\end{verbatim}

\begin{Shaded}
\begin{Highlighting}[]
\KeywordTok{print}\NormalTok{(vecManip3)}
\end{Highlighting}
\end{Shaded}

\begin{verbatim}
## [1]  1  5  6  8 NA 45 NA 14
\end{verbatim}

\begin{Shaded}
\begin{Highlighting}[]
\KeywordTok{unique}\NormalTok{(vecManip3)}
\end{Highlighting}
\end{Shaded}

\begin{verbatim}
## [1]  1  5  6  8 NA 45 14
\end{verbatim}

\section{Otras funciones útiles}\label{otras-funciones-utiles}

No podemos abordar todas las funciones útiles, aquí solo abordaremos
ciertas funciones. A lo largo de este libro, se usarán nuevas funciones.
Cuando se utiliza una nueva función, nuestro reflejo siempre debe ser el
mismo: \textbf{consultar la documentación} con la función
\texttt{help()}.

\subsection{\texorpdfstring{\texttt{seq\_along()}}{seq\_along()}}\label{l015seqalong}

La función \texttt{seq\_along()} se usa para crear un \texttt{vector}
del tamaño del objeto rellenado y tomando como valores los números de 1
a N (N corresponde al número de elementos del objeto). Esta función nos
servirá mucho en el capítulo sobre bucles.

\begin{Shaded}
\begin{Highlighting}[]
\KeywordTok{print}\NormalTok{(vecManip3)}
\end{Highlighting}
\end{Shaded}

\begin{verbatim}
## [1]  1  5  6  8 NA 45 NA 14
\end{verbatim}

\begin{Shaded}
\begin{Highlighting}[]
\KeywordTok{seq_along}\NormalTok{(vecManip3)}
\end{Highlighting}
\end{Shaded}

\begin{verbatim}
## [1] 1 2 3 4 5 6 7 8
\end{verbatim}

\subsection{\texorpdfstring{\texttt{:}}{:}}\label{l0152points}

La función \texttt{:} permite crear una secuencia desde \texttt{a} hacia
\texttt{b} por pasos de 1. Ha sido difícil escribir los capítulos
anteriores sin usarlo ya que esta función es muy útil. Aquí estan
algunos ejemplos.

\begin{Shaded}
\begin{Highlighting}[]
\DecValTok{5}\OperatorTok{:}\DecValTok{10}
\end{Highlighting}
\end{Shaded}

\begin{verbatim}
## [1]  5  6  7  8  9 10
\end{verbatim}

\begin{Shaded}
\begin{Highlighting}[]
\KeywordTok{head}\NormalTok{(iris[, }\KeywordTok{c}\NormalTok{(}\DecValTok{1}\NormalTok{, }\DecValTok{2}\NormalTok{, }\DecValTok{3}\NormalTok{, }\DecValTok{4}\NormalTok{)])}
\end{Highlighting}
\end{Shaded}

\begin{verbatim}
##   Sepal.Length Sepal.Width Petal.Length Petal.Width
## 1          5.1         3.5          1.4         0.2
## 2          4.9         3.0          1.4         0.2
## 3          4.7         3.2          1.3         0.2
## 4          4.6         3.1          1.5         0.2
## 5          5.0         3.6          1.4         0.2
## 6          5.4         3.9          1.7         0.4
\end{verbatim}

\begin{Shaded}
\begin{Highlighting}[]
\KeywordTok{head}\NormalTok{(iris[, }\DecValTok{1}\OperatorTok{:}\DecValTok{4}\NormalTok{]) }\CommentTok{# ;-)}
\end{Highlighting}
\end{Shaded}

\begin{verbatim}
##   Sepal.Length Sepal.Width Petal.Length Petal.Width
## 1          5.1         3.5          1.4         0.2
## 2          4.9         3.0          1.4         0.2
## 3          4.7         3.2          1.3         0.2
## 4          4.6         3.1          1.5         0.2
## 5          5.0         3.6          1.4         0.2
## 6          5.4         3.9          1.7         0.4
\end{verbatim}

\begin{Shaded}
\begin{Highlighting}[]
\NormalTok{miVec01 <-}\StringTok{ }\KeywordTok{c}\NormalTok{(}\DecValTok{1}\NormalTok{, }\DecValTok{2}\NormalTok{, }\DecValTok{3}\NormalTok{, }\DecValTok{4}\NormalTok{)}
\NormalTok{miVec01 <-}\StringTok{ }\DecValTok{1}\OperatorTok{:}\DecValTok{4} \CommentTok{# ;-)}
\OperatorTok{-}\DecValTok{10}\OperatorTok{:}\DecValTok{12}
\end{Highlighting}
\end{Shaded}

\begin{verbatim}
##  [1] -10  -9  -8  -7  -6  -5  -4  -3  -2  -1   0   1   2   3   4   5   6
## [18]   7   8   9  10  11  12
\end{verbatim}

\begin{Shaded}
\begin{Highlighting}[]
\DecValTok{5}\OperatorTok{:-}\DecValTok{5}
\end{Highlighting}
\end{Shaded}

\begin{verbatim}
##  [1]  5  4  3  2  1  0 -1 -2 -3 -4 -5
\end{verbatim}

\begin{Shaded}
\begin{Highlighting}[]
\KeywordTok{paste}\NormalTok{(}\StringTok{"X"}\NormalTok{, }\DecValTok{1}\OperatorTok{:}\DecValTok{10}\NormalTok{, }\DataTypeTok{sep =} \StringTok{"_"}\NormalTok{)}
\end{Highlighting}
\end{Shaded}

\begin{verbatim}
##  [1] "X_1"  "X_2"  "X_3"  "X_4"  "X_5"  "X_6"  "X_7"  "X_8"  "X_9"  "X_10"
\end{verbatim}

\subsection{\texorpdfstring{\texttt{rep()}}{rep()}}\label{l015rep}

La función \texttt{rep()} permite repetir elementos.

\begin{Shaded}
\begin{Highlighting}[]
\NormalTok{miVec12 <-}\StringTok{ }\KeywordTok{c}\NormalTok{(}\DecValTok{1}\NormalTok{, }\DecValTok{1}\NormalTok{, }\DecValTok{1}\NormalTok{, }\DecValTok{1}\NormalTok{, }\DecValTok{1}\NormalTok{, }\DecValTok{1}\NormalTok{, }\DecValTok{1}\NormalTok{, }\DecValTok{1}\NormalTok{, }\DecValTok{1}\NormalTok{)}
\NormalTok{miVec12 <-}\StringTok{ }\KeywordTok{rep}\NormalTok{(}\DecValTok{1}\NormalTok{, }\DataTypeTok{times =} \DecValTok{9}\NormalTok{) }\CommentTok{# ;-)}
\KeywordTok{rep}\NormalTok{(}\StringTok{"Hola"}\NormalTok{, }\DataTypeTok{times =} \DecValTok{3}\NormalTok{)}
\end{Highlighting}
\end{Shaded}

\begin{verbatim}
## [1] "Hola" "Hola" "Hola"
\end{verbatim}

\begin{Shaded}
\begin{Highlighting}[]
\KeywordTok{rep}\NormalTok{(}\DecValTok{1}\OperatorTok{:}\DecValTok{3}\NormalTok{, }\DataTypeTok{time =} \DecValTok{3}\NormalTok{)}
\end{Highlighting}
\end{Shaded}

\begin{verbatim}
## [1] 1 2 3 1 2 3 1 2 3
\end{verbatim}

\begin{Shaded}
\begin{Highlighting}[]
\KeywordTok{rep}\NormalTok{(}\DecValTok{1}\OperatorTok{:}\DecValTok{3}\NormalTok{, }\DataTypeTok{length.out =} \DecValTok{10}\NormalTok{)}
\end{Highlighting}
\end{Shaded}

\begin{verbatim}
##  [1] 1 2 3 1 2 3 1 2 3 1
\end{verbatim}

\begin{Shaded}
\begin{Highlighting}[]
\KeywordTok{rep}\NormalTok{(}\DecValTok{1}\OperatorTok{:}\DecValTok{3}\NormalTok{, }\DataTypeTok{each =} \DecValTok{3}\NormalTok{)}
\end{Highlighting}
\end{Shaded}

\begin{verbatim}
## [1] 1 1 1 2 2 2 3 3 3
\end{verbatim}

\subsection{\texorpdfstring{\texttt{seq()}}{seq()}}\label{l015seq}

La función \texttt{seq()} permite crear una secuencia personalizada.

\begin{Shaded}
\begin{Highlighting}[]
\KeywordTok{seq}\NormalTok{(}\DataTypeTok{from =} \DecValTok{0}\NormalTok{, }\DataTypeTok{to =} \DecValTok{1}\NormalTok{, }\DataTypeTok{by =} \FloatTok{0.2}\NormalTok{)}
\end{Highlighting}
\end{Shaded}

\begin{verbatim}
## [1] 0.0 0.2 0.4 0.6 0.8 1.0
\end{verbatim}

\begin{Shaded}
\begin{Highlighting}[]
\KeywordTok{seq}\NormalTok{(}\DataTypeTok{from =} \DecValTok{20}\NormalTok{, }\DataTypeTok{to =} \DecValTok{10}\NormalTok{, }\DataTypeTok{length.out =} \DecValTok{10}\NormalTok{)}
\end{Highlighting}
\end{Shaded}

\begin{verbatim}
##  [1] 20.00000 18.88889 17.77778 16.66667 15.55556 14.44444 13.33333
##  [8] 12.22222 11.11111 10.00000
\end{verbatim}

\begin{Shaded}
\begin{Highlighting}[]
\NormalTok{letters[}\KeywordTok{seq}\NormalTok{(}\DataTypeTok{from =} \DecValTok{1}\NormalTok{, }\DataTypeTok{to =} \DecValTok{26}\NormalTok{, }\DataTypeTok{by =} \DecValTok{2}\NormalTok{)]}
\end{Highlighting}
\end{Shaded}

\begin{verbatim}
##  [1] "a" "c" "e" "g" "i" "k" "m" "o" "q" "s" "u" "w" "y"
\end{verbatim}

\begin{Shaded}
\begin{Highlighting}[]
\KeywordTok{rep}\NormalTok{(}\KeywordTok{seq}\NormalTok{(}\DataTypeTok{from =} \DecValTok{1}\NormalTok{, }\DataTypeTok{to =} \DecValTok{2}\NormalTok{, }\DataTypeTok{by =} \FloatTok{0.5}\NormalTok{), }\DataTypeTok{times =} \DecValTok{3}\NormalTok{)}
\end{Highlighting}
\end{Shaded}

\begin{verbatim}
## [1] 1.0 1.5 2.0 1.0 1.5 2.0 1.0 1.5 2.0
\end{verbatim}

\subsection{\texorpdfstring{\texttt{getwd()}}{getwd()}}\label{l015getwd}

La función \texttt{getwd()} establece la carpeta de trabajo. Esto
corresponde a la ubicación relativa desde la cual R se posiciona para
identificar los archivos. Este concepto tendrá sentido cuando veamos
cómo importar y exportar datos.

\begin{Shaded}
\begin{Highlighting}[]
\KeywordTok{getwd}\NormalTok{()}
\end{Highlighting}
\end{Shaded}

\begin{verbatim}
## [1] "C:/Users/nous/Documents/Francois/TRAVAIL/GitHub/myRbook_SP"
\end{verbatim}

\subsection{\texorpdfstring{\texttt{setwd()}}{setwd()}}\label{l015setwd}

La función \texttt{setwd()} se usa para definir un nuevo directorio de
trabajo (carpeta de trabajo).

\begin{Shaded}
\begin{Highlighting}[]
\NormalTok{oldWd <-}\StringTok{ }\KeywordTok{getwd}\NormalTok{()}
\KeywordTok{print}\NormalTok{(oldWd)}
\end{Highlighting}
\end{Shaded}

\begin{verbatim}
## [1] "C:/Users/nous/Documents/Francois/TRAVAIL/GitHub/myRbook_SP"
\end{verbatim}

\begin{Shaded}
\begin{Highlighting}[]
\KeywordTok{setwd}\NormalTok{(}\StringTok{".."}\NormalTok{)}
\KeywordTok{getwd}\NormalTok{()}
\end{Highlighting}
\end{Shaded}

\begin{verbatim}
## [1] "C:/Users/nous/Documents/Francois/TRAVAIL/GitHub"
\end{verbatim}

\begin{Shaded}
\begin{Highlighting}[]
\KeywordTok{setwd}\NormalTok{(oldWd)}
\KeywordTok{getwd}\NormalTok{()}
\end{Highlighting}
\end{Shaded}

\begin{verbatim}
## [1] "C:/Users/nous/Documents/Francois/TRAVAIL/GitHub/myRbook_SP"
\end{verbatim}

\subsection{\texorpdfstring{\texttt{list.files()}}{list.files()}}\label{l015listfiles}

La función \texttt{list.files\ ()} se usa para listar todos los archivos
en el directorio de trabajo.

\begin{Shaded}
\begin{Highlighting}[]
\KeywordTok{list.files}\NormalTok{(}\DataTypeTok{pattern =} \StringTok{"(html)$"}\NormalTok{) }\CommentTok{# html}
\end{Highlighting}
\end{Shaded}

\begin{verbatim}
## character(0)
\end{verbatim}

\begin{Shaded}
\begin{Highlighting}[]
\KeywordTok{list.files}\NormalTok{(}\DataTypeTok{pattern =} \StringTok{"(pdf)$"}\NormalTok{) }\CommentTok{# pdf}
\end{Highlighting}
\end{Shaded}

\begin{verbatim}
## character(0)
\end{verbatim}

\subsection{\texorpdfstring{\texttt{ls()}}{ls()}}\label{l015ls}

Al igual que la función \texttt{list.files()} hace posible listar todos
los archivos presentes en el directorio de trabajo, la función
\texttt{ls()} permite listar todos los objetos presentes en el entorno
de trabajo de R.

\begin{Shaded}
\begin{Highlighting}[]
\KeywordTok{ls}\NormalTok{()}
\end{Highlighting}
\end{Shaded}

\begin{verbatim}
##  [1] "aLogic"       "bddInsect"    "characters"   "contrib"     
##  [5] "dfForMat"     "factor01"     "i"            "irisCopy"    
##  [9] "j"            "k"            "logicals"     "mdat"        
## [13] "miArray"      "miArray02"    "miDf01"       "miDfSub01"   
## [17] "miDfSub02"    "miDfSub03"    "miDfSub04"    "miList01"    
## [21] "miList02"     "miList0203"   "miList03"     "miList04"    
## [25] "miList05"     "miList06"     "miMat"        "miMat01"     
## [29] "miMat02"      "miVec01"      "miVec02"      "miVec03"     
## [33] "miVec04"      "miVec05"      "miVec06"      "miVec07"     
## [37] "miVec08"      "miVec09"      "miVec10"      "miVec11"     
## [41] "miVec12"      "miVec13"      "miVec14"      "miVec15"     
## [45] "miVec20"      "miVec21"      "miVec22"      "miVec23"     
## [49] "miVec24"      "miVec25"      "miVecArr"     "miVecArr02"  
## [53] "miVecConf"    "miVecNA"      "miVecOp"      "msg"         
## [57] "myCol"        "myRow"        "myText"       "myText2"     
## [61] "myText3"      "myText4"      "myText5"      "nbrRep"      
## [65] "newVec"       "newVec2"      "numbers"      "oldWd"       
## [69] "opAriDf"      "sumIntDou"    "sumIntInt"    "termino01"   
## [73] "termino02"    "vecForMat01"  "vecForMat02"  "vecForMat03" 
## [77] "vecForMatrix" "vecManip"     "vecManip2"    "vecManip3"
\end{verbatim}

\begin{Shaded}
\begin{Highlighting}[]
\NormalTok{zzz <-}\StringTok{ "a new object"}
\KeywordTok{ls}\NormalTok{()}
\end{Highlighting}
\end{Shaded}

\begin{verbatim}
##  [1] "aLogic"       "bddInsect"    "characters"   "contrib"     
##  [5] "dfForMat"     "factor01"     "i"            "irisCopy"    
##  [9] "j"            "k"            "logicals"     "mdat"        
## [13] "miArray"      "miArray02"    "miDf01"       "miDfSub01"   
## [17] "miDfSub02"    "miDfSub03"    "miDfSub04"    "miList01"    
## [21] "miList02"     "miList0203"   "miList03"     "miList04"    
## [25] "miList05"     "miList06"     "miMat"        "miMat01"     
## [29] "miMat02"      "miVec01"      "miVec02"      "miVec03"     
## [33] "miVec04"      "miVec05"      "miVec06"      "miVec07"     
## [37] "miVec08"      "miVec09"      "miVec10"      "miVec11"     
## [41] "miVec12"      "miVec13"      "miVec14"      "miVec15"     
## [45] "miVec20"      "miVec21"      "miVec22"      "miVec23"     
## [49] "miVec24"      "miVec25"      "miVecArr"     "miVecArr02"  
## [53] "miVecConf"    "miVecNA"      "miVecOp"      "msg"         
## [57] "myCol"        "myRow"        "myText"       "myText2"     
## [61] "myText3"      "myText4"      "myText5"      "nbrRep"      
## [65] "newVec"       "newVec2"      "numbers"      "oldWd"       
## [69] "opAriDf"      "sumIntDou"    "sumIntInt"    "termino01"   
## [73] "termino02"    "vecForMat01"  "vecForMat02"  "vecForMat03" 
## [77] "vecForMatrix" "vecManip"     "vecManip2"    "vecManip3"   
## [81] "zzz"
\end{verbatim}

\subsection{\texorpdfstring{\texttt{rm()}}{rm()}}\label{l015rm}

La función \texttt{rm()} permite eliminar un objeto presente en el
entorno de trabajo de R.

\begin{Shaded}
\begin{Highlighting}[]
\KeywordTok{rm}\NormalTok{(zzz)}
\KeywordTok{ls}\NormalTok{()}
\end{Highlighting}
\end{Shaded}

\begin{verbatim}
##  [1] "aLogic"       "bddInsect"    "characters"   "contrib"     
##  [5] "dfForMat"     "factor01"     "i"            "irisCopy"    
##  [9] "j"            "k"            "logicals"     "mdat"        
## [13] "miArray"      "miArray02"    "miDf01"       "miDfSub01"   
## [17] "miDfSub02"    "miDfSub03"    "miDfSub04"    "miList01"    
## [21] "miList02"     "miList0203"   "miList03"     "miList04"    
## [25] "miList05"     "miList06"     "miMat"        "miMat01"     
## [29] "miMat02"      "miVec01"      "miVec02"      "miVec03"     
## [33] "miVec04"      "miVec05"      "miVec06"      "miVec07"     
## [37] "miVec08"      "miVec09"      "miVec10"      "miVec11"     
## [41] "miVec12"      "miVec13"      "miVec14"      "miVec15"     
## [45] "miVec20"      "miVec21"      "miVec22"      "miVec23"     
## [49] "miVec24"      "miVec25"      "miVecArr"     "miVecArr02"  
## [53] "miVecConf"    "miVecNA"      "miVecOp"      "msg"         
## [57] "myCol"        "myRow"        "myText"       "myText2"     
## [61] "myText3"      "myText4"      "myText5"      "nbrRep"      
## [65] "newVec"       "newVec2"      "numbers"      "oldWd"       
## [69] "opAriDf"      "sumIntDou"    "sumIntInt"    "termino01"   
## [73] "termino02"    "vecForMat01"  "vecForMat02"  "vecForMat03" 
## [77] "vecForMatrix" "vecManip"     "vecManip2"    "vecManip3"
\end{verbatim}

\section{Algunos ejercicios para
practicar}\label{algunos-ejercicios-para-practicar}

Aquí hay algunos ejercicios para mejorar el uso de las funciones y
aprender nuevas gracias a la documentación. Algunos ejercicios son
difíciles, podremos volver a resolverlos más tarde.

\subsection{Secuencias}\label{secuencias}

\subsubsection{Vamos a reproducir las siguientes
secuencias:}\label{vamos-a-reproducir-las-siguientes-secuencias}

-3 -4 -5 -6 -7 -8 -9 -10 --11

-3 -1 1 3 5 7 9 11

3.0 3.2 3.4 3.6 3.8 4.0

20 18 16 14 12 10 8 6

``a'' ``f'' ``k'' ``p'' ``u'' ``z''

``a'' ``a'' ``a'' ``a'' ``a'' ``f'' ``f'' ``f'' ``f'' ``f'' ``k'' ``k''
``k'' ``k'' ``k'' ``p'' ``p'' ``p'' ``p'' ``p'' ``u'' ``u'' ``u'' ``u''
``u'' ``z'' ``z'' ``z'' ``z'' ``z''

\subsubsection{Posibles soluciones (porque siempre hay varias
soluciones):}\label{posibles-soluciones-porque-siempre-hay-varias-soluciones}

\begin{Shaded}
\begin{Highlighting}[]
\OperatorTok{-}\DecValTok{3}\OperatorTok{:-}\DecValTok{11}
\end{Highlighting}
\end{Shaded}

\begin{verbatim}
## [1]  -3  -4  -5  -6  -7  -8  -9 -10 -11
\end{verbatim}

\begin{Shaded}
\begin{Highlighting}[]
\KeywordTok{seq}\NormalTok{(}\DataTypeTok{from =} \OperatorTok{-}\DecValTok{3}\NormalTok{, }\DataTypeTok{to =} \DecValTok{11}\NormalTok{, }\DataTypeTok{by =} \DecValTok{2}\NormalTok{)}
\end{Highlighting}
\end{Shaded}

\begin{verbatim}
## [1] -3 -1  1  3  5  7  9 11
\end{verbatim}

\begin{Shaded}
\begin{Highlighting}[]
\KeywordTok{seq}\NormalTok{(}\DataTypeTok{from =} \FloatTok{3.0}\NormalTok{, }\DataTypeTok{to =} \FloatTok{4.0}\NormalTok{, }\DataTypeTok{by =} \FloatTok{0.2}\NormalTok{)}
\end{Highlighting}
\end{Shaded}

\begin{verbatim}
## [1] 3.0 3.2 3.4 3.6 3.8 4.0
\end{verbatim}

\begin{Shaded}
\begin{Highlighting}[]
\NormalTok{letters[}\KeywordTok{seq}\NormalTok{(}\DataTypeTok{from =} \DecValTok{1}\NormalTok{, }\DataTypeTok{to =} \DecValTok{26}\NormalTok{, }\DataTypeTok{by =} \DecValTok{5}\NormalTok{)]}
\end{Highlighting}
\end{Shaded}

\begin{verbatim}
## [1] "a" "f" "k" "p" "u" "z"
\end{verbatim}

\begin{Shaded}
\begin{Highlighting}[]
\NormalTok{letters[}\KeywordTok{rep}\NormalTok{(}\KeywordTok{seq}\NormalTok{(}\DataTypeTok{from =} \DecValTok{1}\NormalTok{, }\DataTypeTok{to =} \DecValTok{26}\NormalTok{, }\DataTypeTok{by =} \DecValTok{5}\NormalTok{), }\DataTypeTok{each =} \DecValTok{5}\NormalTok{)]}
\end{Highlighting}
\end{Shaded}

\begin{verbatim}
##  [1] "a" "a" "a" "a" "a" "f" "f" "f" "f" "f" "k" "k" "k" "k" "k" "p" "p"
## [18] "p" "p" "p" "u" "u" "u" "u" "u" "z" "z" "z" "z" "z"
\end{verbatim}

\subsection{Estadísticas
descriptivas}\label{estadisticas-descriptivas-1}

En el conjunto de datos \texttt{iris}, ¿cuántos valores de ancho del
sépalo son mayores que 3? Entre 2.8 y 3.2?

¿Cómo se puede visualizar la distribución de datos (función
\texttt{table()})?

¿Cuáles son los 10 valores más pequeños?

¿Cómo se calcula el intervalo que contiene el 90\% de los valores?

Si la distribución de los datos era Normal, ¿cuál sería el valor teórico
de este intervalo del 90\% (función \texttt{qnorm()})?

Soluciones:

\begin{Shaded}
\begin{Highlighting}[]
\KeywordTok{length}\NormalTok{(iris}\OperatorTok{$}\NormalTok{Sepal.Width[iris}\OperatorTok{$}\NormalTok{Sepal.Width }\OperatorTok{>}\StringTok{ }\DecValTok{3}\NormalTok{])}
\end{Highlighting}
\end{Shaded}

\begin{verbatim}
## [1] 67
\end{verbatim}

\begin{Shaded}
\begin{Highlighting}[]
\KeywordTok{length}\NormalTok{(iris}\OperatorTok{$}\NormalTok{Sepal.Width[iris}\OperatorTok{$}\NormalTok{Sepal.Width }\OperatorTok{>}\StringTok{ }\FloatTok{2.8} \OperatorTok{&}\StringTok{ }
\StringTok{  }\NormalTok{iris}\OperatorTok{$}\NormalTok{Sepal.Width }\OperatorTok{<}\StringTok{ }\FloatTok{3.2}\NormalTok{])}
\end{Highlighting}
\end{Shaded}

\begin{verbatim}
## [1] 47
\end{verbatim}

\begin{Shaded}
\begin{Highlighting}[]
\KeywordTok{table}\NormalTok{(iris}\OperatorTok{$}\NormalTok{Sepal.Width)}
\end{Highlighting}
\end{Shaded}

\begin{verbatim}
## 
##   2 2.2 2.3 2.4 2.5 2.6 2.7 2.8 2.9   3 3.1 3.2 3.3 3.4 3.5 3.6 3.7 3.8 
##   1   3   4   3   8   5   9  14  10  26  11  13   6  12   6   4   3   6 
## 3.9   4 4.1 4.2 4.4 
##   2   1   1   1   1
\end{verbatim}

\begin{Shaded}
\begin{Highlighting}[]
\KeywordTok{table}\NormalTok{(}\KeywordTok{round}\NormalTok{(iris}\OperatorTok{$}\NormalTok{Sepal.Width))}
\end{Highlighting}
\end{Shaded}

\begin{verbatim}
## 
##   2   3   4 
##  19 106  25
\end{verbatim}

\begin{Shaded}
\begin{Highlighting}[]
\NormalTok{irisSepWCopy <-}\StringTok{ }\NormalTok{iris}\OperatorTok{$}\NormalTok{Sepal.Width}
\NormalTok{irisSepWCopy <-}\StringTok{ }\NormalTok{irisSepWCopy[}\KeywordTok{order}\NormalTok{(irisSepWCopy)]}
\KeywordTok{head}\NormalTok{(irisSepWCopy, }\DataTypeTok{n =} \DecValTok{10}\NormalTok{)}
\end{Highlighting}
\end{Shaded}

\begin{verbatim}
##  [1] 2.0 2.2 2.2 2.2 2.3 2.3 2.3 2.3 2.4 2.4
\end{verbatim}

\begin{Shaded}
\begin{Highlighting}[]
\KeywordTok{quantile}\NormalTok{(irisSepWCopy, }\DataTypeTok{probs =} \KeywordTok{c}\NormalTok{(}\FloatTok{0.05}\NormalTok{, }\FloatTok{0.95}\NormalTok{))}
\end{Highlighting}
\end{Shaded}

\begin{verbatim}
##    5%   95% 
## 2.345 3.800
\end{verbatim}

\begin{Shaded}
\begin{Highlighting}[]
\KeywordTok{qnorm}\NormalTok{(}
  \DataTypeTok{p =} \KeywordTok{c}\NormalTok{(}\FloatTok{0.05}\NormalTok{, }\FloatTok{0.95}\NormalTok{), }
  \DataTypeTok{mean =} \KeywordTok{mean}\NormalTok{(irisSepWCopy), }
  \DataTypeTok{sd =} \KeywordTok{sd}\NormalTok{(irisSepWCopy)}
\NormalTok{)}
\end{Highlighting}
\end{Shaded}

\begin{verbatim}
## [1] 2.340397 3.774270
\end{verbatim}

\section{Escribir una función}\label{escribir-una-funcion}

XXX

\section{Conclusión}\label{conclusion-4}

\chapter{Importar y exportar datos}\label{import}

\section{Leer datos de un archivo}\label{leer-datos-de-un-archivo}

XXX

\section{Guardar datos para R}\label{guardar-datos-para-r}

save load

\section{Exportar datos}\label{exportar-datos}

write XXX

\chapter{Los bucles}\label{loops}

\section{¿Por qué hacer bucles?}\label{por-que-hacer-bucles}

XXX algorítmico \ldots{}

\section{\texorpdfstring{El bucle
\texttt{if}}{El bucle if}}\label{el-bucle-if}

XXX

\section{\texorpdfstring{El bucle
\texttt{switch}}{El bucle switch}}\label{el-bucle-switch}

XXX

\section{\texorpdfstring{El bucle
\texttt{for}}{El bucle for}}\label{el-bucle-for}

XXX

\section{\texorpdfstring{El bucle
\texttt{while}}{El bucle while}}\label{el-bucle-while}

XXX

\section{\texorpdfstring{\texttt{repeat}, \texttt{next}, \texttt{break},
\texttt{stop}}{repeat, next, break, stop}}\label{repeat-next-break-stop}

XXX

\section{\texorpdfstring{Los bucles de la familia
\texttt{apply}}{Los bucles de la familia apply}}\label{los-bucles-de-la-familia-apply}

\subsection{\texorpdfstring{\texttt{apply}}{apply}}\label{apply}

XXX

\subsection{\texorpdfstring{\texttt{sapply}}{sapply}}\label{sapply}

XXX

\subsection{\texorpdfstring{\texttt{lapply}}{lapply}}\label{lapply}

XXX

\subsection{\texorpdfstring{\texttt{tapply}}{tapply}}\label{tapply}

XXX

\subsection{\texorpdfstring{\texttt{mapply}}{mapply}}\label{mapply}

XXX

\part{Los gráficos}\label{part-los-graficos}

\chapter{Gráficos simples}\label{graph1}

\section{plot}\label{plot}

\section{hist}\label{hist}

\section{barplot}\label{barplot}

\section{boxplot}\label{boxplot}

\section{image y contour}\label{image-y-contour}

\chapter{Gestión del color}\label{graph2}

\section{colors()}\label{colors}

\section{RGB}\label{rgb}

\section{Paletas}\label{paletas}

\chapter{Gráficos compuestos}\label{graph3}

\section{mfrow}\label{mfrow}

\section{layout}\label{layout}

\chapter{Manipular gráficos}\label{graph4}

\section{Inkscape}\label{inkscape}

\section{The Gimp}\label{the-gimp}

\part{Estadísticas con R}\label{part-estadisticas-con-r}

\chapter{Estadísticas descriptivas}\label{stats1}

\part{Estudio de caso}\label{part-estudio-de-caso}

\chapter{Analizar datos de loggers de temperatura}\label{studyCase1}


\end{document}
